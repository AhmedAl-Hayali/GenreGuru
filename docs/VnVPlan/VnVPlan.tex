\documentclass[12pt, titlepage]{article}

\usepackage{booktabs}
\usepackage{tabularx}
\usepackage{hyperref}
\usepackage{amssymb}
\usepackage{enumitem}
\hypersetup{
    colorlinks,
    citecolor=blue,
    filecolor=black,
    linkcolor=red,
    urlcolor=blue
}
\usepackage[round]{natbib}
\usepackage{enumitem}

%% Comments

\usepackage{color}

\newif\ifcomments\commentstrue %displays comments
%\newif\ifcomments\commentsfalse %so that comments do not display

\ifcomments
\newcommand{\authornote}[3]{\textcolor{#1}{[#3 ---#2]}}
\newcommand{\todo}[1]{\textcolor{red}{[TODO: #1]}}
\else
\newcommand{\authornote}[3]{}
\newcommand{\todo}[1]{}
\fi

\newcommand{\wss}[1]{\authornote{blue}{SS}{#1}} 
\newcommand{\plt}[1]{\authornote{magenta}{TPLT}{#1}} %For explanation of the template
\newcommand{\an}[1]{\authornote{cyan}{Author}{#1}}

%% Common Parts

\newcommand{\progname}{Software Engineering} % PUT YOUR PROGRAM NAME HERE
\newcommand{\authname}{Team 8 -- Rhythm Rangers\\
\\ Ansel Chen
\\ Muhammad Jawad
\\ Mohamad-Hassan Bahsoun
\\ Matthew Baleanu
\\ Ahmed Al-Hayali} % AUTHOR NAMES                  

\usepackage{hyperref}
    \hypersetup{colorlinks=true, linkcolor=blue, citecolor=blue, filecolor=blue,
                urlcolor=blue, unicode=false}
    \urlstyle{same}
                                


\begin{document}

\title{System Verification and Validation Plan for \progname{}} 
\author{\authname}
\date{\today}
	
\maketitle

\pagenumbering{roman}

\section*{Revision History}

\begin{tabularx}{\textwidth}{p{3cm}p{2cm}X}
\toprule {\bf Date} & {\bf Version} & {\bf Notes}\\
\midrule
Date 1 & 1.0 & Notes\\
Date 2 & 1.1 & Notes\\
\bottomrule
\end{tabularx}

~\\
\wss{The intention of the VnV plan is to increase confidence in the software.
However, this does not mean listing every verification and validation technique
that has ever been devised.  The VnV plan should also be a \textbf{feasible}
plan. Execution of the plan should be possible with the time and team available.
If the full plan cannot be completed during the time available, it can either be
modified to ``fake it'', or a better solution is to add a section describing
what work has been completed and what work is still planned for the future.}

\wss{The VnV plan is typically started after the requirements stage, but before
the design stage.  This means that the sections related to unit testing cannot
initially be completed.  The sections will be filled in after the design stage
is complete.  the final version of the VnV plan should have all sections filled
in.}

\newpage

\tableofcontents

\listoftables
\wss{Remove this section if it isn't needed}

\listoffigures
\wss{Remove this section if it isn't needed}

\newpage

\section{Symbols, Abbreviations, and Acronyms}

\renewcommand{\arraystretch}{1.2}
\begin{tabular}{l l} 
  \toprule		
  \textbf{symbol} & \textbf{description}\\
  \midrule 
  T & Test\\
  \bottomrule
\end{tabular}\\

\wss{symbols, abbreviations, or acronyms --- you can simply reference the SRS
  \citep{SRS} tables, if appropriate}

\wss{Remove this section if it isn't needed}

\newpage

\pagenumbering{arabic}

This document details the Verification and Validation plan, outlining the strategies, 
tools, and processes to ensure the system meets its design specifications and quality standards. 
Automated testing tools, including linters, unit testing frameworks, and continuous integration 
(e.g., GitHub Actions), will maintain code quality and enable efficient verification of functionality. 
The roadmap covers key phases: initial setup and planning to define testing tools and team responsibilities; 
systematic unit, integration, and automated testing to validate module interactions and overall functionality; 
structured usability assessments to gauge user satisfaction; and a final verification phase to ensure all functional 
and non-functional requirements are met. This phased approach, combined with both automated and manual validation 
techniques, aligns the project with its technical, usability, and compliance objectives.

\section{General Information}

\subsection{Summary}

\wss{Say what software is being tested.  Give its name and a brief overview of
  its general functions.}

  \subsection{Objectives}

  The objectives of this Verification and Validation (VnV) Plan are twofold: to provide confidence that the project satisfies all specified functional and nonfunctional requirements (verification) and to confirm that the end product aligns with user expectations and project goals (validation).
  
\begin{description}[leftmargin=0cm]
  \item[Verification] Ensuring the product is built correctly according to the defined requirements and specifications, i.e.,
  \begin{itemize}
    \item Assessing each atomic item (unit testing) to confirm it functions as intended.
    \item Testing interactions between atomic items (integration testing) to verify that components work together seamlessly.
    \item Performing end-to-end (system) testing to confirm that the entire system functions correctly and meets project specifications.
    \item Applying Test-Driven Development (TDD) practices where feasible to enhance code reliability from early stages.
    \item Refactoring code regularly to eliminate code smells and maintain clean, efficient, and maintainable code.
  \end{itemize}
  \item[Validation] Ensuring the project meets needs of the stakeholders within the intended environment, typically through acceptance testing. This process is often performed by individuals external to the project team.
\end{description}

\subsection{Challenge Level and Extras}

\begin{description}[leftmargin=0cm]
  \item[Challenge Level] As stated in the Problem Statement document, the challenge level of this project is general. In light of the revised scope of the project, i.e., omitting the generation component and delaying the recommendation component until after acceptable completion of the featurizer, as described in the hazard analysis document, we have not changed the challenge level of our project. In retrospect, perhaps it was likely the case that our project should have been labelled as a challenging project instead.
  \item[Extras] As stated in the probelm statement document, the team plans to include user \& API documentation for reference, usability testing for easy startup, and design thinking to build an intuitive user interface. Note that the generation component is now a stretch goal instead of a core feature, as discussed in the hazard analysis document.
\end{description}

\subsection{Relevant Documentation}

\wss{Reference relevant documentation.  This will definitely include your SRS
  and your other project documents (design documents, like MG, MIS, etc).  You
  can include these even before they are written, since by the time the project
  is done, they will be written.  You can create BibTeX entries for your
  documents and within those entries include a hyperlink to the documents.}

\citet{SRS}

\wss{Don't just list the other documents.  You should explain why they are relevant and 
how they relate to your VnV efforts.}

\section{Plan}

\wss{Introduce this section.  You can provide a roadmap of the sections to
  come.}

\subsection{Verification and Validation Team}

\wss{Your teammates.  Maybe your supervisor.
  You should do more than list names.  You should say what each person's role is
  for the project's verification.  A table is a good way to summarize this information.}

\subsection{SRS Verification Plan}
The following approaches will be used for SRS verification:
\begin{itemize}
  \item Formal reviews with the supervisor
  \item A checklist that will be given to the supervisor and any peer reviewers. It will also
  serve as a guide for the developers of the system
  \item Using feedback from grading to create new checklists and update existing checklists
  \item Ad-hoc reviews from peers and other teams in the course
\end{itemize}
This is the initial SRS checklist that reviewers will use. It will be updated as reviews are performed:
\begin{enumerate}[label=$\square$]
  \item Does each functional requirement have a detailed and accurate description, rationale and fit criteria?
  \item Is each requirement (both functional and non-functional) relevant and necessary?
  \item Are all functional requirements traceable to at least one use case?
  \item Are all fit critera unambiguous and verifiable?
  \item Have all issues opened by reviewers been closed? 
\end{enumerate}

\subsection{Design Verification Plan}

The design verification will ensure that each component of the system meets the requirements outlined in the Software Requirements Specification (SRS). Verification will occur through systematic reviews, testing, and validation checklists tailored to each component.

\subsubsection{Client Application Design Verification}
User-facing component for inputting track information and displaying system output.
\begin{description}
  \item[Verification Approach] The verification will proceed as we:
  
  \begin{itemize}[leftmargin=0cm]
    \item Conduct usability testing with users to assess ease of input and clarity of output.
    \item Review design documents to ensure UI/UX aligns with requirements.
  \end{itemize}
  \item[Checklist] The design is considered verified if the:
  
    \begin{itemize}[label=$\square$,leftmargin=0cm]
    \item User can input track information without errors.
    \item System output is clear and comprehensible.
    \item User feedback is collected and assessed.
  \end{itemize}
\end{description}

\subsubsection{Server Networking Design Verification}
Represents communication between the server and external components, e.g., between the client application and external service APIs.
\begin{description}
  \item[Verification Approach] The verification will proceed as we:
  \begin{itemize}[leftmargin=0cm]
    \item Review network architecture to confirm it supports required protocols.
    \item Test for successful data transmission between components.
  \end{itemize}
  \item[Checklist] The design is considered verified if:
    \begin{itemize}[label=$\square$,leftmargin=0cm]
    \item All intended connections are established.
    \item Data loss during transmission is within acceptable limits.
    \item Network security measures are implemented and verified.
  \end{itemize}
\end{description}

\subsubsection{Server Compute Design Verification}
Component responsible for processing user requests and issuing featurization and data access requests.
\begin{description}
  \item[Verification Approach] The verification will proceed as we:
  \begin{itemize}[leftmargin=0cm]
    \item Review processing algorithms for efficiency.
    \item Conduct code reviews to ensure best practices are followed.
  \end{itemize}
  \item[Checklist] The design is considered verified if the:
    \begin{itemize}[label=$\square$,leftmargin=0cm]
    \item Processing time is optimized for expected load.
    \item Code is clean and adheres to project standards.
    \item Edge cases are handled appropriately.
  \end{itemize}
\end{description}

\subsubsection{Server Storage Design Verification}
Represents the database for storing data.
\begin{description}
  \item[Verification Approach] The verification will proceed as we:
  \begin{itemize}[leftmargin=0cm]
    \item Review database schema against the SRS.
    \item Test data retrieval and storage for accuracy.
  \end{itemize}
  \item[Checklist] The design is considered verified if the:
    \begin{itemize}[label=$\square$,leftmargin=0cm]
    \item Schema supports all required data types and relationships.
    \item Data retrieval times meet performance criteria.
    \item Data integrity is maintained during transactions.
  \end{itemize}
\end{description}

\subsubsection{External Service APIs Design Verification}
Represents integration with external services like Spotify or Deezer.
\begin{description}
  \item[Verification Approach] The verification will proceed as we:
  \begin{itemize}[leftmargin=0cm]
    \item Review API documentation for compliance.
    \item Test integration for successful data exchange.
  \end{itemize}
  \item[Checklist] The design is considered verified if:
    \begin{itemize}[label=$\square$,leftmargin=0cm]
    \item All API endpoints are functional as expected.
    \item Data that is exchanged is accurate and formatted correctly.
    \item Error handling for API failures is in place.
  \end{itemize}
\end{description}

\subsubsection{Featurizer Design Verification}
Handles the featurization process.
\begin{description}
  \item[Verification Approach] The verification will proceed as we:
  \begin{itemize}[leftmargin=0cm]
    \item Review featurization algorithms for correctness.
    \item Test output against known input cases to validate accuracy.
  \end{itemize}
  \item[Checklist] The design is considered verified if:
    \begin{itemize}[label=$\square$,leftmargin=0cm]
    \item All features are correctly extracted from input data.
    \item Featurization time is within acceptable limits.
    \item Results match expected outputs for test cases.
  \end{itemize}
\end{description}

A group discussion will be held bi-weekly to review the verification process and address any outstanding issues. Feedback from classmates will be incorporated throughout the development process, especially during review sessions.

\subsection{Verification and Validation Plan Verification Plan}
% \wss{The verification and validation plan is an artifact that should also be
% verified.  Techniques for this include review and mutation testing.}

% \wss{The review will include reviews by your classmates}

% \wss{Create a checklists?}
The primary method of verification for the VnV plan should be reviews and mutation testing. The main goals these reviews 
are used for is to capture how complete the test cases are and if they are correct. A checklist would be provided
to the reviewers in order to make their job simpler. To these ends, we would like to use 
the following methods to review the VnV plan: 

\begin{itemize}
  \item Classmate reviews: The VnV plan will be reviewed by another group in order to identify any gaps or inconsistencies. 
  As we cannot expect other classmates to fully grasp the full nature of our project, we mainly expect the peer review to be 
  convering consistency issues and sanity checking the test cases. 
  \item TA review: The course instructors will review the VnV plan to check if it meets the rubric. The rubric would be turned into a 
  checklist for the TA and as the course instructors would understand the project better than the peer reviewers, they could help 
  determine whether the test cases fully cover all the requirements and if the plan is feasible within the capstone timeline. 
  \item Mutation testing: this method would be used to test the quality of our test cases. By introducing small changes to the code 
  of the project, we would be able to test if the resulting output would be the different under the same inputs, thus allowing us to 
  verify whether the test cases that were derived during the VnV plan are truly capable of detecting faults or not.
  \item Internal team review: Our current process assigns at least one reviewer for each new section before committing changes to the VnV plan.
  This practice ensures that updates align with the overall project requirements and verifies any new test cases or methodologies introduced.

\end{itemize}

\subsection{Implementation Verification Plan}
\subsubsection{Static Analysis}
Static analysis is the analysis of program content without execution. Below is a description of static analysis techniques we plan to implement as part of our testing process.

\begin{description}[style=unboxed,leftmargin=0cm]
  \item[Linting] Linters like \href{https://pylint.readthedocs.io/}{\texttt{Pylint}}, \href{https://docs.astral.sh/ruff/}{\texttt{Ruff}}, or \href{https://flake8.pycqa.org/en/stable/}{\texttt{Flake8}} check for errors, enforce a coding standard, identify \href{https://refactoring.guru/refactoring/smells}{code smells}, and can make code refactoring suggestions.
  
  \item[Formatting] Formatters like \href{https://black.readthedocs.io/}{\texttt{Black}} and \texttt{Ruff} (indeed, Ruff is also a formatter) standardize code appearance and adhere to style guides, allowing the code reader to focus on code content.
  
  \item[Type Checking] Type checkers like \href{https://mypy.readthedocs.io/}{\texttt{mypy}} ensure correct use of variables and functions in code using type hints, as outlined in \href{https://peps.python.org/pep-484/}{PEP 484}. Type hinting can also serve as documentation when publishing an API reference or developer guide.
  
  \item[Security Checking] Static security checkers like \href{https://bandit.readthedocs.io/}{\texttt{Bandit}} find common security vulnerabilities in code, e.g., framework misconfiguration (\texttt{B2XX}, e.g., exposing the Flask debugger in a production application, allowing \href{https://flask.palletsprojects.com/en/stable/quickstart/#debug-mode:~:text=Warning}{remote code execution}), blacklist calls (\href{https://bandit.readthedocs.io/en/latest/blacklists/blacklist_calls.html}{\texttt{B3XX}}, e.g., loading serialized pickle files), blacklist imports (\href{https://bandit.readthedocs.io/en/latest/blacklists/blacklist_imports.html}{\texttt{B4XX}}, e.g., importing \texttt{ftplib} for insecure file transfer), cryptography (\texttt{B5XX}, e.g., \href{https://bandit.readthedocs.io/en/latest/plugins/b501_request_with_no_cert_validation.html}{missing certificate validation}), and injection (\texttt{B6XX}, e.g., testing for \href{https://bandit.readthedocs.io/en/latest/plugins/b608_hardcoded_sql_expressions.html}{SQL injection}).
  
  \item[Code Metrics Analysis] A code metrics analysis tool like \href{https://radon.readthedocs.io/en/stable/index.html}{\texttt{radon}} can provide insights on various aspects of the codebase:
    \begin{description}[style=unboxed,labelindent=1cm]
      \item[Raw metrics] Number of lines of source code (SLOC), logic (LLOC), comments, and whitespace.
      \item[Cyclomatic complexity] Number of decisions (or linearly independent paths) in a code block.
      \item[Halstead metrics] Statically-generated program \href{https://radon.readthedocs.io/en/stable/intro.html#halstead-metrics}{metrics}.
    \end{description}
  
  \item[Composite Analysis Techniques] Tools like \href{https://radon.readthedocs.io/en/stable/index.html}{\texttt{Prospector}} combine multiple analysis techniques into one, i.e., linting via \texttt{Pylint}, \href{https://launchpad.net/pyflakes}{\texttt{Pyflakes}}, or \texttt{Ruff}, \href{https://peps.python.org/pep-8}{PEP 8} and PEP 257 formatting via \href{https://pycodestyle.pycqa.org/en/stable/}{\texttt{pycodestyle}} and \href{https://www.pydocstyle.org/en/stable/}{\texttt{pydocstyle}}, code complexity analysis via \href{https://flake8.pycqa.org/en/stable/}{\texttt{McCabe}}, simple security checking via \href{https://github.com/prospector-dev/dodgy}{\texttt{Dodgy}}, packaging quality checking via \href{https://github.com/regebro/pyroma}{\texttt{Pyroma}}, unused modules checking via \href{https://github.com/jendrikseipp/vulture}{\texttt{Vulture}}, type checking via \texttt{Mypy} or \href{https://microsoft.github.io/pyright/}{\texttt{Pyright}}, and security checking via \texttt{Bandit}.
  
  \item[Code Walkthroughs] Checklist-driven walkthroughs of featurization algorithms with other teammates and optionally the supervisor. Refer to the appendix section \ref{A-Code-walkthrough-checklist} for a sample checklist.
  
  \item[Peer Desk Checks] Changes to the codebase will require approval from at least one other group member via a pull request review before merging to the main/production branch.
\end{description}

\subsubsection{Dynamic Testing}
Dynamic testing is the analysis of program runtime responses during and after execution. Below is a description of dynamic testing techniques we plan to implement as part of our testing process. For further details about the automated components, refer to section \ref{3.6-auto-test-verif-tools}.

\begin{description}[style=unboxed,leftmargin=0cm]
  \item[System Testing] Tests are orchestrated via Testing orchestration tools like \href{https://tox.wiki/}{\texttt{tox}} can manage all testing, from atomic unit tests to end-to-end system tests. System tests outlined in section 4 will be run to ensure necessary requirements are met.
  \item[Unit Testing] Unit testing frameworks like \href{https://docs.pytest.org/en/stable/}{\texttt{pytest}} or \href{https://docs.python.org/3/library/unittest}{\texttt{unittest}} can verify that the implementation matches designs described in other system documents.
  % \href{https://realpython.com/python-testing/#testing-data-driven-applications}{Guide to testing data-driven applications}. \href{https://coderpad.io/blog/development/a-guide-to-database-unit-testing-with-pytest-and-sqlalchemy/}{Guide to testing databases with \texttt{pytest}}.
  \item[User Interface Testing] UI components can be described as a set of discrete interactions, i.e., a transition model can capture user interactions as events then test it. Tools like the \href{https://selenium-python.readthedocs.io/}{\texttt{Selenium} Python API} automate web-based interactions, i.e., can serve as a testing framework to automate interaction sequences using the UI model.
  % If using the MVC/PAC architecture, we can test the M/P and V/A separately using stubs/mocks.
  \item[Integration Testing] Testing frameworks like \texttt{pytest} can be combined with modular \href{https://docs.pytest.org/explanation/fixtures.html}{fixtures} or factories via \href{https://factoryboy.readthedocs.io/}{\texttt{factoryboy}} to simulate databases or other complex objects for testing operability between various interfaces, i.e., integration testing.
  \item[Regression Testing] Persistence of tests across iterations of the project facilitates ease of regression testing, with automation via GitHub actions.
  \item[Coverage Testing] Code coverage libraries like \href{https://coverage.readthedocs.io/en/7.6.4/}{\texttt{coverage}} or \href{https://pypi.org/project/pytest-cov/}{\texttt{pytest-cov}} can be used to generate code coverage reports. These reports will inform developers of any possible code execution paths that have not been covered by unit tests.
\end{description}
% \item \textcolor{red}{Ownership of code/data - likely algorithm person}
% \item Regression testing w/ unit tests; \textcolor{red}{stubs/mocks in place of other classes?}; compose, don't inherit

\subsection{Automated Testing and Verification Tools} \label{3.6-auto-test-verif-tools}

\wss{What tools are you using for automated testing.  Likely a unit testing
  framework and maybe a profiling tool, like ValGrind.  Other possible tools
  include a static analyzer, make, continuous integration tools, test coverage
  tools, etc.  Explain your plans for summarizing code coverage metrics.
  Linters are another important class of tools.  For the programming language
  you select, you should look at the available linters.  There may also be tools
  that verify that coding standards have been respected, like flake9 for
  Python.}

\wss{If you have already done this in the development plan, you can point to
that document.}

\wss{The details of this section will likely evolve as you get closer to the
  implementation.}

\subsection{Software Validation Plan}

% \wss{If there is any external data that can be used for validation, you should
%   point to it here.  If there are no plans for validation, you should state that
%   here.}

% \wss{You might want to use review sessions with the stakeholder to check that
% the requirements document captures the right requirements.  Maybe task based
% inspection?}

% \wss{For those capstone teams with an external supervisor, the Rev 0 demo should 
% be used as an opportunity to validate the requirements.  You should plan on 
% demonstrating your project to your supervisor shortly after the scheduled Rev 0 demo.  
% The feedback from your supervisor will be very useful for improving your project.}

% \wss{For teams without an external supervisor, user testing can serve the same purpose 
% as a Rev 0 demo for the supervisor.}

% \wss{This section might reference back to the SRS verification section.}

\subsubsection{External Data For Validation}
The project cannot function very effectively without access to songs made available by \href{https://developers.deezer.com/api}{Deezer's API}, song snippets made available by \href{https://developer.spotify.com/documentation/web-api}{Spotify's API}, or the \href{https://developer.spotify.com/documentation/web-api/reference/get-several-audio-features}{track audio features} offered by Spotify's API that can serve as the ground truth for us to verify \href{https://en.wikipedia.org/wiki/Probably_approximately_correct_learning#:~:text=In%20this%20framework,the%20samples.}{approximate correctness} of our featurizer.

\subsubsection{Requirement Reviews \& Task-Based Inspections}
Task-based inspections seek to verify the implementation of functional requirements and the degree to which the nonfunctional requirements are met. Requirement reviews, however, are conducted with the stakeholders to select and refine a subset of requirements.
\begin{itemize}
\item Requirement reviews with stakeholders can be conducted to offer assurance on the completeness of requirements documents. The goal is to survey stakeholders on the requirements to help with requirement prioritization, refinement, or omission.

\item A task-based inspection can also be conducted to verify the correctness of the implementation with respect to the requirements. This would involve stakeholder being given a list of tasks designed to test functional and nonfunctional requirments. The stakeholder would document their experience carrying out these tasks. Ideally, the task-based inspection is conducted with the same stakeholder group as the one involved in the requirements review to obtain a revised opinion on the selection and granularity of requirements.
\end{itemize}

\subsubsection{Project Supervisor Demo}
The project supervisor is Dr. Martin V. Mohrenschildt. During the rev 0 demo, we should explain and justify the project requirements, followed by a demonstration of a project prototype, then a characterization of the correctness of the functional requirements implementation and adherance to the nonfunctional requirements. Unmet requirements must be revised or omitted with justification. Note that Dr. Martin V. Mohrenschildt is an expert in signals processing, thus we strive primarily to collect feedback on our signal processing components.

\section{System Tests}

\wss{There should be text between all headings, even if it is just a roadmap of
the contents of the subsections.}

\subsection{Tests for Functional Requirements}

\wss{Subsets of the tests may be in related, so this section is divided into
  different areas.  If there are no identifiable subsets for the tests, this
  level of document structure can be removed.}

\wss{Include a blurb here to explain why the subsections below
  cover the requirements.  References to the SRS would be good here.}

\subsubsection{Area of Testing1}

\wss{It would be nice to have a blurb here to explain why the subsections below
  cover the requirements.  References to the SRS would be good here.  If a section
  covers tests for input constraints, you should reference the data constraints
  table in the SRS.}
		
\paragraph{Title for Test}

\begin{enumerate}

\item{test-id1\\}

Control: Manual versus Automatic
					
Initial State: 
					
Input: 
					
Output: \wss{The expected result for the given inputs.  Output is not how you
are going to return the results of the test.  The output is the expected
result.}

Test Case Derivation: \wss{Justify the expected value given in the Output field}
					
How test will be performed: 
					
\item{test-id2\\}

Control: Manual versus Automatic
					
Initial State: 
					
Input: 
					
Output: \wss{The expected result for the given inputs}

Test Case Derivation: \wss{Justify the expected value given in the Output field}

How test will be performed: 

\end{enumerate}

\subsubsection{Area of Testing2}

...

\subsection{Tests for Nonfunctional Requirements}

\subsubsection{APR1 - Minimalist Layout}
\begin{itemize}
    \item \textbf{Test ID:} TAPR1
    \item \textbf{Type:} Static, Manual
    \item \textbf{Initial State:} User interface loaded.
    \item \textbf{Input/Condition:} Visual inspection of layout.
    \item \textbf{Output/Result:} Interface presents a minimalist layout with minimal distractions.
    \item \textbf{How test will be performed:} Manually inspect interface layout to ensure it follows minimalist design guidelines.
\end{itemize}

\subsubsection{APR2 - High Contrast}
\begin{itemize}
    \item \textbf{Test ID:} TAPR2
    \item \textbf{Type:} Static, Manual
    \item \textbf{Initial State:} Interface set to default theme.
    \item \textbf{Input/Condition:} Check for visual contrast.
    \item \textbf{Output/Result:} All text and elements display high contrast for readability.
    \item \textbf{How test will be performed:} Perform a manual inspection of UI contrast using WCAG contrast standards.
\end{itemize}

\subsubsection{APR3 - Intuitive Navigation}
\begin{itemize}
    \item \textbf{Test ID:} TAPR3
    \item \textbf{Type:} Dynamic, Manual
    \item \textbf{Initial State:} System interface opened.
    \item \textbf{Input/Condition:} Navigate through different pages.
    \item \textbf{Output/Result:} Users can easily navigate and locate functions within 3 clicks.
    \item \textbf{How test will be performed:} Manually navigate to different features and confirm efficient accessibility.
\end{itemize}

\subsubsection{STR1 - Consistent Button Styles}
\begin{itemize}
    \item \textbf{Test ID:} TSTR1
    \item \textbf{Type:} Static, Manual
    \item \textbf{Initial State:} Interface loaded.
    \item \textbf{Input/Condition:} Check for style consistency in buttons.
    \item \textbf{Output/Result:} All buttons follow a consistent color and shape style.
    \item \textbf{How test will be performed:} Visually inspect all buttons to ensure they meet the style guidelines.
\end{itemize}

\subsubsection{EUR1 - Tooltip Visibility}
\begin{itemize}
    \item \textbf{Test ID:} TEUR1
    \item \textbf{Type:} Dynamic, Manual
    \item \textbf{Initial State:} Interface loaded with tooltips.
    \item \textbf{Input/Condition:} Hover over interactive elements.
    \item \textbf{Output/Result:} Tooltips display with descriptive content.
    \item \textbf{How test will be performed:} Manually hover over elements to confirm tooltip visibility.
\end{itemize}

\subsubsection{PIR1 - Customizable Color Themes}
\begin{itemize}
    \item \textbf{Test ID:} TPIR1
    \item \textbf{Type:} Dynamic, Manual
    \item \textbf{Initial State:} Interface with theme options.
    \item \textbf{Input/Condition:} User switches between themes.
    \item \textbf{Output/Result:} All themes display correctly with no visual errors.
    \item \textbf{How test will be performed:} Switch themes manually and verify consistent color scheme application.
\end{itemize}

\subsubsection{LR1 - Initial Tutorial}
\begin{itemize}
    \item \textbf{Test ID:} TLR1
    \item \textbf{Type:} Dynamic, Manual
    \item \textbf{Initial State:} First-time user experience loaded.
    \item \textbf{Input/Condition:} User accesses the system for the first time.
    \item \textbf{Output/Result:} System displays an introductory tutorial.
    \item \textbf{How test will be performed:} Manually confirm tutorial launches on initial access.
\end{itemize}

\subsubsection{LR2 - Tutorial Completion Time}
\begin{itemize}
    \item \textbf{Test ID:} TLR2
    \item \textbf{Type:} Dynamic, Manual
    \item \textbf{Initial State:} Tutorial in progress.
    \item \textbf{Input/Condition:} Measure time for tutorial completion.
    \item \textbf{Output/Result:} Users complete the tutorial within 5 minutes.
    \item \textbf{How test will be performed:} Track completion time for new users.
\end{itemize}

\subsubsection{UPR1 - Friendly Feedback}
\begin{itemize}
    \item \textbf{Test ID:} TUPR1
    \item \textbf{Type:} Static, Manual
    \item \textbf{Initial State:} Error states are accessible.
    \item \textbf{Input/Condition:} Trigger common errors.
    \item \textbf{Output/Result:} System provides friendly, clear feedback.
    \item \textbf{How test will be performed:} Manually review error messages to confirm they are polite and helpful.
\end{itemize}

\subsubsection{UPR2 - Standardized Iconography}
\begin{itemize}
    \item \textbf{Test ID:} TUPR2
    \item \textbf{Type:} Static, Manual
    \item \textbf{Initial State:} A subset of complete interface iconography is available.
    \item \textbf{Input/Condition:} Developer reviews subset of complete interface iconography.
    \item \textbf{Output/Result:} Subset of complete interface iconography is deemed standard and inoffensive.
    \item \textbf{How test will be performed:} Manually review individual icons to confirm they are standard and inoffensive.
\end{itemize}

\subsubsection{ACR1 - Accessible Fonts}
\begin{itemize}
    \item \textbf{Test ID:} TACR1
    \item \textbf{Type:} Static, Manual
    \item \textbf{Initial State:} Font-defining code is available.
    \item \textbf{Input/Condition:} Developer reviews font-defining code.
    \item \textbf{Output/Result:} Font-defining code is deemed to only contain accessible fonts in accordance to the \href{https://www.w3.org/TR/2023/REC-WCAG22-20231005/}{WCAG 2.2} standard.
    \item \textbf{How test will be performed:} Perform a manual inspection of fonts available in the UI and cross-check them with the WCAG 2.2 standard.
\end{itemize}

\subsubsection{ACR2 - Color Blind Mode}
\begin{itemize}
    \item \textbf{Test ID:} TACR2
    \item \textbf{Type:} Dynamic, Manual
    \item \textbf{Initial State:} Interface set to default visibility mode.
    \item \textbf{Input/Condition:} Enable color blind mode.
    \item \textbf{Output/Result:} Interface set to color blind visibility mode.
    \item \textbf{How test will be performed:} Use a web driver like \texttt{Selenium} to perform input and record output.
\end{itemize}

\subsubsection{PAR1 - Query Request Time Precision}
\begin{itemize}
    \item \textbf{Test ID:} TPAR1
    \item \textbf{Type:} Dynamic, Automated
    \item \textbf{Initial State:} Server active and client application open.
    \item \textbf{Input/Condition:} Issue any valid query.
    \item \textbf{Output/Result:} Server returns response and client application displays query request time.
    \item \textbf{How test will be performed:} Use a web driver like \texttt{Selenium} to perform input and record output, capturing displayed time for verification.
\end{itemize}

\subsubsection{PAR2 - Rounding Accuracy}
\begin{itemize}
    \item \textbf{Test ID:} TPAR2
    \item \textbf{Type:} Dynamic, Automated
    \item \textbf{Initial State:} System loaded with rounding functions.
    \item \textbf{Input/Condition:} Input values with decimals.
    \item \textbf{Output/Result:} Values are rounded accurately according to specification.
    \item \textbf{How test will be performed:} Perform automated tests on rounding functions with pre-defined decimal values.
\end{itemize}

\subsubsection{RAR1 - Fault Tolerance}
\begin{itemize}
    \item \textbf{Test ID:} TRAR1
    \item \textbf{Type:} Dynamic, Manual
    \item \textbf{Initial State:} Server operational with any state or load, e.g., idle, under little load (2 or fewer users interleaving requests less than once every 30 minutes), under intermediate use (2 or more users interleaving requests at least once every 30 minutes), or under strenuous use (4 or more users interleaving requests at least once every 5 minutes).
    \item \textbf{Input/Condition:} Server operational under any sequence of state transitions, e.g., from any of idle, under little, intermediate, or strenuous load, to idle, under little, intermediate, or strenuous load.
    \item \textbf{Output/Result:} Server operational with any state or load.
    \item \textbf{How test will be performed:} Schedule a 3-day monitoring period and use Ubuntu Server's \href{https://manpages.ubuntu.com/manpages/xenial/man1/uptime.1.html}{\texttt{uptime}} command to find the uptime across the monitoring period, allowing assessment and extrapolation of results from 3 days to 30 days. If possible, repeat the test with a 30-day (or longer) monitoring period. \textcolor{red}{\emph{Note}: Precise uptime metrics can only be achieved by formal checking which we do not have the expertise, time, or necessity for. A simple extrapolation result suffices for the given scope.}
\end{itemize}

\subsubsection{CR1 - Minimum Concurrent Users Load}
\begin{itemize}
    \item \textbf{Test ID:} TCR1
    \item \textbf{Type:} Dynamic, Load Test
    \item \textbf{Initial State:} System idle.
    \item \textbf{Input/Condition:} Simulate multiple user logins.
    \item \textbf{Output/Result:} System handles expected concurrent user limit.
    \item \textbf{How test will be performed:} Use a load testing tool to simulate concurrent users.
\end{itemize}

\subsubsection{CR2 - Data Storage}
\begin{itemize}
    \item \textbf{Test ID:} TCR2
    \item \textbf{Type:} Dynamic, automated
    \item \textbf{Initial State:} Mock database active.
    \item \textbf{Input/Condition:} Issue any valid query.
    \item \textbf{Output/Result:} Database stores appropriate data related to songs and query information.
    \item \textbf{How test will be performed:} Use frameworks like \texttt{factoryboy} and \texttt{Pytest}'s fixtures to mock a database and test storage functionality.
\end{itemize}

\subsubsection{EPER1 - Server Device}
\begin{itemize}
    \item \textbf{Test ID:} TEPER1
    \item \textbf{Type:} Environment, Inspection
    \item \textbf{Initial State:} Server is available for inspection.
    \item \textbf{Input/Condition:} Inspect server specifications and body.
    \item \textbf{Output/Result:} Server is of make and model Dell OptiPlex 3050.
    \item \textbf{How test will be performed:} View server using los ojos.
\end{itemize}

\subsection{Productization Requirements}

\subsubsection{PRR1 - Production Readiness Verification}
\begin{itemize}
    \item \textbf{Test ID:} TPRR1
    \item \textbf{Type:} Dynamic, Manual
    \item \textbf{Initial State:} System deployed in staging environment.
    \item \textbf{Input/Condition:} Perform a complete end-to-end run.
    \item \textbf{Output/Result:} System operates without failure in a production-like setting.
    \item \textbf{How test will be performed:} Conduct end-to-end test in staging to verify production readiness.
\end{itemize}

\subsubsection{MR1 - Ease of Code Updates}
\begin{itemize}
    \item \textbf{Test ID:} TMR1
    \item \textbf{Type:} Static, Manual
    \item \textbf{Initial State:} Source code repository active.
    \item \textbf{Input/Condition:} Review update process.
    \item \textbf{Output/Result:} Codebase is structured for easy updates.
    \item \textbf{How test will be performed:} Review code structure and modularity to ensure maintainability.
\end{itemize}

\subsubsection{ACCR1 - Licensed Song Access}
\begin{itemize}
    \item \textbf{Test ID:} TMR1
    \item \textbf{Type:} Dynamic, Automated
    \item \textbf{Initial State:} Mock database active with multiple queries from different (at least 2 unique) users already loaded.
    \item \textbf{Input/Condition:} Issue queries to access songs requested other users, not the current user.
    \item \textbf{Output/Result:} Database returns an empty response because the requesting user does not have access (or a license) to the songs uploaded/licensed by the other user(s).
    \item \textbf{How test will be performed:} Use frameworks like \texttt{factoryboy} and \texttt{Pytest}'s fixtures to mock a database and entries, then issue queries, capture response, and ensure it is empty.
\end{itemize}

\subsubsection{IR1 - Database Backup}
\begin{itemize}
    \item \textbf{Test ID:} TIR1
    \item \textbf{Type:} Static \& Manual, Dynamic \& Automated
    \item \textbf{Initial State:} Database layout configured \& backup code is complete.
    \item \textbf{Input/Condition:} Conduct code review to ensure backup functionality is correct, and simulate it running (on-command as opposed to weekly) to ensure it does backup.
    \item \textbf{Output/Result:} Database backup functionality is found to be correct, with ad-hoc generated backup artifacts to check it live.
    \item \textbf{How test will be performed:} Conduct a code review/walkthrough and use both unit and integration testing through \texttt{Pytest} with fixtures (alongside \texttt{factoryboy}) to ensure the correct backup artifacts are generated.
\end{itemize}

\subsubsection{IR2 - Database Deduplication}
\begin{itemize}
    \item \textbf{Test ID:} TIR2
    \item \textbf{Type:} Static \& Manual, Dynamic \& Automated
    \item \textbf{Initial State:} Database layout configured \& deduplication code is complete.
    \item \textbf{Input/Condition:} Conduct code review to ensure deduplication functionality is correct, and insert duplicate records to ensure the mechanism prevents their insertion or ``\href{https://oncodingstyle.blogspot.com/2008/10/fail-early-fail-loudly.html#:~:text=Failing%20loudly%20is%20a%20benefit,this%20is%20a%20checked%20exception.}{fails loudly}''.
    \item \textbf{Output/Result:} Database deduplication functionality is found to be correct, with induced duplicate insertions prevented.
    \item \textbf{How test will be performed:} Conduct a code review/walkthrough and use both unit and integration testing through \texttt{Pytest} with fixtures (alongside \texttt{factoryboy}) to ensure the correct deduplication behaviour is encountered.
\end{itemize}

\subsubsection{PR1 - Data Encryption Verification}
\begin{itemize}
    \item \textbf{Test ID:} TPR1
    \item \textbf{Type:} Static, Manual
    \item \textbf{Initial State:} Database system in use.
    \item \textbf{Input/Condition:} Inspect database for encryption protocols.
    \item \textbf{Output/Result:} All sensitive data is encrypted in storage.
    \item \textbf{How test will be performed:} Review encryption settings in database configuration.
\end{itemize}

\subsubsection{AUR1 - Access Logs for User Sessions}
\begin{itemize}
    \item \textbf{Test ID:} TAUR1
    \item \textbf{Type:} Dynamic, Manual
    \item \textbf{Initial State:} System active with user sessions.
    \item \textbf{Input/Condition:} Access user session logs.
    \item \textbf{Output/Result:} Logs capture all user activities accurately.
    \item \textbf{How test will be performed:} Review session logs for accuracy and completeness.
\end{itemize}

\subsubsection{CUR1 - Multilingual Support}
\begin{itemize}
    \item \textbf{Test ID:} TCUR1
    \item \textbf{Type:} Dynamic, Manual
    \item \textbf{Initial State:} System interface displayed.
    \item \textbf{Input/Condition:} Switch to different language options.
    \item \textbf{Output/Result:} System adapts to selected language without errors.
    \item \textbf{How test will be performed:} Switch languages manually and verify accurate translations.
\end{itemize}

\subsubsection{LGR1 - Compliance with Copyright Laws}
\begin{itemize}
    \item \textbf{Test ID:} TLGR1
    \item \textbf{Type:} Static, Manual
    \item \textbf{Initial State:} Content library loaded.
    \item \textbf{Input/Condition:} Inspect all music content for licensing.
    \item \textbf{Output/Result:} All content has proper copyright attributions.
    \item \textbf{How test will be performed:} Check each music file and source for copyright compliance.
\end{itemize}

\subsubsection{LGR2 - Adherence to Data Protection Regulations}
\begin{itemize}
    \item \textbf{Test ID:} TLGR2
    \item \textbf{Type:} Static, Manual
    \item \textbf{Initial State:} User data system in place.
    \item \textbf{Input/Condition:} Inspect data management policies.
    \item \textbf{Output/Result:} User data management meets legal requirements.
    \item \textbf{How test will be performed:} Review data management practices to confirm legal compliance.
\end{itemize}

\subsection{Traceability Between Test Cases and Requirements}

\wss{Provide a table that shows which test cases are supporting which
  requirements.}

\section{Unit Test Description}

\wss{This section should not be filled in until after the MIS (detailed design
  document) has been completed.}

\wss{Reference your MIS (detailed design document) and explain your overall
philosophy for test case selection.}  

\wss{To save space and time, it may be an option to provide less detail in this section.  
For the unit tests you can potentially layout your testing strategy here.  That is, you 
can explain how tests will be selected for each module.  For instance, your test building 
approach could be test cases for each access program, including one test for normal behaviour 
and as many tests as needed for edge cases.  Rather than create the details of the input 
and output here, you could point to the unit testing code.  For this to work, you code 
needs to be well-documented, with meaningful names for all of the tests.}

\subsection{Unit Testing Scope}

\wss{What modules are outside of the scope.  If there are modules that are
  developed by someone else, then you would say here if you aren't planning on
  verifying them.  There may also be modules that are part of your software, but
  have a lower priority for verification than others.  If this is the case,
  explain your rationale for the ranking of module importance.}

\subsection{Tests for Functional Requirements}

\wss{Most of the verification will be through automated unit testing.  If
  appropriate specific modules can be verified by a non-testing based
  technique.  That can also be documented in this section.}

\subsubsection{Module 1}

\wss{Include a blurb here to explain why the subsections below cover the module.
  References to the MIS would be good.  You will want tests from a black box
  perspective and from a white box perspective.  Explain to the reader how the
  tests were selected.}

\begin{enumerate}

\item{test-id1\\}

Type: \wss{Functional, Dynamic, Manual, Automatic, Static etc. Most will
  be automatic}
					
Initial State: 
					
Input: 
					
Output: \wss{The expected result for the given inputs}

Test Case Derivation: \wss{Justify the expected value given in the Output field}

How test will be performed: 
					
\item{test-id2\\}

Type: \wss{Functional, Dynamic, Manual, Automatic, Static etc. Most will
  be automatic}
					
Initial State: 
					
Input: 
					
Output: \wss{The expected result for the given inputs}

Test Case Derivation: \wss{Justify the expected value given in the Output field}

How test will be performed: 

\item{...\\}
    
\end{enumerate}

\subsubsection{Module 2}

...

\subsection{Tests for Nonfunctional Requirements}

\wss{If there is a module that needs to be independently assessed for
  performance, those test cases can go here.  In some projects, planning for
  nonfunctional tests of units will not be that relevant.}

\wss{These tests may involve collecting performance data from previously
  mentioned functional tests.}

\subsubsection{Module ?}
		
\begin{enumerate}

\item{test-id1\\}

Type: \wss{Functional, Dynamic, Manual, Automatic, Static etc. Most will
  be automatic}
					
Initial State: 
					
Input/Condition: 
					
Output/Result: 
					
How test will be performed: 
					
\item{test-id2\\}

Type: Functional, Dynamic, Manual, Static etc.
					
Initial State: 
					
Input: 
					
Output: 
					
How test will be performed: 

\end{enumerate}

\subsubsection{Module ?}

...

\subsection{Traceability Between Test Cases and Modules}

\wss{Provide evidence that all of the modules have been considered.}
				
\bibliographystyle{plainnat}

\bibliography{../../refs/References}

\newpage

\section{Appendix}

This is where you can place additional information.

\subsection{Symbolic Parameters}

The definition of the test cases will call for SYMBOLIC\_CONSTANTS.
Their values are defined in this section for easy maintenance.

\subsection{Usability Survey Questions}

To validate the usability requirements outlined in the SRS, the following survey questions have been structured as 'check criteria', 
allowing for quick evaluation with follow-up prompts to gather additional details if necessary.

\begin{itemize}
    \item \textbf{Tooltip Effectiveness (EUR1)}:
    \begin{itemize}
        \item \textit{Survey Question:} "Were the tooltips helpful in understanding the function of each feature? (Yes/No)  
        \begin{itemize}
            \item If No, please specify which tooltips were unclear or missing."
        \end{itemize}
    \end{itemize}

    \item \textbf{Ease of Personalization (PIR1)}:
    \begin{itemize}
        \item \textit{Survey Question:} "Were you able to find and switch between different display color themes easily? (Yes/No)  
        \begin{itemize}
            \item If No, please describe any difficulties you encountered or suggestions for improvement."
        \end{itemize}
    \end{itemize}

    \item \textbf{First-Time User Guide Clarity and Timing (LR1, LR2)}:
    \begin{itemize}
        \item \textit{Survey Question:} "Did the first-time user guide help you understand the main features within 10 minutes? (Yes/No)  
        \begin{itemize}
            \item If No, please indicate what aspects were unclear or took longer than expected."
        \end{itemize}
    \end{itemize}

    \item \textbf{Content Politeness and Appropriateness (UPR1, UPR2)}:
    \begin{itemize}
        \item \textit{Survey Question:} "Was all content and iconography appropriate and free from offensive material? (Yes/No)  
        \begin{itemize}
            \item If No, please specify any instances where content was found inappropriate."
        \end{itemize}
    \end{itemize}

    \item \textbf{Accessibility of Font and Color Options (ACR1, ACR2)}:
    \begin{itemize}
        \item \textit{Survey Question:} "Did you find the font choices easy to read, and was the color-blind mode accessible if you used it? (Yes/No)  
        \begin{itemize}
            \item If No, please describe any accessibility challenges you faced with font readability or color options."
        \end{itemize}
    \end{itemize}
\end{itemize}

\noindent
These questions are designed to verify specific usability requirements established in the SRS.


\subsection{Code Walkthrough Checklist}
\label{A-Code-walkthrough-checklist}
Here is a sample code walkthrough checklist. It will be updated as reviews are performed:\\

\noindent
Functionality
\begin{itemize}[label=$\square$]
  \item Does the code perform its intended task?
  \item Can this code be traced to a requirement?
  \item Is there redundant or unnecessary code?
\end{itemize}

\noindent
Readability
\begin{itemize}[label=$\square$]
  \item Do the functions and variables have meaningful names?
  \item Does the source code contain sufficient commenting?
  \item Does the code follow PEP8 style guidelines
\end{itemize}

\noindent
Modularity
\begin{itemize}[label=$\square$]
  \item Are there excessively long functions that can be broken down?
  \item Is the code organized? 
  \item Is the code easy to change or expand upon?
\end{itemize}

\noindent
Error Handling
\begin{itemize}[label=$\square$]
  \item Are exceptions and errors handled gracefully?
  \item Do exceptions and erros have useful error messages? 
\end{itemize}

\noindent
Testing
\begin{itemize}[label=$\square$]
  \item Are there enough unit tests such that the code coverage tool reports 100\% code coverage?
  \item Does \texttt{Pytest} report that 100\% of unit tests are passing? 
  \item Are there enough system tests to cover all requirements? 
  \item Are all system tests passing? 
\end{itemize}

\noindent
Reliability
\begin{itemize}[label=$\square$]
  \item Is the code fault-tolerant? I.e., does the system continue to operate despite failures/faults?
  \item Does the code have effective exception-handling and error recovery mechanisms?
\end{itemize}

\noindent
Efficiency
\begin{itemize}[label=$\square$]
  \item How much memory or processor capacity does the program consume?
  \item Are algorithms optimized, avoiding unnecessary operations?
\end{itemize}

\noindent
Reusability
\begin{itemize}[label=$\square$]
  \item Can components be reused in other applications/other parts of the application?
  \item Does the program have a well-partitioned, modular design with \emph{strong cohesion} and \emph{loose coupling}?
\end{itemize}

\noindent
Scalability
\begin{itemize}[label=$\square$]
  \item Can the system grow to accomodate more users, servers, data or other components?
  \item Can it do so with acceptable performance and at acceptable cost?
\end{itemize}

\newpage{}
\section*{Appendix --- Reflection}

\wss{This section is not required for CAS 741}

The information in this section will be used to evaluate the team members on the
graduate attribute of Lifelong Learning.

The purpose of reflection questions is to give you a chance to assess your own
learning and that of your group as a whole, and to find ways to improve in the
future. Reflection is an important part of the learning process.  Reflection is
also an essential component of a successful software development process.  

Reflections are most interesting and useful when they're honest, even if the
stories they tell are imperfect. You will be marked based on your depth of
thought and analysis, and not based on the content of the reflections
themselves. Thus, for full marks we encourage you to answer openly and honestly
and to avoid simply writing ``what you think the evaluator wants to hear.''

Please answer the following questions.  Some questions can be answered on the
team level, but where appropriate, each team member should write their own
response:


\begin{enumerate}
  \item What went well while writing this deliverable? 
  \item What pain points did you experience during this deliverable, and how
    did you resolve them?
  \item What knowledge and skills will the team collectively need to acquire to
  successfully complete the verification and validation of your project?
  Examples of possible knowledge and skills include dynamic testing knowledge,
  static testing knowledge, specific tool usage, Valgrind etc.  You should look to
  identify at least one item for each team member.
  \item For each of the knowledge areas and skills identified in the previous
  question, what are at least two approaches to acquiring the knowledge or
  mastering the skill?  Of the identified approaches, which will each team
  member pursue, and why did they make this choice?
\end{enumerate}

\end{document}