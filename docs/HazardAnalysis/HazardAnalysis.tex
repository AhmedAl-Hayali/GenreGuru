\documentclass{article}

\usepackage{booktabs}
\usepackage{tabularx}
\usepackage{hyperref}

\hypersetup{
    colorlinks=true,       % false: boxed links; true: colored links
    linkcolor=red,          % color of internal links (change box color with linkbordercolor)
    citecolor=green,        % color of links to bibliography
    filecolor=magenta,      % color of file links
    urlcolor=cyan           % color of external links
}

\title{Hazard Analysis\\\progname}

\author{\authname}

\date{}

\input{../Comments}
%% Common Parts

\newcommand{\progname}{GenreGuru} % PUT YOUR PROGRAM NAME HERE
\newcommand{\authname}{Team 8 -- Rhythm Rangers\\
\\ Ansel Chen
\\ Muhammad Jawad
\\ Mohamad-Hassan Bahsoun
\\ Matthew Baleanu
\\ Ahmed Al-Hayali} % AUTHOR NAMES                  

\usepackage{hyperref}
    \hypersetup{colorlinks=true, linkcolor=blue, citecolor=blue, filecolor=blue,
                urlcolor=blue, unicode=false}
    \urlstyle{same}
                                


\begin{document}

\maketitle
\thispagestyle{empty}

~\newpage

\pagenumbering{roman}

\begin{table}[hp]
\caption{Revision History} \label{TblRevisionHistory}
\begin{tabularx}{\textwidth}{llX}
\toprule
\textbf{Date} & \textbf{Developer(s)} & \textbf{Change}\\
\midrule
Date1 & Name(s) & Description of changes\\
Date2 & Name(s) & Description of changes\\
... & ... & ...\\
\bottomrule
\end{tabularx}
\end{table}

~\newpage

\tableofcontents

~\newpage

\pagenumbering{arabic}

\wss{You are free to modify this template.}

\section{Introduction}

\wss{You can include your definition of what a hazard is here.}

This document is dedicated as a Hazard Analysis of the GenreGuru Music System. The GenreGuru software is designed
to aid its users with their consumption and creation of music, doing so by providing song analysis, recommendation,
and music generation services. As such, we define a hazard as a potential malfunction in the system, either due to internal
factors (such as the training data, defects in the software) or external factors (such as user inputs).  

\section{Scope and Purpose of Hazard Analysis}

\wss{You should say what \textbf{loss} could be incurred because of the hazards.}

The purpose of hazard analysis for this project is to determine points and causes of failure in the system. This 
includes their effects and considering migitation methods. Here are some brief types of loss we expect hazards could 
incurr: 
\begin{itemize}
    \item Compromised Data - such as corruption of the training dataset/outputs
    \item Project History - improper github merging causing portions of documentation to vanish
    \item User Experience Loss - generated recommendations/snippets do not satisfy user needs. 
\end{itemize}

\section{System Boundaries and Components}

\wss{Dividing the system into components will help you brainstorm the hazards.
You shouldn't do a full design of the components, just get a feel for the major
ones.  For projects that involve hardware, the components will typically include
each individual piece of hardware.  If your software will have a database, or an
important library, these are also potential components.}

The GenreGuru Software Application is primarily composed of the following components: 
\begin{itemize}
\item Track Analysis (Featurization) System
\item Track recommendations generation system
\item The snippet generation system
\item User interaction interface
\item Backend Processing System (On-Premise Server)
\end{itemize}
In addition to the software components, the GenreGuru Project also relies on the following 
external resources/components:
\begin{itemize}
    \item Music Streaming Service Provider API (Such as Spotify's)
    \item Music Data set
    \item Music-Related Machine Learning Algorithms/Libraries
\end{itemize}
Finally, the github repository is the project's documentation, contains the source code
and handles version control for the project. 
\section{Critical Assumptions}

\wss{These assumptions that are made about the software or system.  You should
minimize the number of assumptions that remove potential hazards.  For instance,
you could assume a part will never fail, but it is generally better to include
this potential failure mode.} 

Currently, we have one critical assumption for this project. This assumption is that volume control
is handled by the user's hardware in a way such that the user cannot experience sounds that 
could potentially physically harm them. 

\section{Failure Mode and Effect Analysis}

\wss{Include your FMEA table here. This is the most important part of this document.}
\wss{The safety requirements in the table do not have to have the prefix SR.
The most important thing is to show traceability to your SRS. You might trace to
requirements you have already written, or you might need to add new
requirements.}
\wss{If no safety requirement can be devised, other mitigation strategies can be
entered in the table, including strategies involving providing additional
documentation, and/or test cases.}

\section{Safety and Security Requirements}

\wss{Newly discovered requirements.  These should also be added to the SRS.  (A
rationale design process how and why to fake it.)}

\section{Roadmap}

\wss{Which safety requirements will be implemented as part of the capstone timeline?
Which requirements will be implemented in the future?}

\newpage{}

\section*{Appendix --- Reflection}

\wss{Not required for CAS 741}

\input{../Reflection.tex}

\begin{enumerate}
    \item What went well while writing this deliverable? 
    \item What pain points did you experience during this deliverable, and how
    did you resolve them?
    \item Which of your listed risks had your team thought of before this
    deliverable, and which did you think of while doing this deliverable? For
    the latter ones (ones you thought of while doing the Hazard Analysis), how
    did they come about?
    \item Other than the risk of physical harm (some projects may not have any
    appreciable risks of this form), list at least 2 other types of risk in
    software products. Why are they important to consider?

    Two types of risks in software products that aren't related to physical 
    are 1) the risk of comprising data (both the user's and the software's), 2) 
    Scope Creep, where the software is uncontrollably growing and adding features
    beyond what was intially agreed to. 

    1) is important because user and application data can be very sensitive, eg
    users often reuse their passwords/emails, thus said data being compromised could
    have far-reaching effects beyond just the user's ability to interact with the service.
    Company Data can also be very valuable, if it is exposed it could be used by competitors 
    or used to exploit other vulnerabilities within the software.

    2) is important because scope creep arises when the project was not properly 
    defined during the early planning stages. This often leads to extra developement work,
    costs, all for features that might not work properly because of the creep or are 
    unecessary and were not requested by the client. 

\end{enumerate}

\end{document}