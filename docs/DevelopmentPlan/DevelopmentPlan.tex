\documentclass{article}

\usepackage{booktabs}
\usepackage{tabularx}

\title{Development Plan\\\progname}

\author{\authname}

\date{}

%% Comments

\usepackage{color}

\newif\ifcomments\commentstrue %displays comments
%\newif\ifcomments\commentsfalse %so that comments do not display

\ifcomments
\newcommand{\authornote}[3]{\textcolor{#1}{[#3 ---#2]}}
\newcommand{\todo}[1]{\textcolor{red}{[TODO: #1]}}
\else
\newcommand{\authornote}[3]{}
\newcommand{\todo}[1]{}
\fi

\newcommand{\wss}[1]{\authornote{blue}{SS}{#1}} 
\newcommand{\plt}[1]{\authornote{magenta}{TPLT}{#1}} %For explanation of the template
\newcommand{\an}[1]{\authornote{cyan}{Author}{#1}}

%% Common Parts

\newcommand{\progname}{Software Engineering} % PUT YOUR PROGRAM NAME HERE
\newcommand{\authname}{Team 8 -- Rhythm Rangers\\
\\ Ansel Chen
\\ Muhammad Jawad
\\ Mohamad-Hassan Bahsoun
\\ Matthew Baleanu
\\ Ahmed Al-Hayali} % AUTHOR NAMES                  

\usepackage{hyperref}
    \hypersetup{colorlinks=true, linkcolor=blue, citecolor=blue, filecolor=blue,
                urlcolor=blue, unicode=false}
    \urlstyle{same}
                                


\begin{document}

\maketitle

\begin{table}[hp]
\caption{Revision History} \label{TblRevisionHistory}
\begin{tabularx}{\textwidth}{llX}
\toprule
\textbf{Date} & \textbf{Developer(s)} & \textbf{Change}\\
\midrule
2024-09-24 & All members & Complete Revision 0\\
\bottomrule
\end{tabularx}
\end{table}

\newpage{}

\wss{Put your introductory blurb here.  Often the blurb is a brief roadmap of
what is contained in the report.}

\section{Confidential Information?}

\wss{State whether your project has confidential information from industry, or
not.  If there is confidential information, point to the agreement you have in
place.}

\wss{For most teams this section will just state that there is no confidential
information to protect.}
\section{IP to Protect}

\wss{State whether there is IP to protect.  If there is, point to the agreement.
All students who are working on a project that requires an IP agreement are also
required to sign the ``Intellectual Property Guide Acknowledgement.''}

\section{Copyright License}

\wss{What copyright license is your team adopting.  Point to the license in your
repo.}

\section{Team Meeting Plan}
Team meetings will be scheduled in a relatively ad-hoc fashion. Members have shared their daily schedules throughout the working week with each other, and weekly availability notes are accounted for when scheduling meetings. Availability notes are to be shared before the start of the working week so a week's meeting plan can be drafted and voted on by the end of Monday. Team meetings can occur for:
\begin{itemize}
  \item task delegation for upcoming deliverables \& discussing deliverable progress
  
  \emph{we hope to make these meetings brief and infrequent in the future by using GitHub Projects and asynchronous communication instead};
  \item work sessions to collaborate and discuss ideas (deliverable-related) synchronously, \emph{preferably in-person};
  \item pair programming;
  \item conducting deliverable reviews;
  \item conducting deliverable retrospectives, i.e., reflecting on successful and unsucessful practices used in the most recent deliverable after its completion.
\end{itemize}

\wss{How will the meetings be structured?  There should be a chair for all meetings.  There should be an agenda for all meetings.}

\section{Team Communication Plan}
All team communication is done through a discord server. The discord has three text channels:
\begin{itemize}
  \item General (anything not project-related is posted here)
  \item Important Updates (anything related to weekly meeting availability goes here)
  \item Locked In (anything related to project work goes here)
\end{itemize}
\noindent
The discord also has 2 voice channels:

\begin{itemize}
  \item Weekly Meeting (administrative meetings happen here)
  \item Locked In (collaboration of project development happens here)
\end{itemize}
\section{Team Member Roles}
\begin{itemize}
  \item Ahmed: Meeting manager and scheduler, developer
  \item Ansel: Team liaison, developer
  \item Matthew: Developer
  \item Mohammed-Hassan: Developer
  \item Muhammad: Developer
\end{itemize}

\section{Workflow Plan}
\subsection{General Workflow}
\begin{itemize}
  \item Issues are created, assigned, and attached to the project. Issues will have a template akin to those find on \href{https://github.com/stevemao/github-issue-templates/tree/master/system/ISSUE_TEMPLATE}{stevemao/github-issue-templates}. These issues should pertain to deliverable sections, split into a completion assignee and a reviewer;
  \item The \texttt{main} branch is protected, so team members must work in independent branches. We hope to restrict branch naming to a standardized format, e.g., \href{https://dev.to/varbsan/a-simplified-convention-for-naming-branches-and-commits-in-git-il4}{this} or akin to \href{https://www.conventionalcommits.org/en/v1.0.0/}{conventional commits} (which comes with \href{https://github.com/conventional-changelog/commitlint/tree/master/%40commitlint/config-conventional}{a linter}!);
  \item Team members' commits must follow the \href{https://www.conventionalcommits.org/en/v1.0.0/}{conventional commits} standard;
  \item Whenever necessary, a team member can choose to merge their changes to the \texttt{main} branch with a pull request. Pull requests should have a template akin to the simple pull request found \href{https://graphite.dev/guides/pull-request-templates}{here} or the more complicated \href{https://github.com/dbt-labs/dbt-init/blob/master/starter-project/.github/pull_request_template.md}{data-centric template from dbt}. A pull request should be attached to the \href{https://github.com/users/AhmedAl-Hayali/projects/1}{GenreGuru Project on GitHub} with a \emph{dedicated} reviewer and potentially a review timeline and checklist. Having only one reviewer avoids the issue of \href{https://en.wikipedia.org/wiki/Diffusion_of_responsibility}{diffusion of responsiblity}.
\end{itemize}

\subsection{Usability Testing}
\begin{itemize}
  \item Issues must contain a comparison between what is expected of the code and what is actually seen by the user;
  \item If possible, issue creators should attach any log files they collect to the issue;
  \item The issue must be assigned to the developer who originally pushed the code that is causing the bug.
\end{itemize}

\section{Proof of Concept Demonstration Plan}
There are two main risks to the project --- the song data collection and song generation.

\subsection{Risks regarding song collection}
\begin{itemize}
  \item License acquisition may be necessary for some songs;
  \begin{itemize}
    \item Acquisition of the song in general may require a license from the artist, label, publisher, or platform, e.g., Spotify;
    \item Platform providing songs may have limited API access.
  \end{itemize}
  \item Songs may be only partially accessible, e.g., song snippets from Spotify;
  \item Songs may be available but we are prohibited from using them to train a machine learning model.
\end{itemize}
These risks can be dismissed if the project is to use different, less strict, song providers, or tailor the project to only use non-copyrighted songs.

\subsection{Risks regarding song generation}
\begin{itemize}
  \item The generative mechanism will inherently be a machine learning model, which entails issues,
  \begin{itemize}
    \item The model may hallucinate and produce unexpected outputs, i.e., music of the wrong genre (particularly of concern if training data is unbalanced), or just uncomfortable nonsensical sounds. \emph{This could be a result of too little data to train a complex model (resulting in high variance), or too simplistic of a model (resulting in high bias)};
  \end{itemize}
  \item The model will be challenging to formulate, e.g., establishing architecture, objective function, and optimizer;
  \item The model will be so complex, i.e., will contain many parameters, such that it requires tremendous quantities of data and training time to converge to sensical results.
\end{itemize}
These risks cannot be entirely dismissed, but can be remedied greatly by considering the work of previous similar works and following their process, i.e., reusing architecture, data, or training mechanism. Nonetheless, this project is doable, as a parallel of it was completed \href{https://www.reddit.com/r/MachineLearning/comments/6476kj/projectmusic_generated_using_my_rnn_some_bach/}{in 2017}.

\section{Expected Technology}

The technologies and tools expected for this project include:

\begin{itemize}
    \item \textbf{Programming Language:} Python, due to its vast libraries in machine learning and audio processing, such as \texttt{librosa} and \texttt{pydub}.
    \item \textbf{Libraries:} 
        \begin{itemize}
            \item \texttt{librosa}: For music and audio analysis.
            \item \texttt{pydub}: For audio processing and manipulation.
            \item \texttt{scikit-learn} and \texttt{TensorFlow}: For building machine learning models to classify and generate music.
        \end{itemize}
    \item \textbf{Frameworks:} 
        \begin{itemize}
            \item \texttt{Flask} or \texttt{Django}: For the web-based interface, allowing users to interact with the system.
            \item \texttt{PyTorch} or \texttt{TensorFlow}: For implementing deep learning models to generate and classify music.
        \end{itemize}
    \item \textbf{External APIs:} Spotify API will be used for fetching song previews, features, and other metadata for recommendation purposes.
    \item \textbf{Pre-trained Models:} We may leverage some pre-trained models for audio generation, such as OpenAI's Jukebox or similar publicly available models, while customizing them to fit our needs.
    \item \textbf{Linters:} a CI-integrable Python-specific linter, e.g., \texttt{pylint}, \texttt{ruff}, or \texttt{flake8}. There also are git-specific linters like \href{https://github.com/conventional-changelog/commitlint/tree/master/%40commitlint/config-conventional}{conventional commit's linter}.
    \item \textbf{Formatter(s):} a CI-integrable formatter formatter, e.g., \texttt{black}.
    \item \textbf{Type-checker(s):} a CI-integrable type-checker, e.g., \texttt{mypy}.
    \item \textbf{Documentation Generation:} a CI-integrable documentation generator, e.g., \texttt{sphinx}.
    \item \textbf{Tester(s):} a CI-integrable testing framework, e.g., \texttt{PyTest}.
    \item \textbf{Continuous Integration:} the 5 previous ``CI-integrable'' components are a good baseline, but we plan on using a local server, and deployment onto it with CI would be sweet. As much automation with CI, within reason, is desirable as it is excellent experience to be transferred to a working setting.
    \item \textbf{Environment Management:} environment managers for Python like \texttt{pipenv} and \texttt{poetry} are excellent for ensuring the program works the same everywhere, installation becomes straightforward, and CI becomes marginally easier because of more standardization.
\end{itemize}

\wss{git, GitHub and GitHub projects should be part of your technology.}
\section{Coding Standard}
The coding standards will be a series of PEPs, but importantly \href{https://peps.python.org/pep-0008/}{PEP 8} for coding style and \href{https://peps.python.org/pep-0257/}{PEP 257} for docstrings that will form the basis for the API and user reference. For simplicity, a \href{https://stackoverflow.com/a/69842362}{package-like folder structure} should be used in conjunction with the provided template.

\section{Project Scheduling}
There will be no GANTT charts - we will conduct weekly meetings as "standups" and decide what is to be worked on. Please refer to section 1, ``Team Meeting Plan'' for details.

\newpage{}

\section*{Appendix --- Reflection}

The purpose of reflection questions is to give you a chance to assess your own
learning and that of your group as a whole, and to find ways to improve in the
future. Reflection is an important part of the learning process.  Reflection is
also an essential component of a successful software development process.  

Reflections are most interesting and useful when they're honest, even if the
stories they tell are imperfect. You will be marked based on your depth of
thought and analysis, and not based on the content of the reflections
themselves. Thus, for full marks we encourage you to answer openly and honestly
and to avoid simply writing ``what you think the evaluator wants to hear.''

Please answer the following questions.  Some questions can be answered on the
team level, but where appropriate, each team member should write their own
response:


\begin{enumerate}
    \item Why is it important to create a development plan prior to starting the
    project?\\
    - Creating a development plan is crucial for a few reasons:
    \begin{itemize}
      \item It gives the project a clear direction and scope since all goals are outlined
      \item It sets boundaries on the project to prevent unplanned expansions
      \item It outlines anticipated challenges and contingency plans
      \item It provides transparent communication of the project expectations to the stakeholders
    \end{itemize}
    \item In your opinion, what are the advantages and disadvantages of using CI/CD?
    
    CI/CD is a wonderful tool that automates mind-numbing tasks of linting, styling, formatting, deployment, and document generation, for example, but it comes at the heavy cost of relatively complicated setup process for said CI/CD to work. Thankfully, there are many GitHub actions templates online, so we often need not to worry about starting from scratch.
    \item What disagreements did your group have in this deliverable, if any, and how did you resolve them?
    
    There were few disagreements, but the frequency of meetings is a hotly-contested topic. We chose to have relatively frequent meetings during the first 3 weeks of the project, will experiment with far fewer in the future and rely on asynchronous communication. Depending on the success of it during the next few weeks, we will continue or adjust it to better suit our needs in completing future deliverables succesfully.
\end{enumerate}

\newpage{}

\section*{Appendix --- Team Charter}

\wss{borrows from
\href{https://engineering.up.edu/industry_partnerships/files/team-charter.pdf}
{University of Portland Team Charter}}

\subsection*{External Goals}

\wss{What are your team's external goals for this project? These are not the
goals related to the functionality or quality fo the project.  These are the
goals on what the team wishes to achieve with the project.  Potential goals are
to win a prize at the Capstone EXPO, or to have something to talk about in
interviews, or to get an A+, etc.}

\subsection*{Attendance}

\subsubsection*{Expectations}

\wss{What are your team's expectations regarding meeting attendance (being on
time, leaving early, missing meetings, etc.)?}

\subsubsection*{Acceptable Excuse}

\wss{What constitutes an acceptable excuse for missing a meeting or a deadline?
What types of excuses will not be considered acceptable?}

\subsubsection*{In Case of Emergency}

\wss{What process will team members follow if they have an emergency and cannot
attend a team meeting or complete their individual work promised for a team
deliverable?}

\subsection*{Accountability and Teamwork}

\subsubsection*{Quality} 

\wss{What are your team's expectations regarding the quality
of team members' preparation for team meetings and the quality of the
deliverables that members bring to the team?}

\subsubsection*{Attitude}

\wss{What are your team's expectations regarding team members' ideas,
interactions with the team, cooperation, attitudes, and anything else regarding
team member contributions?}

\subsubsection*{Stay on Track}

\wss{What methods will be used to keep the team on track? How will your team
ensure that members contribute as expected to the team and that the team
performs as expected? How will your team reward members who do well and manage
members whose performance is below expectations?  What are the consequences for
someone not contributing their fair share?}

\wss{You may wish to use the project management metrics collected for the TA and
instructor for this.}

\wss{You can set target metrics for attendance, commits, etc.  What are the
consequences if someone doesn't hit their targets?  Do they need to bring the
coffee to the next team meeting?  Does the team need to make an appointment with
their TA, or the instructor?  Are there incentives for reaching targets early?}

\subsubsection*{Team Building}

\wss{How will you build team cohesion (fun time, group rituals, etc.)? }

\subsubsection*{Decision Making} 

\wss{How will you make decisions in your group? Consensus?  Vote? How will you
handle disagreements? }

\end{document}