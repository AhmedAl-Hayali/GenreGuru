\documentclass{article}

\usepackage{booktabs}
\usepackage{tabularx}

\title{Development Plan\\\progname}

\author{\authname}

\date{}

\input{../Comments}
%% Common Parts

\newcommand{\progname}{GenreGuru} % PUT YOUR PROGRAM NAME HERE
\newcommand{\authname}{Team 8 -- Rhythm Rangers\\
\\ Ansel Chen
\\ Muhammad Jawad
\\ Mohamad-Hassan Bahsoun
\\ Matthew Baleanu
\\ Ahmed Al-Hayali} % AUTHOR NAMES                  

\usepackage{hyperref}
    \hypersetup{colorlinks=true, linkcolor=blue, citecolor=blue, filecolor=blue,
                urlcolor=blue, unicode=false}
    \urlstyle{same}
                                


\begin{document}

\maketitle

\begin{table}[hp]
\caption{Revision History} \label{TblRevisionHistory}
\begin{tabularx}{\textwidth}{llX}
\toprule
\textbf{Date} & \textbf{Developer(s)} & \textbf{Change}\\
\midrule
2024-09-24 & All members & Complete Revision 0\\
\bottomrule
\end{tabularx}
\end{table}

\newpage{}

\wss{Put your introductory blurb here.  Often the blurb is a brief roadmap of
what is contained in the report.}

\wss{Additional information on the development plan can be found in the
\href{https://gitlab.cas.mcmaster.ca/courses/capstone/-/blob/main/Lectures/L02b_POCAndDevPlan/POCAndDevPlan.pdf?ref_type=heads}
{lecture slides}.}

\section{Confidential Information?}

\wss{State whether your project has confidential information from industry, or
not.  If there is confidential information, point to the agreement you have in
place.}

\wss{For most teams this section will just state that there is no confidential
information to protect.}
\section{IP to Protect}

\wss{State whether there is IP to protect.  If there is, point to the agreement.
All students who are working on a project that requires an IP agreement are also
required to sign the ``Intellectual Property Guide Acknowledgement.''}

\section{Copyright License}

\wss{What copyright license is your team adopting.  Point to the license in your
repo.}

\section{Team Meeting Plan}
Team meetings will be scheduled in a relatively ad-hoc fashion. Members have shared their daily schedules throughout the working week with each other, and weekly availability notes are accounted for when scheduling meetings. Availability notes are to be shared before the start of the working week so a week's meeting plan can be drafted and voted on by the end of Monday. Team meetings can occur for:
\begin{itemize}
  \item task delegation for upcoming deliverables \& discussing deliverable progress
  
  \emph{we hope to make these meetings brief and infrequent in the future by using GitHub Projects and asynchronous communication instead};
  \item work sessions to collaborate and discuss ideas (deliverable-related) synchronously, \emph{preferably in-person};
  \item pair programming;
  \item conducting deliverable reviews;
  \item conducting deliverable retrospectives, i.e., reflecting on successful and unsucessful practices used in the most recent deliverable after its completion.
\end{itemize}

\wss{How will the meetings be structured?  There should be a chair for all meetings.  There should be an agenda for all meetings.}

\section{Team Communication Plan}
All team communication is done through a discord server. The discord has three text channels:
\begin{itemize}
  \item General (anything not project-related is posted here)
  \item Important Updates (anything related to weekly meeting availability goes here)
  \item Locked In (anything related to project work goes here)
\end{itemize}
\noindent
The discord also has 2 voice channels:

\begin{itemize}
  \item Weekly Meeting (administrative meetings happen here)
  \item Locked In (collaboration of project development happens here)
\end{itemize}
\section{Team Member Roles}
\begin{itemize}
  \item Ansel Chen: Team liaison, developer
  \item Muhammad Jawad: Developer
  \item Mohamad-Hassan Bahsoun: Developer
  \item Matthew Baleanu: Developer
  \item Ahmed Al-Hayali: Meeting manager and scheduler, developer
\end{itemize}

\section{Workflow Plan}
\subsection{General Workflow}
\begin{itemize}
  \item Issues are created, assigned, and attached to the project. Issues will have a template akin to those find on \href{https://github.com/stevemao/github-issue-templates/tree/master/system/ISSUE_TEMPLATE}{stevemao/github-issue-templates}. These issues should pertain to deliverable sections, split into a completion assignee and a reviewer;
  \item The \texttt{main} branch is protected, so team members must work in independent branches. We hope to restrict branch naming to a standardized format, e.g., \href{https://dev.to/varbsan/a-simplified-convention-for-naming-branches-and-commits-in-git-il4}{this} or akin to \href{https://www.conventionalcommits.org/en/v1.0.0/}{conventional commits} (which comes with \href{https://github.com/conventional-changelog/commitlint/tree/master/%40commitlint/config-conventional}{a linter}!);
  \item Team members' commits must follow the \href{https://www.conventionalcommits.org/en/v1.0.0/}{conventional commits} standard;
  \item Whenever necessary, a team member can choose to merge their changes to the \texttt{main} branch with a pull request. Pull requests should have a template akin to the simple pull request found \href{https://graphite.dev/guides/pull-request-templates}{here} or the more complicated \href{https://github.com/dbt-labs/dbt-init/blob/master/starter-project/.github/pull_request_template.md}{data-centric template from dbt}. A pull request should be attached to the \href{https://github.com/users/AhmedAl-Hayali/projects/1}{GenreGuru Project on GitHub} with a \emph{dedicated} reviewer and potentially a review timeline and checklist. Having only one reviewer avoids the issue of \href{https://en.wikipedia.org/wiki/Diffusion_of_responsibility}{diffusion of responsiblity}.
\end{itemize}

\subsection{Usability Testing}
\begin{itemize}
	\item How will you be using git, including branches, pull request, etc.?
	\item How will you be managing issues, including template issues, issue
	classification, etc.?
  \item Use of CI/CD
\end{itemize}

\section{Project Decomposition and Scheduling}

\begin{itemize}
  \item How will you be using GitHub projects?
  \item Include a link to your GitHub project
\end{itemize}

\wss{How will the project be scheduled?  This is the big picture schedule, not
details. You will need to reproduce information that is in the course outline
for deadlines.}

\section{Proof of Concept Demonstration Plan}

\begin{description}[leftmargin=0cm]
  \item[Demonstration] Our reduced-scope demonstration should illustrate the ability to acquire data programmatically, i.e., accessing \texttt{MP3} files from \href{https://developers.deezer.com/api}{Deezer} or \href{https://developer.spotify.com/documentation/web-api}{Spotify}'s APIs, and producing a single musical feature of the song, e.g., pitch, timbre, or whatever other feature is found to be representative of songs and relatively easy to implement.
  \item[Post-demonstration] A successful demonstration will allow us to later scale up our data ingestion to satisfy multiple users' requests automatically, and provide songs' musical features on demand, expanding the available features incrementally. If the featurization/feature engineering component works correctly, a song recommendation component can later be integrated by using song features as distance metrics.
\end{description}

There is one main risk to the project --- the song data collection.

\subsection{Risks regarding song collection}
\begin{itemize}
  \item License acquisition may be necessary for some songs;
  \begin{itemize}
    \item Acquisition of the song in general may require a license from the artist, label, publisher, or platform, e.g., Spotify;
    \item Platform providing songs may have limited API access.
  \end{itemize}
  \item Songs may be only partially accessible, e.g., song snippets from Spotify;
  \item Songs may be available but we are prohibited from using them to train a machine learning model;
  \item API rate limits may restrict the number of requests we can make within a specific time period, impacting scalability and usability for demonstration.
\end{itemize}
These risks can be dismissed if the project is to use different, less strict, song providers, or tailor the project to only use non-copyrighted songs.

\section{Expected Technology}

The technologies and tools expected for this project include:

\begin{itemize}
    \item \textbf{Programming Language:} Python, due to its vast libraries in machine learning and audio processing, such as \texttt{librosa} and \texttt{pydub}.
    \item \textbf{Libraries:} 
        \begin{itemize}
            \item \texttt{librosa}: For music and audio analysis.
            \item \texttt{pydub}: For audio processing and manipulation.
            \item \texttt{scikit-learn} and \texttt{TensorFlow}: For building machine learning models to classify and generate music.
        \end{itemize}
    \item \textbf{Frameworks:} 
        \begin{itemize}
            \item \texttt{Flask} or \texttt{Django}: For the web-based interface, allowing users to interact with the system.
            \item \texttt{PyTorch} or \texttt{TensorFlow}: For implementing deep learning models to generate and classify music.
        \end{itemize}
    \item \textbf{External APIs:} Spotify API will be used for fetching song previews, features, and other metadata for recommendation purposes.
    \item \textbf{Pre-trained Models:} We may leverage some pre-trained models for audio generation, such as OpenAI's Jukebox or similar publicly available models, while customizing them to fit our needs.
    \item \textbf{Linters:} a CI-integrable Python-specific linter, e.g., \texttt{pylint}, \texttt{ruff}, or \texttt{flake8}. There also are git-specific linters like \href{https://github.com/conventional-changelog/commitlint/tree/master/%40commitlint/config-conventional}{conventional commit's linter}.
    \item \textbf{Formatter(s):} a CI-integrable formatter formatter, e.g., \texttt{black}.
    \item \textbf{Type-checker(s):} a CI-integrable type-checker, e.g., \texttt{mypy}.
    \item \textbf{Documentation Generation:} a CI-integrable documentation generator, e.g., \texttt{sphinx}.
    \item \textbf{Tester(s):} a CI-integrable testing framework, e.g., \texttt{PyTest}.
    \item \textbf{Continuous Integration:} the 5 previous ``CI-integrable'' components are a good baseline, but we plan on using a local server, and deployment onto it with CI would be sweet. As much automation with CI, within reason, is desirable as it is excellent experience to be transferred to a working setting.
    \item \textbf{Environment Management:} environment managers for Python like \texttt{pipenv} and \texttt{poetry} are excellent for ensuring the program works the same everywhere, installation becomes straightforward, and CI becomes marginally easier because of more standardization.
\end{itemize}

\wss{git, GitHub and GitHub projects should be part of your technology.}

\section{Coding Standard}
The coding standards will be a series of PEPs, but importantly \href{https://peps.python.org/pep-0008/}{PEP 8} for coding style and \href{https://peps.python.org/pep-0257/}{PEP 257} for docstrings that will form the basis for the API and user reference. For simplicity, a \href{https://stackoverflow.com/a/69842362}{package-like folder structure} should be used in conjunction with the provided template.

\section{Project Scheduling}
There will be no GANTT charts - we will conduct weekly meetings as "standups" and decide what is to be worked on. Please refer to section 1, ``Team Meeting Plan'' for details.

\newpage{}

\section*{Appendix --- Reflection}

\input{../Reflection.tex}

\begin{enumerate}
    \item Why is it important to create a development plan prior to starting the
    project?\\
    - Creating a development plan is crucial for a few reasons:
    \begin{itemize}
      \item It gives the project a clear direction and scope since all goals are outlined
      \item It sets boundaries on the project to prevent unplanned expansions
      \item It outlines anticipated challenges and contingency plans
      \item It provides transparent communication of the project expectations to the stakeholders
    \end{itemize}
    \item In your opinion, what are the advantages and disadvantages of using CI/CD?
    
    CI/CD is a wonderful tool that automates mind-numbing tasks of linting, styling, formatting, deployment, and document generation, for example, but it comes at the heavy cost of relatively complicated setup process for said CI/CD to work. Thankfully, there are many GitHub actions templates online, so we often need not to worry about starting from scratch.
    \item What disagreements did your group have in this deliverable, if any, and how did you resolve them?
    
    There were few disagreements, but the frequency of meetings is a hotly-contested topic. We chose to have relatively frequent meetings during the first 3 weeks of the project, will experiment with far fewer in the future and rely on asynchronous communication. Depending on the success of it during the next few weeks, we will continue or adjust it to better suit our needs in completing future deliverables succesfully.
\end{enumerate}

\newpage{}

\section*{Appendix --- Team Charter}

\wss{borrows from
\href{https://engineering.up.edu/industry_partnerships/files/team-charter.pdf}
{University of Portland Team Charter}}

\subsection*{External Goals}

\begin{itemize}
  \item Make a project that can be but on a resumé
  \item Learn software engineering industry standards
  \item Follow the engineering process
  \begin{itemize}
    \item Draft documentation
    \item Conduct research
    \item Research Documentation
    \item Prototyping
    \item Implementation
  \end{itemize}
  \item Learn signal processing
  \item Have a project we can work on after graduating
  \item Impress peers (interviewers, fellow peers at the EXPO)
  \item Gain familiarity with development tools and frameworks
\end{itemize}

\subsection*{Attendance}

\subsubsection*{Expectations}

\wss{What are your team's expectations regarding meeting attendance (being on
time, leaving early, missing meetings, etc.)?}
\begin{itemize}
  \item Administrator schedules meeting (time, location)
  \item Administrator outlines meeting agenda
  \item Administrator creates team meeting issue
  \item Administrator completes meeting minutes
  \item Team members are expected to attend most work sessions
  \item All team members are expected to attend TA deliverable feedback meetings
  \item All concerned team members should attend necessary deliverable check-in meetings
  \item Administrator must give 'reasonable' advanced notice to meeting attendees (at least 24 hours in advance unless team members agree to scheduling a time earlier)
  \item All members are expected to attend and take initiative in the issue provision process when delegating work
  \item Members to attend a meeting during the scheduled lecture or tutorial times (except for Friday) if that meeting is arranged before 10pm the night prior
\end{itemize}

\subsubsection*{ Acceptable Excuse}

\begin{itemize}
  \item If the task assigned to a team member is independent of others' contributions, the group can elect the member to miss the work session 
\end{itemize}
\subsubsection*{In Case of Emergency}
\begin{itemize}
  \item If an emergency arises for a member, they are responsible for delegating their tasks to be completed by others. However, if those tasks are not completed, the person is responsible for repurcussions
  \item If an emergency arises for a member, they must notify the group on discord as soon as possible, and they must reach out to that week's administrator to catch up on the missed meeting
\end{itemize}

\subsection*{Accountability and Teamwork}

\subsubsection*{Quality} 

\wss{What are your team's expectations regarding the quality
of team members' preparation for team meetings and the quality of the
deliverables that members bring to the team?}
\begin{itemize}
  \item The meeting administrator must prepare the meeting agenda and attach it to the meeting issue at least 2 hours before the scheduled meeting time
  \item For a work session, the administrator must prepare a list of tasks to be completed before and during the work session
  \item For a work session, the atendees are expected to complete the necessary work prior to arrival and update the administrator on tasks to be completed during the work session
  \item For a check-in, the administrator must prepare a list of tasks to be completed before and during the check-in
  \item For a check-in, the atendees are expected to complete the necessary work prior to arrival and update the administrator on tasks to be completed during the check-in
  \item For a delegation meeting, the administrator must prepare a list of issues that must be delegated
  \item For a delegation meeting, the atendees are expected to take initiative and assume responsibility for all presented issues
  \item The member who is delegated an issue should look into the deliverable checklist and rubric on avenue, then generate a checklist for the PR reviewer to consult as they review the PR
\end{itemize}

\subsubsection*{Attitude}

All members are expected an enjoyable work environment.

\subsubsection*{Stay on Track}
\begin{itemize}
  \item The github project board and meetings will keep the team on track
  \item Issue completion and status checks during meetings will ensure that members contribute as expected and that the team performs as expected
  \item Verbal affirmations will boost team morale after completing issues at a high quality
  \item Recurrent sub-expectation performance from a member will be raised at the following delegation meeting
  \item If a team member recurrently under-performs and fails to meet consecutive check-ins without justification, their performance will be raised to the TA and professor Smith
\end{itemize}

\subsubsection*{Team Building}

The team will sometimes have a team outing (e.g team lunch, dinner, gaming session)

\subsubsection*{Decision Making} 

\noindent
Decisions will primarily be made through a discussion, followed by a consensus. If that cannot be achieved, a vote will be cast where the majority rules

\end{document}