\documentclass{article}

\usepackage{booktabs}
\usepackage{tabularx}

\title{Development Plan\\\progname}

\author{\authname}

\date{}

%% Comments

\usepackage{color}

\newif\ifcomments\commentstrue %displays comments
%\newif\ifcomments\commentsfalse %so that comments do not display

\ifcomments
\newcommand{\authornote}[3]{\textcolor{#1}{[#3 ---#2]}}
\newcommand{\todo}[1]{\textcolor{red}{[TODO: #1]}}
\else
\newcommand{\authornote}[3]{}
\newcommand{\todo}[1]{}
\fi

\newcommand{\wss}[1]{\authornote{blue}{SS}{#1}} 
\newcommand{\plt}[1]{\authornote{magenta}{TPLT}{#1}} %For explanation of the template
\newcommand{\an}[1]{\authornote{cyan}{Author}{#1}}

%% Common Parts

\newcommand{\progname}{Software Engineering} % PUT YOUR PROGRAM NAME HERE
\newcommand{\authname}{Team 8 -- Rhythm Rangers\\
\\ Ansel Chen
\\ Muhammad Jawad
\\ Mohamad-Hassan Bahsoun
\\ Matthew Baleanu
\\ Ahmed Al-Hayali} % AUTHOR NAMES                  

\usepackage{hyperref}
    \hypersetup{colorlinks=true, linkcolor=blue, citecolor=blue, filecolor=blue,
                urlcolor=blue, unicode=false}
    \urlstyle{same}
                                


\begin{document}

\maketitle

\begin{table}[hp]
\caption{Revision History} \label{TblRevisionHistory}
\begin{tabularx}{\textwidth}{llX}
\toprule
\textbf{Date} & \textbf{Developer(s)} & \textbf{Change}\\
\midrule
Date1 & Name(s) & Description of changes\\
Date2 & Name(s) & Description of changes\\
... & ... & ...\\
\bottomrule
\end{tabularx}
\end{table}

\wss{Put your introductory blurb here.}

\section{Team Meeting Plan}
Team Meeting objective(s):
\begin{itemize}
  \item Discuss and delegate tasks for upcoming deliverable(s)
  \item Group work sessions on upcoming deliverable(s)
\end{itemize}
\noindent
Team meetings are schedulied throughout the week, at the start of the week, via a discord poll.
Each team member is expected to respond to each poll. If a team member confirms that they can attend, then
they are expected to attend.
\section{Team Communication Plan}
All team communication is done through a WhatsApp group chat and a discord server. The discord has three text
channels:
\begin{itemize}
  \item General (anything not project-related is posted here)
  \item Important Updates (anything related to weekly meeting availability goes here)
  \item Locked In (anything related to project work goes here)
\end{itemize}
\noindent
The discord also has 2 voice channels:

\begin{itemize}
  \item Weekly Meeting (administrative meetings happen here)
  \item Locked In (collaboration of project development happens here)
\end{itemize}
\noindent
All other communications happen through the WhatsApp group chat
\section{Team Member Roles}
\begin{itemize}
  \item Ahmed: Meeting manager + scheduler, developer
  \item Ansel: Team liaison, developer
  \item Matthew: Developer
  \item Mohammed-Hassan: Developer
  \item Muhammad: Developer
\end{itemize}

\section{Workflow Plan}

\begin{itemize}
	\item Git standards
  \begin{itemize}
    \item Development branches will follow the "dev/*name*/*description*" naming convention
    \item Documentation branches will follow the "docs/*name*/*description*" naming convention
    \item A feature branch will be created for every major component of the project
    \item Developers will base their branches off the feature branch they are contributing towards
    \item Developers will create pull requests that concisely explain what the pull request is doing,
    with any useful information being written in the PR description
  \end{itemize}
	\item Issue standards
  \begin{itemize}
    \item Issues are created during usability testing
    \item Issues must contain a comparison between what is expected of the code
    and what is actually seen by the user
    \item If possible, issue creators should attach any log files they collect to the issue
    \item The issue must be assigned to the developer who originally pushed
    the code that is causing the bug
  \end{itemize}
\end{itemize}

\section{Proof of Concept Demonstration Plan}

What is the main risk, or risks, for the success of your project?  What will you
demonstrate during your proof of concept demonstration to convince yourself that
you will be able to overcome this risk?

\section{Expected Technology}

\wss{What programming language or languages do you expect to use?  What external
libraries?  What frameworks?  What technologies.  Are there major components of
the implementation that you expect you will implement, despite the existence of
libraries that provide the required functionality.  For projects with machine
learning, will you use pre-trained models, or be training your own model?  }

\wss{The implementation decisions can, and likely will, change over the course
of the project.  The initial documentation should be written in an abstract way;
it should be agnostic of the implementation choices, unless the implementation
choices are project constraints.  However, recording our initial thoughts on
implementation helps understand the challenge level and feasibility of a
project.  It may also help with early identification of areas where project
members will need to augment their training.}

Topics to discuss include the following:

\begin{itemize}
\item Specific programming language
\item Specific libraries
\item Pre-trained models
\item Specific linter tool (if appropriate)
\item Specific unit testing framework
\item Investigation of code coverage measuring tools
\item Specific plans for Continuous Integration (CI), or an explanation that CI
  is not being done
\item Specific performance measuring tools (like Valgrind), if
  appropriate
\item Tools you will likely be using?
\end{itemize}

\section{Coding Standard}

\section{Project Scheduling}

\wss{How will the project be scheduled?}

\newpage{}

\section*{Appendix --- Reflection}

\wss{Not required for CAS 741}

The purpose of reflection questions is to give you a chance to assess your own
learning and that of your group as a whole, and to find ways to improve in the
future. Reflection is an important part of the learning process.  Reflection is
also an essential component of a successful software development process.  

Reflections are most interesting and useful when they're honest, even if the
stories they tell are imperfect. You will be marked based on your depth of
thought and analysis, and not based on the content of the reflections
themselves. Thus, for full marks we encourage you to answer openly and honestly
and to avoid simply writing ``what you think the evaluator wants to hear.''

Please answer the following questions.  Some questions can be answered on the
team level, but where appropriate, each team member should write their own
response:


\begin{enumerate}
    \item Why is it important to create a development plan prior to starting the
    project?
    \item In your opinion, what are the advantages and disadvantages of using
    CI/CD?
    \item What disagreements did your group have in this deliverable, if any,
    and how did you resolve them?
\end{enumerate}

\end{document}