\documentclass[12pt, titlepage]{article}

\usepackage{amsmath, mathtools}

\usepackage[round]{natbib}
\usepackage{amsfonts}
\usepackage{amssymb}
\usepackage{graphicx}
\usepackage{colortbl}
\usepackage{xr}
\usepackage{hyperref}
\usepackage{longtable}
\usepackage{xfrac}
\usepackage{tabularx}
\usepackage{float}
\usepackage{siunitx}
\usepackage{booktabs}
\usepackage{multirow}
\usepackage[section]{placeins}
\usepackage{caption}
\usepackage{fullpage}
\usepackage{soul}
\usepackage{xcolor}
\usepackage[normalem]{ulem} 

\hypersetup{
bookmarks=true,     % show bookmarks bar?
colorlinks=true,       % false: boxed links; true: colored links
linkcolor=red,          % color of internal links (change box color with linkbordercolor)
citecolor=blue,      % color of links to bibliography
filecolor=magenta,  % color of file links
urlcolor=cyan          % color of external links
}

\usepackage{array}

\externaldocument{../../SRS/SRS}

%% Comments

\usepackage{color}

\newif\ifcomments\commentstrue %displays comments
%\newif\ifcomments\commentsfalse %so that comments do not display

\ifcomments
\newcommand{\authornote}[3]{\textcolor{#1}{[#3 ---#2]}}
\newcommand{\todo}[1]{\textcolor{red}{[TODO: #1]}}
\else
\newcommand{\authornote}[3]{}
\newcommand{\todo}[1]{}
\fi

\newcommand{\wss}[1]{\authornote{blue}{SS}{#1}} 
\newcommand{\plt}[1]{\authornote{magenta}{TPLT}{#1}} %For explanation of the template
\newcommand{\an}[1]{\authornote{cyan}{Author}{#1}}

%% Common Parts

\newcommand{\progname}{Software Engineering} % PUT YOUR PROGRAM NAME HERE
\newcommand{\authname}{Team 8 -- Rhythm Rangers\\
\\ Ansel Chen
\\ Muhammad Jawad
\\ Mohamad-Hassan Bahsoun
\\ Matthew Baleanu
\\ Ahmed Al-Hayali} % AUTHOR NAMES                  

\usepackage{hyperref}
    \hypersetup{colorlinks=true, linkcolor=blue, citecolor=blue, filecolor=blue,
                urlcolor=blue, unicode=false}
    \urlstyle{same}
                                


\begin{document}

\title{Module Interface Specification for \progname{}}

\author{\authname}

\date{\today}

\maketitle

\pagenumbering{roman}

\section{Revision History}

\begin{tabularx}{\textwidth}{p{3cm}p{2cm}X}
\toprule {\bf Date} & {\bf Version} & {\bf Notes}\\
\midrule
Date 1 & 1.0 & Notes\\
Date 2 & 1.1 & Notes\\
\bottomrule
\end{tabularx}

~\newpage

\section{Symbols, Abbreviations and Acronyms}

See SRS Documentation at \wss{give url}

\wss{Also add any additional symbols, abbreviations or acronyms}

\newpage

\tableofcontents

\newpage

\pagenumbering{arabic}

\section{Introduction}

The following document details the Module Interface Specifications for
\wss{Fill in your project name and description}

Complementary documents include the System Requirement Specifications
and Module Guide.  The full documentation and implementation can be
found at \url{...}.  \wss{provide the url for your repo}

\section{Notation}

\wss{You should describe your notation.  You can use what is below as
  a starting point.}

The structure of the MIS for modules comes from \citet{HoffmanAndStrooper1995},
with the addition that template modules have been adapted from
\cite{GhezziEtAl2003}.  The mathematical notation comes from Chapter 3 of
\citet{HoffmanAndStrooper1995}.  For instance, the symbol := is used for a
multiple assignment statement and conditional rules follow the form $(c_1
\Rightarrow r_1 | c_2 \Rightarrow r_2 | ... | c_n \Rightarrow r_n )$.

The following table summarizes the primitive data types used by \progname. 

\begin{center}
\renewcommand{\arraystretch}{1.2}
\noindent 
\begin{tabular}{l l p{7.5cm}} 
\toprule 
\textbf{Data Type} & \textbf{Notation} & \textbf{Description}\\ 
\midrule
character & char & a single symbol or digit\\
integer & $\mathbb{Z}$ & a number without a fractional component in (-$\infty$, $\infty$) \\
natural number & $\mathbb{N}$ & a number without a fractional component in [1, $\infty$) \\
real & $\mathbb{R}$ & any number in (-$\infty$, $\infty$)\\
\bottomrule
\end{tabular} 
\end{center}

\noindent
The specification of \progname \ uses some derived data types: sequences, strings, and
tuples. Sequences are lists filled with elements of the same data type. Strings
are sequences of characters. Tuples contain a list of values, potentially of
different types. In addition, \progname \ uses functions, which
are defined by the data types of their inputs and outputs. Local functions are
described by giving their type signature followed by their specification.

\section{Module Decomposition}

The following table is taken directly from the Module Guide document for this project.

\begin{table}[h!]
\centering
\begin{tabular}{p{0.3\textwidth} p{0.6\textwidth}}
\toprule
\textbf{Level 1} & \textbf{Level 2}\\
\midrule

{Hardware-Hiding} & ~ \\
\midrule

\multirow{7}{0.3\textwidth}{Behaviour-Hiding} & GUI Module\\
& Audio File Upload Module\\
& Search Query Module\\
& Client Communication Module\\
%server side
& Server Communication Module\\
& Driver Module\\
%feature extractions
& Tempo (BPM) Feature Extraction Module\\
& Key and Scale Feature Extraction Module\\
& \textcolor{red}{\st{Instrument Type Feature Extraction Module}}\\
& \textcolor{red}{\st{Vocal Gender Feature Extraction Module}}\\
& \textcolor{red}{\st{Mood Feature Extraction Module}}\\
& Dynamic Range Feature Extraction Module\\
& \textcolor{red}{RMS feature Module}\\
& Instrumentalness Feature Extraction Module\\
& \textcolor{red}{\st{Contour Feature Extraction Module}}\\
& \textcolor{red}{Spectral Centroid Feature Module}\\
& \textcolor{red}{Spectral Bandwidth Feature Module}\\
& \textcolor{red}{Spectral Rolloff Feature Module}\\
& \textcolor{red}{Spectral Flux Feature Module}\\
& \textcolor{red}{Spectral Contrast Feature Module}\\
%final pieces
& Recommendation Module\\
& Program Results Interface Module\\
\midrule

\multirow{3}{0.3\textwidth}{Software Decision} & Database\\
& Spotify API\\
& Deezer API\\
& \textcolor{red}{\st{Genre Feature Module}}\\
& \textcolor{red}{Spleeter Audio Splitter Module}\\
\bottomrule

\end{tabular}
\caption{Module Hierarchy}
\label{TblMH}
\end{table}

\newpage
~\newpage

\section{GUI Module} \label{Module}

\subsection{GUI Module}
\textbf{gui}

\subsection{Uses}
\begin{itemize}
  \item Audio File Input Module
  \item Search Query Module
  \item Spotify API Module
  \item \textcolor{red}{Deezer API Module}
\end{itemize}

\subsection{Syntax}

\subsubsection{Exported Constants}
N/A

\subsubsection{Exported Access Programs}

\begin{center}
\begin{tabular}{p{2cm} p{4cm} p{4cm} p{2cm}}
\hline
\textbf{Name} & \textbf{In} & \textbf{Out} & \textbf{Exceptions}\\
\hline%by results, at this stage, we mean the output of the final program (could include features, includes recs)
gui &N/A &N/A &-\\

\hline
\end{tabular}
\end{center}

\subsection{Semantics}

\subsubsection{State Variables}
\begin{itemize}
  \item user\_selection: Stores the track or audio file chosen by the user
  \item spotify\_results: Stores the top 10 songs that best fit the search query
  \item recommendations: Stores the list of the recommended songs after feature extraction
\end{itemize}

\subsubsection{Environment Variables}
\begin{itemize}
  \item Keyboard
  \item Mouse
  \item Screen
\end{itemize}

\subsubsection{Assumptions}
\begin{itemize}
  \item User inputs are valid
\end{itemize}

\subsubsection{Access Routine Semantics}
\noindent gui
\begin{itemize}
  \item transition: provides methods to build and deploy the GUI to the user
\end{itemize}
\subsubsection{Local Functions}
N/A

%subinput 1
\section{MIS of Audio File \textcolor{red}{Upload} Module} 

\subsection{Audio File Input Module}
\textbf{audioFileIM}

\subsection{Uses}
\begin{itemize}
  \item GUI Module
  \item Client Communication Module
\end{itemize}

\subsection{Syntax}

\subsubsection{Exported Constants}
N/A

\subsubsection{Exported Access Programs}

\begin{center}
\begin{tabular}{p{2cm} p{4cm} p{4cm} p{2cm}}
\hline
\textbf{Name} & \textbf{In} & \textbf{Out} & \textbf{Exceptions}\\
\hline%by results, at this stage, we mean the output of the final program (could include features, includes recs)
audioFileIM &Audio File &Track reference &Invalid File Type\\

\hline
\end{tabular}
\end{center}

\subsection{Semantics}

\subsubsection{State Variables}
\begin{itemize}
  \item user\_af\_input: path to the audio file currently being processed
\end{itemize}

\subsubsection{Environment Variables}
N/A 

\subsubsection{Assumptions}
\begin{itemize}
  \item User has a properly named Audio File.
  \item User audio file input is actually a song. 
\end{itemize}

\subsubsection{Access Routine Semantics}

\noindent audioFileIM
\begin{itemize}
  \item transition: if the provided file is not in the .wav, then after it is converted, the file is sent to the Client Communication Module 
  \end{itemize}

% \wss{A module without environment variables or state variables is unlikely to
%   have a state transition.  In this case a state transition can only occur if
%   the module is changing the state of another module.}

% \wss{Modules rarely have both a transition and an output.  In most cases you
%   will have one or the other.}

\subsubsection{Local Functions}
N/A

% \wss{As appropriate} \wss{These functions are for the purpose of specification.
%   They are not necessarily something that is going to be implemented
%   explicitly.  Even if they are implemented, they are not exported; they only
%   have local scope.}

%subinput 2
\section{MIS of Search Query Module} 

\subsection{Search Query Module}
\textbf{searchQuery}
\subsection{Uses}
\begin{itemize}
  \item GUI Module
  \item Client Communication Module
\end{itemize}

\subsection{Syntax}

\subsubsection{Exported Constants}
N/A

\subsubsection{Exported Access Programs}

\begin{center}
\begin{tabular}{p{2cm} p{4cm} p{4cm} p{2cm}}
\hline
\textbf{Name} & \textbf{In} & \textbf{Out} & \textbf{Exceptions}\\
\hline%by results, at this stage, we mean the output of the final program (could include features, includes recs)
searchQuery &input\_text &Spotify Query &-\\
\hline
\end{tabular}
\end{center}

\subsection{Semantics}

\subsubsection{State Variables}
\begin{itemize}
  \item user\_sq\_input: stores the query being processed 
\end{itemize}

\subsubsection{Environment Variables}
\begin{itemize}
  \item Spotify Client ID
  \item Spotify Client Secret
\end{itemize}

\subsubsection{Assumptions}
N/A

\subsubsection{Access Routine Semantics}

\noindent searchQuery
\begin{itemize}
\item transition: Takes the text input and/or Spotify ID from the GUI Module, and builds 
the query to be sent to the Client Communication Module
% \item output: \wss{if appropriate} 
% \item exception: \wss{if appropriate} 
\end{itemize}

% \wss{A module without environment variables or state variables is unlikely to
%   have a state transition.  In this case a state transition can only occur if
%   the module is changing the state of another module.}

% \wss{Modules rarely have both a transition and an output.  In most cases you
%   will have one or the other.}

\subsubsection{Local Functions}
N/A
% \wss{As appropriate} \wss{These functions are for the purpose of specification.
%   They are not necessarily something that is going to be implemented
%   explicitly.  Even if they are implemented, they are not exported; they only
%   have local scope.}

%server compute
\section{MIS of Client Communication Module} 

\subsection{Client Communication Module}
The module that sends request to and receives responses from the server

\subsection{Uses}
\begin{itemize}
  \item Audio File Input Module
  \item Search Query Module
  \item Server Communication Module
\end{itemize}

\subsection{Syntax}

\subsubsection{Exported Constants}
N/A

\subsubsection{Exported Access Programs}

\begin{center}
\begin{tabular}{p{3cm} p{4cm} p{4cm} p{2cm}}
\hline
\textbf{Name} & \textbf{In} & \textbf{Out} & \textbf{Exceptions}\\
\hline%by results, at this stage, we mean the output of the final program (could include features, includes recs)
send\_request & request (ADT) &- &-\\
await\_response &- &response (ADT) &-\\
\hline
\end{tabular}
\end{center}

\subsection{Semantics}

\subsubsection{State Variables}
N/A

\subsubsection{Environment Variables}
N/A

\subsubsection{Assumptions}
N/A

\subsubsection{Access Routine Semantics}

\noindent send\_request():
\begin{itemize}
\item transition: sends the request to the server, where it is received by the server communication module 
\end{itemize}
\noindent await\_response():
\begin{itemize} 
  \item output: gets the response from the server communication module and sends it to the Program Results Interface Module  
\end{itemize}

\subsubsection{Local Functions}
N/A

%separator
\section{MIS of Server Communication Module} 

\subsection{Server Communication Module}
Sends requests to the server and receives responses from the server

\subsection{Uses}
\begin{itemize}
  \item Server Driver Module
  \item Client Communication Module
\end{itemize}

\subsection{Syntax}

\subsubsection{Exported Constants}
N/A

\subsubsection{Exported Access Programs}

\begin{center}
\begin{tabular}{p{3cm} p{3cm} p{3cm} p{2cm}}
\hline
\textbf{Name} & \textbf{In} & \textbf{Out} & \textbf{Exceptions}\\
\hline%by results, at this stage, we mean the output of the final program (could include features, includes recs)
send\_response & response (ADT) &- &-\\
await\_request &- &request (ADT) &-\\
\hline
\end{tabular}
\end{center}

\subsection{Semantics}

\subsubsection{State Variables}
N/A

\subsubsection{Environment Variables}
N/A

\subsubsection{Assumptions}
N/A

\subsubsection{Access Routine Semantics}

\noindent send\_response():
\begin{itemize}
\item transition: sends the response to the client, where it is received by the Client Communication module 
\end{itemize}
\noindent await\_request():
\begin{itemize} 
  \item output: gets the request from the Client Communication module and sends it to the Server Driver Module  
\end{itemize}

\subsubsection{Local Functions}
N/A

%separator
\section{MIS of Driver Module} 

\subsection{Driver Module}
Controls all the functions of the server
\subsection{Uses}
\begin{itemize}
  \item Featurizer Module
  \item Server Communication Module
  \item Database Module
  \item Recommendation Module
  \item \textcolor{red}{\sout{Deezer API Module}}
\end{itemize}

\subsection{Syntax}

\subsubsection{Exported Constants}
N/A

\subsubsection{Exported Access Programs}

\begin{center}
\begin{tabular}{p{2cm} p{4cm} p{4cm} p{2cm}}
\hline
\textbf{Name} & \textbf{In} & \textbf{Out} & \textbf{Exceptions}\\
\hline%by results, at this stage, we mean the output of the final program (could include features, includes recs)
- &- &- &-\\ 
\hline
\end{tabular}
\end{center}

\subsection{Semantics}

\subsubsection{State Variables}
N/A

\subsubsection{Environment Variables}
\begin{itemize}
  \item \textcolor{red}{\sout{Deezer App ID}}
  \item \textcolor{red}{\sout{Deezer Secret}}
\end{itemize}

\subsubsection{Assumptions}
N/A

\subsubsection{Access Routine Semantics}

\noindent main():
\begin{itemize}
\item transition: Connects all server-side modules together 
\end{itemize}

\subsubsection{Local Functions}
N/A  

% MIS of Audio Lookup Module
\section{MIS of Audio Lookup Module} \label{Module:AudioLookupModule}

\subsection{Module}
Audio Lookup Module

\subsection{Uses}
- Driver Module: Receives the International Standard Recording Code (ISRC) from the Driver Module.
- Deezer API: Responsible for retrieving the audio file, genre, and associated metadata for the provided ISRC.

\subsection{Syntax}

\subsubsection{Exported Constants}
None.

\subsubsection{Exported Access Programs}

\begin{center}
\begin{tabular}{p{2cm} p{4cm} p{4cm} p{2cm}}
\hline
\textbf{Name} & \textbf{In} & \textbf{Out} & \textbf{Exceptions} \\
\hline
getAudioDetails & isrc: String & audioDetails: AudioDetails & AuthenticationFailure, APIRequestError \\
\hline
\end{tabular}
\end{center}

\subsection{Semantics}

\subsubsection{State Variables}
- \texttt{isrc}: The International Standard Recording Code for identifying the requested song.
- \texttt{authToken}: The authentication token used for accessing the Deezer API.
- \texttt{audioDetails}: A structure containing the audio file, genre, and other metadata.

\subsubsection{Environment Variables}
- The Audio Lookup Module interacts with the Deezer API over the internet to fetch the requested audio file, genre, and metadata.

\subsubsection{Assumptions}
- The ISRC provided by the Driver Module is valid and corresponds to an existing song.
- The authentication token for the Deezer API is valid and not expired.
- The Deezer API is available and operational at the time of the request.

\subsubsection{Access Routine Semantics}

\noindent \textbf{getAudioDetails}(isrc: String):
\begin{itemize}
\item \textbf{Transition:} 
    - Authenticates with the Deezer API using \texttt{authToken}.
    - Sends a request to the Deezer API with the provided ISRC to retrieve the audio file, genre, and metadata.
\item \textbf{Output:} 
    - Returns the \texttt{audioDetails} structure, which includes:
      \begin{itemize}
        \item \texttt{audioFile}: The retrieved audio file.
        \item \texttt{genre}: The genre of the song.
        \item \texttt{metadata}: Additional metadata such as song title, artist, and album information.
      \end{itemize}
\item \textbf{Exceptions:} 
    - \texttt{AuthenticationFailure}: Raised if the API authentication fails (e.g., invalid or expired token).
    - \texttt{APIRequestError}: Raised if there is an issue with the API request, such as a network error or invalid ISRC.
\end{itemize}

\subsubsection{Local Functions}

\textbf{authenticateWithDeezer}:
\begin{itemize}
\item Purpose: Handles authentication with the Deezer API and retrieves a valid \texttt{authToken}.
\item Input: None.
\item Output: \texttt{authToken}.
\end{itemize}

\textbf{fetchAudioFile}:
\begin{itemize}
\item Purpose: Sends the ISRC to the Deezer API and retrieves the corresponding audio file.
\item Input: \texttt{isrc}.
\item Output: \texttt{audioFile}.
\end{itemize}

\textbf{fetchGenreAndMetadata}:
\begin{itemize}
\item Purpose: Retrieves the genre and metadata associated with the song from the Deezer API.
\item Input: \texttt{isrc}.
\item Output: \texttt{genre}, \texttt{metadata}.
\end{itemize}

% MIS of Featurizer Module
\section{MIS of Featurizer Module}

\subsection{Featurizer Module}
The Featurizer Module is responsible for extracting 9 distinct feature values from audio files:
\begin{itemize}
    \item Tempo
    \item Key and Scale
    \item \textcolor{red}{\sout{Instrument Type}}
    \item \textcolor{red}{\sout{Vocal Gender}}
    \item Dynamic Range
    \item \textcolor{red}{RMS}
    \item Instrumentalness
    \item \textcolor{red}{Spectral Centroid}
    \item \textcolor{red}{Spectral Bandwidth}
    \item \textcolor{red}{Spectral Rolloff}
    \item \textcolor{red}{Spectral Flux}
    \item \textcolor{red}{Spectral Contrast}
    \item \textcolor{red}{\sout{Contour}}
    \item \textcolor{red}{\sout{Mood}}
    \item \textcolor{red}{\sout{Genre}}
\end{itemize}

The module invokes sub-feature modules to compute these feature values. It consolidates the results into a single \texttt{FeatureValues} object and returns it to the Driver Module.

\subsection{Uses}
- **Driver Module**: Sends requests to the Featurizer Module and receives feature values.
- **Sub-Feature Modules**: Each responsible for computing a specific feature (e.g., Tempo, Key and Scale).

\subsection{Syntax}

\subsubsection{Exported Constants}
None.

\subsubsection{Exported Access Programs}

\begin{center}
\begin{tabular}{p{2cm} p{4cm} p{4cm} p{2cm}}
\hline
\textbf{Name} & \textbf{In} & \textbf{Out} & \textbf{Exceptions}\\
\hline
extractFeatures & audioFile: AudioFile & featureValues: FeatureValues & UnsupportedFileFormatException \\
\hline
\end{tabular}
\end{center}

\subsection{Semantics}

\subsubsection{State Variables}
- \texttt{audioFile}: The input audio file provided for feature extraction.
- \texttt{featureValues}: An object containing the extracted values for all 9 features.

\subsubsection{Environment Variables}
None.

\subsubsection{Assumptions}
- Input audio files are in supported formats (e.g., WAV, MP3).
- All sub-feature modules are functional and return valid outputs for their respective features.

\subsubsection{Access Routine Semantics}
\textbf{extractFeatures}:
\begin{itemize}
    \item \textbf{Precondition:} 
    \begin{itemize}
        \item \texttt{audioFile} is a valid audio file in a supported format.
    \end{itemize}
    \item \textbf{Postcondition:}
    \begin{itemize}
        \item \texttt{featureValues} contains valid results for all 9 features:
            \begin{itemize}
                \item Tempo
                \item Key and Scale
                \item \textcolor{red}{\sout{Instrument Type}}
                \item \textcolor{red}{\sout{Vocal Gender}}
                \item Dynamic Range
                \item \textcolor{red}{RMS}
                \item Instrumentalness
                \item \textcolor{red}{Spectral Centroid}
                \item \textcolor{red}{Spectral Bandwidth}
                \item \textcolor{red}{Spectral Rolloff}
                \item \textcolor{red}{Spectral Flux}
                \item \textcolor{red}{Spectral Contrast}
                \item \textcolor{red}{\sout{Contour}}
                \item \textcolor{red}{\sout{Mood}}
                \item \textcolor{red}{\sout{Genre}}
            \end{itemize}
        \item If the input file format is unsupported, an \texttt{UnsupportedFileFormatException} is raised.
    \end{itemize}
\end{itemize}

\subsubsection{Local Functions}

\textcolor{red}{\textbf{ProcessAudio}:}
\begin{itemize}
    \item \textcolor{red}{Converts the audio file input into a normalized audio time series.}
    \item \textcolor{red}{Input: \texttt{audioFile}}
    \item \textcolor{red}{Output: \texttt{AudioTimeSeries, Vocal\_Signal, Non\_Vocal\_Signal}}
\end{itemize}

\textcolor{red}{\textbf{Divide Signal}:}
\begin{itemize}
    \item \textcolor{red}{Computes the number of windows (for STFT) based on the audio time series length, divides the signal into that many pieces}
    \item \textcolor{red}{Input: \texttt{AudioTimeSeries}}
    \item \textcolor{red}{Output: \texttt{Divided\_Audio\_Time\_Series, Number\_Of\_Windows, Beats\_Per\_Window}}
\end{itemize}

\textcolor{red}{\textbf{Divide STFT}:}
\begin{itemize}
    \item \textcolor{red}{Computes the STFT and STFT magnitudes of the divided signal}
    \item \textcolor{red}{Input:  \texttt{Divided\_Audio\_Time\_Seriess}}
    \item \textcolor{red}{Output: \texttt{Divided\_STFT\_Signal, Divded\_STFT\_Magnitudes}}
\end{itemize}

\textbf{invokeSubFeatureModule}:
\begin{itemize}
    \item Purpose: Calls a specific sub-feature module (e.g., for Tempo, Genre) and retrieves its computed value.
    \item Input: \texttt{audioFile}, \texttt{featureType}
    \item Output: Value of the requested feature.
\end{itemize}

\textbf{aggregateFeatureValues}:
\begin{itemize}
    \item Purpose: Consolidates all feature values into a \texttt{FeatureValues} object.
    \item Input: A list of feature values retrieved from sub-feature modules.
    \item Output: \texttt{FeatureValues} object.
\end{itemize}

%MIS of Feature Extraction algorithms
%MIS of Tempo(BPM) extraction
\section{MIS of Tempo (BPM) Feature Extraction Module} 

\subsection{Tempo (BPM) Feature Extraction Module}

\subsection{Uses}
N/A

\subsection{Syntax}

\subsubsection{Exported Constants}
N/A

\subsubsection{Exported Access Programs}

\begin{center}
\begin{tabular}{p{2cm} p{4cm} p{4cm} p{2cm}}
\hline
\textbf{Name} & \textbf{In} & \textbf{Out} & \textbf{Exceptions}\\
\hline%by results, at this stage, we mean the output of the final program (could include features, includes recs)
\texttt{Extract Tempo} &\texttt{Audio time series (np.ndarray)} &\texttt{Song Tempo} $\in \mathbb{R}$ &- \\
\hline
\end{tabular}
\end{center}

\subsection{Semantics}

\subsubsection{State Variables}
N/A

\subsubsection{Environment Variables}
N/A

\subsubsection{Assumptions}
Valid audio file with coherent song information.

\subsubsection{Access Routine Semantics}

\noindent \texttt{ExtractTempo()}:
\begin{itemize}
\item transition: N/A
\item output: \texttt{Song\textunderscore Tempo} : = \texttt{ExtractTempo(Audio\textunderscore Time\textunderscore Series)}
\item exception: N/A
\end{itemize}

\subsubsection{Local Functions}
N/A



%Key and Scale Feature Extraction Module
\section{MIS of Key and Scale Feature Extraction Module} 

\subsection{Key and Scale Feature Extraction Module}

\subsection{Uses}
N/A

\subsection{Syntax}

\subsubsection{Exported Constants}
N/A

\subsubsection{Exported Access Programs}

\begin{center}
\begin{tabular}{p{2cm} p{4cm} p{4cm} p{2cm}}
\hline
\textbf{Name} & \textbf{In} & \textbf{Out} & \textbf{Exceptions}\\
\hline%by results, at this stage, we mean the output of the final program (could include features, includes recs)
\texttt{Extract Key \& Scale} &\texttt{Audio time series (np.ndarray)} &\texttt{Song Key, Scale} $\in \mathbb{Z}^2$ &-\\
\hline
\end{tabular}
\end{center}

\subsection{Semantics}

\subsubsection{State Variables}
N/A

\subsubsection{Environment Variables}
N/A

\subsubsection{Assumptions}
Valid audio file with coherent song information.

\subsubsection{Access Routine Semantics}

\noindent \texttt{ExtractKeyScale()}:
\begin{itemize}
\item transition: N/A
\item output: \texttt{Song\textunderscore Key, Song\textunderscore Scale} : = \texttt{ExtractKeyScale(Audio\textunderscore Time\textunderscore Series)}
\item exception: N/A
\end{itemize}

\subsubsection{Local Functions}
N/A

%Instrument Type Feature
\section{\textcolor{red}{\sout{MIS of Instrument Type Feature Extraction Module}}} 

\subsection{\textcolor{red}{\sout{Instrument Type Feature Extraction Module}}}

\subsection{\textcolor{red}{\sout{Uses}}}
\textcolor{red}{\sout{N/A}}

\subsection{\textcolor{red}{\sout{Syntax}}}

\subsubsection{\textcolor{red}{\sout{Exported Constants}}}
\textcolor{red}{\sout{N/A}}

\subsubsection{\textcolor{red}{\sout{Exported Access Programs}}}

\begin{center}
\begin{tabular}{p{2cm} p{4cm} p{4cm} p{2cm}}
\hline
\textcolor{red}{\sout{\textbf{Name}}} & \textcolor{red}{\sout{\textbf{In}}} & \textcolor{red}{\sout{\textbf{Out}}} & \textcolor{red}{\sout{\textbf{Exceptions}}}\\
\hline
\textcolor{red}{\sout{\texttt{Extract Instrument Type}}} & \textcolor{red}{\sout{\texttt{Audio time series (np.ndarray)}}} & \textcolor{red}{\sout{\texttt{Instrument Type} $\in \mathbb{Z}^k$}} & \textcolor{red}{\sout{-}}\\
\hline
\end{tabular}
\end{center}

\subsection{\textcolor{red}{\sout{Semantics}}}

\subsubsection{\textcolor{red}{\sout{State Variables}}}
\textcolor{red}{\sout{N/A}}

\subsubsection{\textcolor{red}{\sout{Environment Variables}}}
\textcolor{red}{\sout{N/A}}

\subsubsection{\textcolor{red}{\sout{Assumptions}}}
\textcolor{red}{\sout{Valid audio file with coherent song information.}}

\subsubsection{\textcolor{red}{\sout{Access Routine Semantics}}}

\noindent \textcolor{red}{\sout{\texttt{ExtractInstrumentType()}:}}
\begin{itemize}
\item \textcolor{red}{\sout{transition: N/A}}
\item \textcolor{red}{\sout{output: \texttt{Instrument\_Type} := \texttt{ExtractInstrumentType(Audio\_Time\_Series)}}}
\item \textcolor{red}{\sout{exception: N/A}}
\end{itemize}

\subsubsection{\textcolor{red}{\sout{Local Functions}}}
\textcolor{red}{\sout{N/A}}

%MIS Vocal Gender Feature
\section{\textcolor{red}{\sout{MIS of Vocal Gender Feature Extraction Module}}} 

\subsection{\textcolor{red}{\sout{MIS of Vocal Gender Feature Extraction Module}}}
\textcolor{red}{\sout{This feature seeks to quantify whether the voices features in the inputted audio file 
are largely more feminine or masculine sounding. This is represented by a float with a range between
0 and 1 where 0 means only "masculine" sound signatures are contained and 1 means only "feminine" sounds,
where values in-between represent a blend.}}
%\textcolor{red}{\sout{IMO this conceptions is the easiest way to do it. If you did a 0 or 1 integer the representation wouldn't make
%useful sense for songs where there are vocalists of different genders.}}

\subsection{\textcolor{red}{\sout{Uses}}}
\textcolor{red}{\sout{N/A}}

\subsection{\textcolor{red}{\sout{Syntax}}}

\subsubsection{\textcolor{red}{\sout{Exported Constants}}}
\textcolor{red}{\sout{N/A}}

\subsubsection{\textcolor{red}{\sout{Exported Access Programs}}}

\begin{center}
\begin{tabular}{p{2cm} p{4cm} p{4cm} p{2cm}}
\hline
\textcolor{red}{\sout{\textbf{Name}}} & \textcolor{red}{\sout{\textbf{In}}} & \textcolor{red}{\sout{\textbf{Out}}} & \textcolor{red}{\sout{\textbf{Exceptions}}}\\
\hline
\textcolor{red}{\sout{\texttt{Extract Vocal Gender}}} & \textcolor{red}{\sout{\texttt{Audio time series (np.ndarray)}}} & \textcolor{red}{\sout{\texttt{Vocal Gender} $\in \mathbb{R}$}} & \textcolor{red}{\sout{-}}\\
\hline
\end{tabular}
\end{center}

\subsection{\textcolor{red}{\sout{Semantics}}}

\subsubsection{\textcolor{red}{\sout{State Variables}}}
\textcolor{red}{\sout{N/A}}

\subsubsection{\textcolor{red}{\sout{Environment Variables}}}
\textcolor{red}{\sout{N/A}}

\subsubsection{\textcolor{red}{\sout{Assumptions}}}
\textcolor{red}{\sout{Valid audio file with coherent song information.}}

\subsubsection{\textcolor{red}{\sout{Access Routine Semantics}}}

\noindent \textcolor{red}{\sout{\texttt{ExtractVocalGender()}:}}
\begin{itemize}
\item \textcolor{red}{\sout{transition: N/A}}
\item \textcolor{red}{\sout{output: \texttt{Vocal\_Gender} := \texttt{ExtractVocalGender(Audio\_Time\_Series)}}}
\item \textcolor{red}{\sout{exception: N/A}}
\end{itemize}

\subsubsection{\textcolor{red}{\sout{Local Functions}}}
\textcolor{red}{\sout{N/A}}


%Dynamic Range Feature
\section{MIS of Dynamic Range Feature Extraction Module} 

\subsection{Dynamic Range Feature Extraction Module}
Feature extracts the range of sounds (difference between peak and \textcolor{red}{mean}) of the audio signal.

\subsection{Uses}
N/A

\subsection{Syntax}

\subsubsection{Exported Constants}
N/A

\subsubsection{Exported Access Programs}

\begin{center}
\begin{tabular}{p{2cm} p{4cm} p{4cm} p{2cm}}
\hline
\textbf{Name} & \textbf{In} & \textbf{Out} & \textbf{Exceptions}\\
\hline%by results, at this stage, we mean the output of the final program (could include features, includes recs)
\texttt{Extract Dynamic Range} &\texttt{Audio time series (np.ndarray)} &\texttt{Dynamic Range (decibels)} $\in \mathbb{R}$ &-\\
\hline
\end{tabular}
\end{center}

\subsection{Semantics}

\subsubsection{State Variables}
N/A

\subsubsection{Environment Variables}
N/A

\subsubsection{Assumptions}
Valid audio file with coherent song information.

\subsubsection{Access Routine Semantics}

\noindent \texttt{ExtractDynamicRange()}:
\begin{itemize}
\item transition: N/A
\item output: \texttt{Dynamic\textunderscore Range} : = \texttt{ExtractDynamicRange(Audio\textunderscore Time\textunderscore Series)}
\item exception: N/A
\end{itemize}

\subsubsection{Local Functions}
N/A

% % RMS feature extraction
\section{\textcolor{red}{MIS of RMS Feature Extraction Module}} 

\subsection{\textcolor{red}{Dynamic Range Feature Extraction Module}}
\textcolor{red}{Extracts the mean amplitude (root mean square energy) of the input track.}

\subsection{\textcolor{red}{Uses}}
\textcolor{red}{N/A}

\subsection{\textcolor{red}{Syntax}}

\subsubsection{\textcolor{red}{Exported Constants}}
\textcolor{red}{N/A}

\subsubsection{\textcolor{red}{Exported Access Programs}}

\begin{center}
\begin{tabular}{p{2cm} p{4cm} p{4cm} p{2cm}}
\hline
\textcolor{red}{\textbf{Name}} & \textcolor{red}{\textbf{In}} & \textcolor{red}{\textbf{Out}} & \textcolor{red}{\textbf{Exceptions}}\\
\hline
\textcolor{red}{\texttt{Extract RMS}} & \textcolor{red}{\texttt{Audio time series (np.ndarray)}} & \textcolor{red}{\texttt{RMS (decibels)} $\in \mathbb{R}$} & \textcolor{red}{-}\\
\hline
\end{tabular}
\end{center}

\subsection{\textcolor{red}{Semantics}}

\subsubsection{\textcolor{red}{State Variables}}
\textcolor{red}{N/A}

\subsubsection{\textcolor{red}{Environment Variables}}
\textcolor{red}{N/A}

\subsubsection{\textcolor{red}{Assumptions}}
\textcolor{red}{Valid audio file with coherent song information.}

\subsubsection{\textcolor{red}{Access Routine Semantics}}

\noindent \textcolor{red}{\texttt{ExtractDynamicRange()}:}
\begin{itemize}
\item \textcolor{red}{transition: N/A}
\item \textcolor{red}{output: \texttt{RMS} := \texttt{ExtractRMS(Audio\_Time\_Series)}}
\item \textcolor{red}{exception: N/A}
\end{itemize}

\subsubsection{\textcolor{red}{Local Functions}}
\textcolor{red}{N/A}


%Instrumentalness
\section{MIS of Instrumentalness Feature Extraction Module} 

\subsection{Instrumentalness Feature Extraction Module}
Extracts the how prominent instrumental sounds are within the song. Represented by a float
variable where the range is between 0 and 1, where higher values mean more instrumental sounds
and lower means less. Eg, 0 would mean an acapella piece of music, 1 would be something that purely
features instruments.  

\subsection{Uses}
N/A

\subsection{Syntax}

\subsubsection{Exported Constants}
N/A

\subsubsection{Exported Access Programs}

\begin{center}
\begin{tabular}{p{4cm} p{4cm} p{4cm} p{2cm}}
\hline
\textbf{Name} & \textbf{In} & \textbf{Out} & \textbf{Exceptions}\\
\hline%by results, at this stage, we mean the output of the final program (could include features, includes recs)
\texttt{Extract\linebreak Instrumentalness} &\texttt{\textcolor{red}{vocal audio time series}, \linebreak (np.ndarray) \linebreak \textcolor{red}{non-vocal audio time serie \linebreak (np.ndarray)}} &\texttt{Instrumentalness} $\in \mathbb{R}$ &-\\
\hline
\end{tabular}
\end{center}

\subsection{Semantics}

\subsubsection{State Variables}
N/A

\subsubsection{Environment Variables}
N/A

\subsubsection{Assumptions}
Valid audio file with coherent song information.

\subsubsection{Access Routine Semantics}

\noindent \texttt{ExtractInstrumentalness}():
\begin{itemize}
\item transition: N/A
\item output: \texttt{Instrumentalness} : = \texttt{ExtractInstrumentalness(Audio\textunderscore Time\textunderscore Series)}
\item exception: N/A
\end{itemize}

\subsubsection{Local Functions}
N/A


{\color{red}
\section{MIS of Spectral Centroid Feature Module}

\subsection{Spectral Centroid Feature Module}

\subsection{Uses}
N/A

\subsection{Syntax}

\subsubsection{Exported Constants}
N/A

\subsubsection{Exported Access Programs}

\begin{center}
  \begin{tabular}{|p{6cm}|p{4cm}|p{3cm}|p{2cm}|}
  \hline
  \textbf{Name} & \textbf{In} & \textbf{Out} & \textbf{Exceptions} \\
  \hline
  \texttt{ComputeSpectralCentroid} & \texttt{STFT Magnitude Array (np.ndarray, 2D)} & \texttt{Spectral Centroid Vector (np.ndarray)} & - \\
  \hline
  \texttt{ComputeSpectralCentroidsMean} & \texttt{Divided STFT Magnitude Array (np.ndarray, 3D)} & \texttt{(Spectral Centroid Matrix (np.ndarray), Total Mean Spectral Centroid (float), Mean Spectral Centroids (np.ndarray))} & - \\
  \hline
  \end{tabular}
\end{center}
  

\subsection{Semantics}

\subsubsection{State Variables}
N/A

\subsubsection{Environment Variables}
N/A

\subsubsection{Assumptions}
\begin{itemize}
    \item The input signal is processed such that its Short-Time Fourier Transform (STFT) magnitude is correctly computed.
    \item The frequency bins are precomputed using the sampling rate.
\end{itemize}

\subsubsection{Access Routine Semantics}

\noindent \texttt{ComputeSpectralCentroid()}:
\begin{itemize}
    \item \textbf{transition:} N/A
    \item \textbf{output:} \texttt{Spectral Centroid Vector} = For a given STFT magnitude array, each element is computed as 
    \[
    \text{Spectral Centroid} = \frac{\sum_{n} f(n) \cdot x(n)}{\sum_{n} x(n)},
    \]
    where \(f(n)\) are the precomputed frequency bins and \(x(n)\) are the STFT magnitude values.
    \item \textbf{exception:} N/A
\end{itemize}

\noindent \texttt{ComputeSpectralCentroidsMean()}:
\begin{itemize}
    \item \textbf{transition:} N/A
    \item \textbf{output:} 
    \begin{itemize}
        \item \texttt{Spectral Centroid Matrix}: A 2D array where each row corresponds to the spectral centroid values for each sampling frame of a window.
        \item \texttt{Total Mean Spectral Centroid}: A single float representing the average of the mean spectral centroids across all windows.
        \item \texttt{Mean Spectral Centroids}: A 2D array (or column vector) containing the mean spectral centroid for each window.
    \end{itemize}
    \item \textbf{exception:} N/A
\end{itemize}

\subsubsection{Local Functions}
N/A
}


{\color{red}
\section{MIS of Spectral Bandwidth Feature Module}

\subsection{Spectral Bandwidth Feature Module}

\subsection{Uses}
N/A

\subsection{Syntax}

\subsubsection{Exported Constants}
N/A

\subsubsection{Exported Access Programs}

\begin{center}
  \begin{tabular}{|p{6cm}|p{4cm}|p{3cm}|p{2cm}|}
  \hline
  \textbf{Name} & \textbf{In} & \textbf{Out} & \textbf{Exceptions} \\
  \hline
  \texttt{ComputeSpectralBandwidth} & \texttt{STFT Magnitude Array (np.ndarray, 2D), Centroid (np.ndarray)} & \texttt{Bandwidth (np.ndarray)} & - \\
  \hline
  \texttt{ComputeSpectralBandwidthMean} & \texttt{Divided STFT Magnitude Array (np.ndarray, 3D), Spectral Centroids (np.ndarray)} & \texttt{(Spectral Bandwidths (np.ndarray), Total Mean Spectral Bandwidth (float), Mean Spectral Bandwidths (np.ndarray))} & - \\
  \hline
  \end{tabular}
\end{center}

\subsection{Semantics}

\subsubsection{State Variables}
N/A

\subsubsection{Environment Variables}
N/A

\subsubsection{Assumptions}
\begin{itemize}
    \item The input signal's STFT magnitude is computed correctly.
    \item Frequency bins are precomputed using the given sampling rate.
    \item The spectral centroid values are available for spectral bandwidth computation.
\end{itemize}

\subsubsection{Access Routine Semantics}

\noindent \texttt{ComputeSpectralBandwidth()}:
\begin{itemize}
    \item \textbf{transition:} N/A
    \item \textbf{output:} \texttt{Bandwidth (np.ndarray)} computed for each sampling frame using the formula:
    \[
    \text{Bandwidth} = \sqrt{\text{stft\_magnitude} \times \left( \text{frequency} - \text{centroid} \right)^2},
    \]
    where the frequency bins are precomputed.
    \item \textbf{exception:} N/A
\end{itemize}

\noindent \texttt{ComputeSpectralBandwidthMean()}:
\begin{itemize}
    \item \textbf{transition:} N/A
    \item \textbf{output:} 
    \begin{itemize}
        \item \texttt{Spectral Bandwidths (np.ndarray)}: A 3D array of spectral bandwidth values for each window and frame.
        \item \texttt{Total Mean Spectral Bandwidth (float)}: The average spectral bandwidth over all windows.
        \item \texttt{Mean Spectral Bandwidths (np.ndarray)}: A 2D array (or column vector) containing the mean spectral bandwidth per window.
    \end{itemize}
    \item \textbf{exception:} N/A
\end{itemize}

\subsubsection{Local Functions}
N/A
}

{\color{red}
\section{MIS of Spectral Rolloff Feature Module}

\subsection{Spectral Rolloff Feature Module}

\subsection{Uses}
N/A

\subsection{Syntax}

\subsubsection{Exported Constants}
N/A

\subsubsection{Exported Access Programs}

\begin{center}
  \begin{tabular}{|p{7cm}|p{3cm}|p{3cm}|p{2cm}|}
  \hline
  \textbf{Name} & \textbf{In} & \textbf{Out} & \textbf{Exceptions} \\
  \hline
  \texttt{ComputeSpectralRolloffFrequency} & \texttt{STFT Magnitude Array (np.ndarray, 2D)} & \texttt{(Upper Rolloff (np.ndarray), Lower Rolloff (np.ndarray))} & - \\
  \hline
  \texttt{ComputeFrequencyRange} & \texttt{Divided STFT Magnitude Array (np.ndarray, 3D)} & \texttt{(Frequency Ranges (np.ndarray), Mean Frequency Range (float))} & - \\
  \hline
  \end{tabular}
\end{center}

\subsection{Semantics}

\subsubsection{State Variables}
N/A

\subsubsection{Environment Variables}
N/A

\subsubsection{Assumptions}
\begin{itemize}
    \item The input signal's STFT magnitude is correctly computed.
    \item Frequency bins are precomputed using the provided sampling rate.
    \item The percentile value for roll-off calculation is set and valid.
\end{itemize}

\subsubsection{Access Routine Semantics}

\noindent \texttt{ComputeSpectralRolloffFrequency()}:
\begin{itemize}
    \item \textbf{transition:} N/A
    \item \textbf{output:} \texttt{(Upper Rolloff, Lower Rolloff)} where:
    \begin{itemize}
        \item \texttt{Upper Rolloff (np.ndarray)}: Upper spectral roll-off frequencies at each sampling frame, computed by identifying the first frequency bin where the cumulative energy meets or exceeds the upper threshold.
        \item \texttt{Lower Rolloff (np.ndarray)}: Lower spectral roll-off frequencies at each sampling frame, computed by identifying the first frequency bin where the cumulative energy falls below the lower threshold.
    \end{itemize}
    The computation uses two-sided percentile thresholds on the cumulative energy of the STFT magnitudes.
    \item \textbf{exception:} N/A
\end{itemize}

\noindent \texttt{ComputeFrequencyRange()}:
\begin{itemize}
    \item \textbf{transition:} N/A
    \item \textbf{output:}
    \begin{itemize}
        \item \texttt{Frequency Ranges (np.ndarray)}: An array representing the frequency range (upper minus lower roll-off) for each window.
        \item \texttt{Mean Frequency Range (float)}: The average frequency range computed across all windows.
    \end{itemize}
    \item \textbf{exception:} N/A
\end{itemize}

\subsubsection{Local Functions}
N/A
}

{\color{red}
\section{MIS of Spectral Flux Feature Module}

\subsection{Spectral Flux Feature Module}

\subsection{Uses}
N/A

\subsection{Syntax}

\subsubsection{Exported Constants}
N/A

\subsubsection{Exported Access Programs}

\begin{center}
  \begin{tabular}{|p{6cm}|p{4cm}|p{3cm}|p{2cm}|}
  \hline
  \textbf{Name} & \textbf{In} & \textbf{Out} & \textbf{Exceptions} \\
  \hline
  \texttt{ComputeSpectralFlux} & \texttt{STFT Magnitude Array (np.ndarray, 2D)} & \texttt{Spectral Flux (np.ndarray)} & - \\
  \hline
  \texttt{ComputeSpectralFluxMean} & \texttt{Divided STFT Magnitude Array (np.ndarray, 3D)} & \texttt{(Spectral Flux (np.ndarray), Total Mean Spectral Flux (float), Mean Spectral Flux (np.ndarray))} & - \\
  \hline
  \end{tabular}
\end{center}

\subsection{Semantics}

\subsubsection{State Variables}
N/A

\subsubsection{Environment Variables}
N/A

\subsubsection{Assumptions}
\begin{itemize}
    \item The input STFT magnitude array is correctly computed.
    \item For the mean computation, the divided STFT magnitude array is properly segmented into windows.
\end{itemize}

\subsubsection{Access Routine Semantics}

\noindent \texttt{ComputeSpectralFlux()}:
\begin{itemize}
    \item \textbf{transition:} N/A
    \item \textbf{output:} \texttt{Spectral Flux (np.ndarray)} computed as follows: The STFT magnitudes are first scaled using a logarithmic function, then the flux is determined by summing the positive differences between consecutive time frames.
    \item \textbf{exception:} N/A
\end{itemize}

\noindent \texttt{ComputeSpectralFluxMean()}:
\begin{itemize}
    \item \textbf{transition:} N/A
    \item \textbf{output:} 
    \begin{itemize}
        \item \texttt{Spectral Flux (np.ndarray)}: A 2D array where each row contains the spectral flux values computed for each window.
        \item \texttt{Total Mean Spectral Flux (float)}: The average spectral flux computed across all windows.
        \item \texttt{Mean Spectral Flux (np.ndarray)}: A 2D array (or column vector) where each entry represents the mean spectral flux for a given window.
    \end{itemize}
    \item \textbf{exception:} N/A
\end{itemize}

\subsubsection{Local Functions}
N/A
}


{\color{red}
\section{MIS of Spectral Contrast Feature Module}

\subsection{Spectral Contrast Feature Module}

\subsection{Uses}
N/A

\subsection{Syntax}

\subsubsection{Exported Constants}
N/A

\subsubsection{Exported Access Programs}

\begin{center}
  \begin{tabular}{|p{6cm}|p{4cm}|p{3cm}|p{2cm}|}
  \hline
  \textbf{Name} & \textbf{In} & \textbf{Out} & \textbf{Exceptions} \\
  \hline
  \texttt{ComputeSpectralContrast} & \texttt{Subband Magnitudes (np.ndarray, 2D)} & \texttt{Spectral Contrast (np.ndarray)} & - \\
  \hline
  \texttt{ComputeSpectralContrastMean} & \texttt{Divided STFT Magnitude Array (np.ndarray, 3D)} & \texttt{(Spectral Contrast (np.ndarray), Total Mean Spectral Contrast (float), Mean Spectral Contrast (np.ndarray))} & - \\
  \hline
  \end{tabular}
\end{center}

\subsection{Semantics}

\subsubsection{State Variables}
N/A

\subsubsection{Environment Variables}
N/A

\subsubsection{Assumptions}
\begin{itemize}
    \item The input STFT magnitude arrays are correctly computed.
    \item The frequency range is divided into the specified number of bands.
    \item The parameter \(\alpha\) (fraction of peak/valley magnitudes) is set and valid.
\end{itemize}

\subsubsection{Access Routine Semantics}

\noindent \texttt{ComputeSpectralContrast()}:
\begin{itemize}
    \item \textbf{transition:} N/A
    \item \textbf{output:} \texttt{Spectral Contrast (np.ndarray)} computed as the difference between the subband peak and valley values (both converted to dB scale) for each frame.
    \item \textbf{exception:} N/A
\end{itemize}

\noindent \texttt{ComputeSpectralContrastMean()}:
\begin{itemize}
    \item \textbf{transition:} N/A
    \item \textbf{output:}
    \begin{itemize}
        \item \texttt{Spectral Contrast (np.ndarray)}: A 3D array containing spectral contrast values for each window and frequency band.
        \item \texttt{Total Mean Spectral Contrast (float)}: The average spectral contrast computed over all windows and bands.
        \item \texttt{Mean Spectral Contrast (np.ndarray)}: A 2D array (or column vector) containing the mean spectral contrast for each window and band.
    \end{itemize}
    \item \textbf{exception:} N/A
\end{itemize}

\subsubsection{Local Functions}
\begin{itemize}
    \item \texttt{ComputeSubbandPeak}: Computes the subband peak by averaging the top \(\alpha\)-percent of magnitudes, then converting to dB scale.
    \item \texttt{ComputeSubbandValley}: Computes the subband valley by averaging the lowest \(\alpha\)-percent of magnitudes, then converting to dB scale.
\end{itemize}
}



%Contour Extraction
\section{\textcolor{red}{\sout{MIS of Contour Feature Extraction Module}}} 

\subsection{\textcolor{red}{\sout{Contour Feature Extraction Module}}}

\subsection{\textcolor{red}{\sout{Uses}}}
\textcolor{red}{\sout{N/A}}

\subsection{\textcolor{red}{\sout{Syntax}}}

\subsubsection{\textcolor{red}{\sout{Exported Constants}}}
\textcolor{red}{\sout{N/A}}

\subsubsection{\textcolor{red}{\sout{Exported Access Programs}}}

\begin{center}
\begin{tabular}{p{2cm} p{4cm} p{4cm} p{2cm}}
\hline
\textcolor{red}{\sout{\textbf{Name}}} & \textcolor{red}{\sout{\textbf{In}}} & \textcolor{red}{\sout{\textbf{Out}}} & \textcolor{red}{\sout{\textbf{Exceptions}}}\\
\hline
\textcolor{red}{\sout{\texttt{Extract Melodic Contour}}} & \textcolor{red}{\sout{\texttt{Audio time series (np.ndarray)}}} & \textcolor{red}{\sout{\texttt{Contour}}} & \textcolor{red}{\sout{-}}\\
\hline
\end{tabular}
\end{center}

\subsection{\textcolor{red}{\sout{Semantics}}}

\subsubsection{\textcolor{red}{\sout{State Variables}}}
\textcolor{red}{\sout{N/A}}

\subsubsection{\textcolor{red}{\sout{Environment Variables}}}
\textcolor{red}{\sout{N/A}}

\subsubsection{\textcolor{red}{\sout{Assumptions}}}
\textcolor{red}{\sout{Valid audio file with coherent song information.}}

\subsubsection{\textcolor{red}{\sout{Access Routine Semantics}}}

\noindent \textcolor{red}{\sout{\texttt{ExtractMelodicContour()}:}}
\begin{itemize}
\item \textcolor{red}{\sout{transition: N/A}}
\item \textcolor{red}{\sout{output: \texttt{Contour} := \texttt{ExtractMelodicContour(Audio\_Time\_Series)}}}
\item \textcolor{red}{\sout{exception: N/A}}
\end{itemize}

\subsubsection{\textcolor{red}{\sout{Local Functions}}}
\textcolor{red}{\sout{N/A}}

\section{\textcolor{red}{\sout{MIS of Mood Feature Extraction Module}}} 

\subsection{\textcolor{red}{\sout{Mood Feature Extraction Module}}}

\subsection{\textcolor{red}{\sout{Uses}}}
\textcolor{red}{\sout{N/A}}

\subsection{\textcolor{red}{\sout{Syntax}}}

\subsubsection{\textcolor{red}{\sout{Exported Constants}}}
\textcolor{red}{\sout{N/A}}

\subsubsection{\textcolor{red}{\sout{Exported Access Programs}}}

\begin{center}
\begin{tabular}{p{2cm} p{4cm} p{4cm} p{2cm}}
\hline
\textcolor{red}{\sout{\textbf{Name}}} & \textcolor{red}{\sout{\textbf{In}}} & \textcolor{red}{\sout{\textbf{Out}}} & \textcolor{red}{\sout{\textbf{Exceptions}}}\\
\hline
\textcolor{red}{\sout{\texttt{Extract Mood}}} & \textcolor{red}{\sout{\texttt{Audio time series (np.ndarray)}}} & \textcolor{red}{\sout{\texttt{Mood} $\in{\mathbb{Z}}$}} & \textcolor{red}{\sout{-}}\\
\hline
\end{tabular}
\end{center}

\subsection{\textcolor{red}{\sout{Semantics}}}

\subsubsection{\textcolor{red}{\sout{State Variables}}}
\textcolor{red}{\sout{N/A}}

\subsubsection{\textcolor{red}{\sout{Environment Variables}}}
\textcolor{red}{\sout{N/A}}

\subsubsection{\textcolor{red}{\sout{Assumptions}}}
\textcolor{red}{\sout{Valid audio file with coherent song information.}}

\subsubsection{\textcolor{red}{\sout{Access Routine Semantics}}}

\noindent \textcolor{red}{\sout{\texttt{ExtractMood()}:}}
\begin{itemize}
\item \textcolor{red}{\sout{transition: N/A}}
\item \textcolor{red}{\sout{output: \texttt{Mood} := \texttt{ExtractMood(Audio\_Time\_Series)}}}
\item \textcolor{red}{\sout{exception: N/A}}
\end{itemize}

\subsubsection{\textcolor{red}{\sout{Local Functions}}}
\textcolor{red}{\sout{N/A}}

% MIS of Genre Feature Module

\section{\textcolor{red}{\sout{MIS of Genre Feature Extraction Module}}} %\label{\textcolor{red}{\sout{Module:GenreFeatureExtraction}}}

\subsection{\textcolor{red}{\sout{Module}}}
\textcolor{red}{\sout{Genre Feature Extraction Module}}

\subsection{\textcolor{red}{\sout{Uses}}}
\begin{itemize}
\item \textcolor{red}{\sout{Featurizer Module: Receives metadata from the Featurizer Module and extracts the genre attribute from it.}}
\item \textcolor{red}{\sout{Metadata Structure: Utilizes the metadata structure to locate and retrieve the genre attribute.}}
\end{itemize}

\subsection{\textcolor{red}{\sout{Syntax}}}

\subsubsection{\textcolor{red}{\sout{Exported Constants}}}
\textcolor{red}{\sout{None.}}

\subsubsection{\textcolor{red}{\sout{Exported Access Programs}}}

\begin{center}
\begin{tabular}{p{2cm} p{4cm} p{4cm} p{2cm}}
\hline
\textcolor{red}{\sout{\textbf{Name}}} & \textcolor{red}{\sout{\textbf{In}}} & \textcolor{red}{\sout{\textbf{Out}}} & \textcolor{red}{\sout{\textbf{Exceptions}}} \\
\hline
\textcolor{red}{\sout{extractGenre}} & \textcolor{red}{\sout{metadata: Metadata}} & \textcolor{red}{\sout{genre: String}} & \textcolor{red}{\sout{MissingGenreException, InvalidMetadataException}} \\
\hline
\end{tabular}
\end{center}

\subsection{\textcolor{red}{\sout{Semantics}}}

\subsubsection{\textcolor{red}{\sout{State Variables}}}
\textcolor{red}{\sout{- \texttt{metadata}: The metadata provided by the Featurizer Module, which contains the genre attribute.}}

\subsubsection{\textcolor{red}{\sout{Environment Variables}}}
\textcolor{red}{\sout{None.}}

\subsubsection{\textcolor{red}{\sout{Assumptions}}}
\begin{itemize}
\item \textcolor{red}{\sout{The metadata provided by the Featurizer Module is valid and includes the genre attribute.}}
\item \textcolor{red}{\sout{The genre attribute in the metadata is correctly formatted and accessible.}}
\end{itemize}

\subsubsection{\textcolor{red}{\sout{Access Routine Semantics}}}

\noindent \textcolor{red}{\sout{\textbf{extractGenre}(metadata: Metadata):}}
\begin{itemize}
\item \textcolor{red}{\sout{\textbf{Transition:}}}
    \textcolor{red}{\sout{- Extracts the genre attribute from the provided metadata.}}
\item \textcolor{red}{\sout{\textbf{Output:}}}
    \textcolor{red}{\sout{- Returns the extracted genre as a string.}}
\item \textcolor{red}{\sout{\textbf{Exceptions:}}}
    \begin{itemize}
    \item \textcolor{red}{\sout{\texttt{MissingGenreException}: Raised if the genre attribute is not found in the metadata.}}
    \item \textcolor{red}{\sout{\texttt{InvalidMetadataException}: Raised if the provided metadata is improperly formatted or invalid.}}
    \end{itemize}
\end{itemize}

\subsubsection{\textcolor{red}{\sout{Local Functions}}}

\textcolor{red}{\sout{\textbf{validateMetadata}:}}
\begin{itemize}
\item \textcolor{red}{\sout{Purpose: Ensures the provided metadata is valid and contains the necessary attributes.}}
\item \textcolor{red}{\sout{Input: \texttt{metadata}.}}
\item \textcolor{red}{\sout{Output: Boolean (true if valid, false otherwise).}}
\end{itemize}

\textcolor{red}{\sout{\textbf{retrieveGenre}:}}
\begin{itemize}
\item \textcolor{red}{\sout{Purpose: Locates and retrieves the genre attribute from the metadata.}}
\item \textcolor{red}{\sout{Input: \texttt{metadata}.}}
\item \textcolor{red}{\sout{Output: \texttt{genre} (String).}}
\end{itemize}

%Recommendation module
{\color{red}
\section{MIS of Recommendation System Module}

\subsection{Recommendation System Module}

\subsection{Uses}
Database Module
\subsection{Syntax}

\subsubsection{Exported Constants}
N/A

\subsubsection{Exported Access Programs}

\begin{center}
  \begin{tabular}{|p{6cm}|p{4cm}|p{3cm}|p{2cm}|}
  \hline
  \textbf{Name} & \textbf{In} & \textbf{Out} & \textbf{Exceptions} \\
  \hline
  \texttt{Recommendation (Constructor)} & \texttt{data (pd.DataFrame) [optional], file\_path (str) [optional]} & \texttt{Recommendation Object} & \texttt{ValueError if neither provided} \\
  \hline
  \texttt{get\_similar\_songs} & \texttt{reference\_track (str), top\_n (int, default=10)} & \texttt{Similar Songs (pd.Series) or error message (str)} & - \\
  \hline
  \end{tabular}
\end{center}

\subsection{Semantics}

\subsubsection{State Variables}
\begin{itemize}
    \item \texttt{df}: Original dataset containing song features.
    \item \texttt{df\_encoded}: Dataset with categorical features (\texttt{major} and \texttt{minor}) encoded numerically.
    \item \texttt{feature\_cols}: List of feature columns used for similarity computation.
    \item \texttt{features}: DataFrame containing the selected numerical features.
    \item \texttt{scaler}: StandardScaler used to normalize features.
    \item \texttt{features\_scaled}: Normalized feature matrix.
    \item \texttt{similarity\_matrix}: Cosine similarity matrix computed from the normalized features.
    \item \texttt{similarity\_df}: DataFrame version of the similarity matrix with track names as both index and columns.
\end{itemize}

\subsubsection{Environment Variables}
N/A

\subsubsection{Assumptions}
\begin{itemize}
    \item The input dataset contains a \texttt{track\_name} column along with other feature columns including \texttt{major} and \texttt{minor}.
    \item Categorical features (\texttt{major} and \texttt{minor}) are properly encoded into numerical values.
    \item The CSV file (if used) is properly formatted and accessible.
\end{itemize}

\subsubsection{Access Routine Semantics}

\noindent \texttt{Recommendation (Constructor)}:
\begin{itemize}
    \item \textbf{transition:} Loads the dataset from a provided DataFrame or CSV file, encodes categorical features, selects numerical feature columns, normalizes the data using a StandardScaler, computes the cosine similarity matrix, and creates a similarity DataFrame with track names.
    \item \textbf{output:} A \texttt{Recommendation} object with all necessary data structures initialized.
    \item \textbf{exception:} Raises a \texttt{ValueError} if neither \texttt{data} nor \texttt{file\_path} is provided.
\end{itemize}

\noindent \texttt{get\_similar\_songs}:
\begin{itemize}
    \item \textbf{transition:} Retrieves the top N most similar songs by sorting the similarity scores for a given reference track.
    \item \textbf{output:} A pandas Series containing the names and similarity scores of the top N similar songs (excluding the reference track). If the reference track is not found in the dataset, returns an error message.
    \item \textbf{exception:} N/A
\end{itemize}

\subsubsection{Local Functions}
N/A
}
% \begin{itemize}
%   \item \texttt{Generate_Embeds()}
% \end{itemize}

%Results Display Interface
\section{MIS of Program Results Interface Module} 

\subsection{Program Results Interface Module}
%should just be used to display the generated recommendations. need to decide how that is represented, is it links? do we show 
%the features? etc. 

\subsection{Uses}
\begin{itemize}
  \item Spotify API
\end{itemize}

\subsection{Syntax}

\subsubsection{Exported Constants}
N/A

\subsubsection{Exported Access Programs}

\begin{center}
\begin{tabular}{p{2cm} p{4cm} p{4cm} p{2cm}}
\hline
\textbf{Name} & \textbf{In} & \textbf{Out} & \textbf{Exceptions}\\
\hline
\texttt{Generate Spotify Embed} &\texttt{Rec\textunderscore Track \linebreak (np.ndarray $\in$ Track)} &\texttt{Tracks\textunderscore Embed} (Spotify Embed Element) &-\\
\texttt{Display Features} &\texttt{Song Features\linebreak (np.ndarray $\in$ Feature)} &\texttt{Features\textunderscore Display} (UI Image) &-\\
\hline
\end{tabular}
\end{center}

\subsection{Semantics}

\subsubsection{State Variables}
N/A

\subsubsection{Environment Variables}
N/A

\subsubsection{Assumptions}
N/A

\subsubsection{Access Routine Semantics}

\noindent \texttt{GenerateSpotifyEmbed()}:
\begin{itemize}
\item transition: N/A
\item output: \texttt{Tracks\textunderscore Embed\textunderscore Widget}: = \texttt{GenerateSpotifyEmbed(Tracks)}
\item exception: N/A
\end{itemize}

\noindent \texttt{DisplayFeatures()}:
\begin{itemize}
\item transition: N/A
\item output: \texttt{Features\textunderscore Display}: = \texttt{DisplayFeatures(Song\textunderscore Features)}
\item exception: N/A
\end{itemize}

\subsubsection{Local Functions}
N/A

\newpage

\section{MIS of Database Module} 

\subsection{Database Module}

\subsection{Uses}
N/A

\subsection{Syntax}

\subsubsection{Exported Constants}
N/A

\subsubsection{Exported Access Programs}

\begin{center}
\begin{tabular}{llll}
\hline
\textbf{Name} & \textbf{In} & \textbf{Out} & \textbf{Exceptions}\\
\hline
\texttt{fetch\_sptf\_song\_info} & \texttt{track\_ref} & \texttt{song\_feats} & \texttt{SongNotFoundError} \\
\texttt{fetch\_file\_song\_info} & \texttt{track\_ref} & \texttt{song\_feats} & \texttt{SongNotFoundError} \\
\texttt{deposit\_song\_info} & \texttt{track\_ref}, \texttt{song\_feats} & --- & \texttt{SongAlreadyExistsError}, \\
& & & \texttt{InvalidFeatError} \\
\texttt{update\_song\_info} & \texttt{track\_ref}, \texttt{song\_feats} & --- & \texttt{SongNotFoundError}, \\
& & & \texttt{InvalidFeatError} \\
\hline
\end{tabular}
\end{center}

\subsection{Semantics}

\subsubsection{State Variables}
N/A

\subsubsection{Environment Variables}
N/A

\subsubsection{Assumptions}
N/A

\subsubsection{Access Routine Semantics}

\noindent \texttt{fetch\_sptf\_song\_info}(\texttt{track\_ref}):
\begin{itemize}
\item output: \texttt{song\_feats} := a tuple of song features, including genre.
\item exception: \texttt{SongNotFoundError} if song does not exist in the database.
\end{itemize}

\noindent \texttt{fetch\_file\_song\_info}(\texttt{track\_ref}):
\begin{itemize}
\item output: \texttt{song\_feats} := a tuple of song features, excluding (currently only) genre.
\item exception: \texttt{SongNotFoundError} if song does not exist in the database.
\end{itemize}

\noindent \texttt{deposit\_song\_info}(\texttt{track\_ref}, \texttt{song\_feats}):
\begin{itemize}
\item transition: update database to include \emph{new} song and its corresponding tuple of features.
\item exception: \texttt{SongAlreadyExistsError} if song already exists in the database, so its information must be \emph{updated} not \emph{deposited}. \\ \texttt{InvalidFeatError} if input features are incompatible with constraints set in the database.
\end{itemize}

\noindent \texttt{update\_song\_info}(\texttt{track\_ref}, \texttt{song\_feats}):
\begin{itemize}
\item transition: update database to change and \emph{existing} song's tuple of features.
\item exception: \texttt{SongNotFoundError} if song does not exist in the database. \\ \texttt{InvalidFeatError} if input features are incompatible with constraints set in the database.
\end{itemize}

\subsubsection{Local Functions}
N/A

\newpage

\section{MIS of Spotify API Module} 

\subsection{Spotify API Module}

\subsection{Uses}
N/A

\subsection{Syntax}

\subsubsection{Exported Constants}
\texttt{spotify\_conn}

\subsubsection{Exported Access Programs}

\begin{center}
\begin{tabular}{llll}
\hline
\textbf{Name} & \textbf{In} & \textbf{Out} & \textbf{Exceptions}\\
\hline
\texttt{get\_conn} & --- & \texttt{spotify\_conn} & \texttt{InvalidCredentialsError} \\
\hline
\end{tabular}
\end{center}

\subsection{Semantics}

\subsubsection{State Variables}
N/A

\subsubsection{Environment Variables}
Spotify credentials: \texttt{client\_id} and \texttt{client\_secret}.

\subsubsection{Assumptions}
N/A

\subsubsection{Access Routine Semantics}

\noindent \texttt{get\_conn}():
\begin{itemize}
  \item transition: instantiate a spotify connection if none already exists, otherwise return the existing connection. This follows the singleton design pattern.
  \item output: \texttt{spotify\_conn} := a spotify connection object.
\item exception: \texttt{InvalidCredentialsError} if Spotify credentials are not authenticated.
\end{itemize}

\subsubsection{Local Functions}
N/A

%MIS of spleeter audio separator module
\section{\textcolor{red}{MIS of Spleeter Audio Separator Module}} 

\subsection{\textcolor{red}{Spleeter Audio Separator Module}}

\subsection{\textcolor{red}{Uses}}
\textcolor{red}{N/A}

\subsection{\textcolor{red}{Syntax}}

\subsubsection{\textcolor{red}{Exported Constants}}

\subsubsection{\textcolor{red}{Exported Access Programs}}

\begin{center}
\begin{tabular}{llll}
\hline
\textcolor{red}{\textbf{Name}} & \textcolor{red}{\textbf{In}} & \textcolor{red}{\textbf{Out}} & \textcolor{red}{\textbf{Exceptions}}\\
\hline
\textcolor{red}{\texttt{split\_audio}} & \textcolor{red}{\texttt{Audio\_Time\_Series }} & \textcolor{red}{\texttt{Vocal\_Signal (np.ndarray)}} & \textcolor{red}{---} \\
& \textcolor{red}{\texttt{(np.ndarray)}} & \textcolor{red}{\texttt{Non\_Vocal\_Signal (np.ndarray)}} & \\
\hline
\end{tabular}
\end{center}

\subsection{\textcolor{red}{Semantics}}

\subsubsection{\textcolor{red}{State Variables}}
\textcolor{red}{N/A}

\subsubsection{\textcolor{red}{Environment Variables}}
\textcolor{red}{N/A}

\subsubsection{\textcolor{red}{Assumptions}}
\textcolor{red}{N/A}

\subsubsection{\textcolor{red}{Access Routine Semantics}}

\noindent \textcolor{red}{\texttt{get\_conn}():}
\begin{itemize}
  \item \textcolor{red}{transition: Spins up a tensorflow object in order to split the vocal elements from the non-vocal elements of the track.}
  \item \textcolor{red}{output: \texttt{Vocal\_Signal} := isolated vocals audio time series of the original track,\\ \texttt{Non\_Vocal\_Signal} := isolated non-vocals audio time series of the original track.}
  \item \textcolor{red}{exception: N/A}
\end{itemize}

\subsubsection{\textcolor{red}{Local Functions}}
\textcolor{red}{N/A}


\newpage

\bibliographystyle {plainnat}
\bibliography {../../../refs/References}

\newpage

\section{Appendix} \label{Appendix}

\wss{Extra information if required}

\newpage{}

\section*{Appendix --- Reflection}

\wss{Not required for CAS 741 projects}

The information in this section will be used to evaluate the team members on the
graduate attribute of Problem Analysis and Design.

The purpose of reflection questions is to give you a chance to assess your own
learning and that of your group as a whole, and to find ways to improve in the
future. Reflection is an important part of the learning process.  Reflection is
also an essential component of a successful software development process.  

Reflections are most interesting and useful when they're honest, even if the
stories they tell are imperfect. You will be marked based on your depth of
thought and analysis, and not based on the content of the reflections
themselves. Thus, for full marks we encourage you to answer openly and honestly
and to avoid simply writing ``what you think the evaluator wants to hear.''

Please answer the following questions.  Some questions can be answered on the
team level, but where appropriate, each team member should write their own
response:


\begin{enumerate}
  \item What went well while writing this deliverable?
  Writing this deliverable allowed us to develop a comprehensive understanding of our system's overall structure. We successfully broke the system down into its individual components, which clarified the responsibilities of each module and how they interact with one another. Additionally, designing the UI helped us visualize the user experience, ensuring alignment with the system's functionality. This process also provided us with a clearer idea of the workload required for implementation, enabling better planning and resource allocation for the upcoming phases.

  \item What pain points did you experience during this deliverable, and how
    did you resolve them?
    One major pain point was syncing as a team on what the system should look like. Initially, there were differing opinions and ideas about the core functionalities and structure of the system. To address this, we held a team meeting where we collaboratively broke down the core functionalities of the system modules. During the meeting, we used a whiteboard to diagram the system structure, which helped us align our understanding and reach a consensus. This collaborative effort ensured everyone was on the same page moving forward.
  \item Which of your design decisions stemmed from speaking to your client(s)
  or a proxy (e.g. your peers, stakeholders, potential users)? For those that
  were not, why, and where did they come from?
  Currently, none of our design decisions have stemmed from speaking to our stakeholders or potential users, as we have not yet consulted them. Our plan is to present the design to stakeholders in the near future to gather their feedback and ensure alignment with their expectations. In the meantime, our design decisions have been based on internal team discussions and brainstorming sessions, where we leveraged our collective understanding of the system requirements and potential user needs.
  \item While creating the design doc, what parts of your other documents (e.g.
  requirements, hazard analysis, etc), it any, needed to be changed, and why?
  While creating the design document, we needed to modify SRS. Specifically, we talked about refining the requirements related to feature extraction as part of the system’s core functionality. This involved finalizing the set of features to be extracted, which we determined to be nine key features. These changes were necessary to ensure that the design document aligned with the system’s requirements and provided clarity for implementation.

  \item What are the limitations of your solution?  Put another way, given
  unlimited resources, what could you do to make the project better? (LO\_ProbSolutions)
  The primary limitations of our solution stem from constraints in time, resources, and access to advanced tools. For example, the accuracy of our feature extraction algorithms could be improved with access to more sophisticated machine learning models or advanced computational resources for real-time processing. Additionally, the user interface could be enhanced to include more dynamic and interactive elements, improving the overall user experience. Given unlimited resources, we would also invest in conducting extensive usability testing and obtaining feedback from a diverse group of stakeholders to ensure our system meets the needs of all potential users. Furthermore, integrating additional features such as real-time genre detection and support for multiple audio formats could significantly enhance the system’s versatility and appeal.
  \item Give a brief overview of other design solutions you considered.  What
  are the benefits and tradeoffs of those other designs compared with the chosen
  design?  From all the potential options, why did you select the documented design?
  (LO\_Explores)

  We considered a few alternative design solutions during the initial phases of the project. One option was to use a monolithic design where all the modules were tightly integrated into a single system. While this approach would have simplified communication between modules, it would have reduced modularity and made the system harder to maintain, test, and scale.

Another option was to use a distributed system with separate microservices for each feature extraction module. This design would have offered excellent scalability and flexibility but introduced significant complexity in terms of managing inter-module communication and dependencies.

We ultimately selected the documented design because it balances modularity and simplicity. By organizing the system into clearly defined modules with specific responsibilities, we can maintain a clear structure while minimizing complexity. This approach also allows us to allocate tasks efficiently among team members, ensure modular testing, and accommodate future changes or additions with minimal disruption.
\end{enumerate}


\end{document}