\documentclass[12pt, titlepage]{article}

\usepackage{amsmath, mathtools}

\usepackage[round]{natbib}
\usepackage{amsfonts}
\usepackage{amssymb}
\usepackage{graphicx}
\usepackage{colortbl}
\usepackage{xr}
\usepackage{hyperref}
\usepackage{longtable}
\usepackage{xfrac}
\usepackage{tabularx}
\usepackage{float}
\usepackage{siunitx}
\usepackage{booktabs}
\usepackage{multirow}
\usepackage[section]{placeins}
\usepackage{caption}
\usepackage{fullpage}

\hypersetup{
bookmarks=true,     % show bookmarks bar?
colorlinks=true,       % false: boxed links; true: colored links
linkcolor=red,          % color of internal links (change box color with linkbordercolor)
citecolor=blue,      % color of links to bibliography
filecolor=magenta,  % color of file links
urlcolor=cyan          % color of external links
}

\usepackage{array}

\externaldocument{../../SRS/SRS}

%% Comments

\usepackage{color}

\newif\ifcomments\commentstrue %displays comments
%\newif\ifcomments\commentsfalse %so that comments do not display

\ifcomments
\newcommand{\authornote}[3]{\textcolor{#1}{[#3 ---#2]}}
\newcommand{\todo}[1]{\textcolor{red}{[TODO: #1]}}
\else
\newcommand{\authornote}[3]{}
\newcommand{\todo}[1]{}
\fi

\newcommand{\wss}[1]{\authornote{blue}{SS}{#1}} 
\newcommand{\plt}[1]{\authornote{magenta}{TPLT}{#1}} %For explanation of the template
\newcommand{\an}[1]{\authornote{cyan}{Author}{#1}}

%% Common Parts

\newcommand{\progname}{Software Engineering} % PUT YOUR PROGRAM NAME HERE
\newcommand{\authname}{Team 8 -- Rhythm Rangers\\
\\ Ansel Chen
\\ Muhammad Jawad
\\ Mohamad-Hassan Bahsoun
\\ Matthew Baleanu
\\ Ahmed Al-Hayali} % AUTHOR NAMES                  

\usepackage{hyperref}
    \hypersetup{colorlinks=true, linkcolor=blue, citecolor=blue, filecolor=blue,
                urlcolor=blue, unicode=false}
    \urlstyle{same}
                                


\begin{document}

\title{Module Interface Specification for \progname{}}

\author{\authname}

\date{\today}

\maketitle

\pagenumbering{roman}

\section{Revision History}

\begin{tabularx}{\textwidth}{p{3cm}p{2cm}X}
\toprule {\bf Date} & {\bf Version} & {\bf Notes}\\
\midrule
Date 1 & 1.0 & Notes\\
Date 2 & 1.1 & Notes\\
\bottomrule
\end{tabularx}

~\newpage

\section{Symbols, Abbreviations and Acronyms}

See SRS Documentation at \wss{give url}

\wss{Also add any additional symbols, abbreviations or acronyms}

\newpage

\tableofcontents

\newpage

\pagenumbering{arabic}

\section{Introduction}

The following document details the Module Interface Specifications for
\wss{Fill in your project name and description}

Complementary documents include the System Requirement Specifications
and Module Guide.  The full documentation and implementation can be
found at \url{...}.  \wss{provide the url for your repo}

\section{Notation}

\wss{You should describe your notation.  You can use what is below as
  a starting point.}

The structure of the MIS for modules comes from \citet{HoffmanAndStrooper1995},
with the addition that template modules have been adapted from
\cite{GhezziEtAl2003}.  The mathematical notation comes from Chapter 3 of
\citet{HoffmanAndStrooper1995}.  For instance, the symbol := is used for a
multiple assignment statement and conditional rules follow the form $(c_1
\Rightarrow r_1 | c_2 \Rightarrow r_2 | ... | c_n \Rightarrow r_n )$.

The following table summarizes the primitive data types used by \progname. 

\begin{center}
\renewcommand{\arraystretch}{1.2}
\noindent 
\begin{tabular}{l l p{7.5cm}} 
\toprule 
\textbf{Data Type} & \textbf{Notation} & \textbf{Description}\\ 
\midrule
character & char & a single symbol or digit\\
integer & $\mathbb{Z}$ & a number without a fractional component in (-$\infty$, $\infty$) \\
natural number & $\mathbb{N}$ & a number without a fractional component in [1, $\infty$) \\
real & $\mathbb{R}$ & any number in (-$\infty$, $\infty$)\\
\bottomrule
\end{tabular} 
\end{center}

\noindent
The specification of \progname \ uses some derived data types: sequences, strings, and
tuples. Sequences are lists filled with elements of the same data type. Strings
are sequences of characters. Tuples contain a list of values, potentially of
different types. In addition, \progname \ uses functions, which
are defined by the data types of their inputs and outputs. Local functions are
described by giving their type signature followed by their specification.

\section{Module Decomposition}

The following table is taken directly from the Module Guide document for this project.

\begin{table}[h!]
\centering
\begin{tabular}{p{0.3\textwidth} p{0.6\textwidth}}
\toprule
\textbf{Level 1} & \textbf{Level 2}\\
\midrule

{Hardware-Hiding} & ~ \\
\midrule

\multirow{7}{0.3\textwidth}{Behaviour-Hiding} & GUI Module\\
& Audio File Input Module\\
& Search Query Module\\
%server compute elements
& Client Communication Module\\
& Server Communication Module\\
& Driver Module\\
%feature extractions
& Tempo (BPM) Feature Extraction Module\\
& Key and Scale Feature Extraction Module\\
& Instrument Type Feature Extraction Module\\
& Vocal Gender Feature Extraction Module\\
& Dynamic Range Feature Extraction Module\\
& Instrumentalness Feature Extraction Module\\
& Contour Feature Extraction Module\\
& Mood Feature Extraction Module\\
%Results and display
& Recommendation Module\\
& Program Results Interface\\
\midrule

\multirow{3}{0.3\textwidth}{Software Decision} & Database\\
& Spotify API\\
& Deezer API\\
& Genre Feature Module\\
\bottomrule

\end{tabular}
\caption{Module Hierarchy}
\label{TblMH}
\end{table}

\newpage
~\newpage

\section{GUI Module} \label{Module}

\subsection{GUI Module}

\subsection{Uses}
\begin{itemize}
  \item First-Match Text Field Input Module
  \item URL Input module
  \item Audio File Input Module
  \item Spotify Query Search \& Select
\end{itemize}

\subsection{Syntax}

\subsubsection{Exported Constants}
N/A

\subsubsection{Exported Access Programs}

\begin{center}
\begin{tabular}{p{2cm} p{4cm} p{4cm} p{2cm}}
\hline
\textbf{Name} & \textbf{In} & \textbf{Out} & \textbf{Exceptions}\\
\hline%by results, at this stage, we mean the output of the final program (could include features, includes recs)
Consolidate Inputs &Up to 4 collection(s) of reference(s) to track(s) &Merged collection of track references &-\\

\hline
\end{tabular}
\end{center}

\subsection{Semantics}

\subsubsection{State Variables}
\begin{itemize}
  \item Data type of the collection of track reference(s)
\end{itemize}

\subsubsection{Environment Variables}
N/A

\subsubsection{Assumptions}
N/A

\subsubsection{Access Routine Semantics}

\noindent consolidate\_inputs():
\begin{itemize}
  \item output: parses the user input and returns the songs that are sent to be processed
\end{itemize}
\subsubsection{Local Functions}
\begin{itemize}
  \item parse\_wav\_file(file)
  \begin{itemize}
    \item
  \end{itemize}
  \item parse\_url(url)
  \begin{itemize}
    \item 
  \end{itemize}
  \item parse\_text(text)
  \begin{itemize}
    \item 
  \end{itemize}
\end{itemize}

%subinput 1
\section{MIS of Audio File Input Module} 

\subsection{Audio File Input Module}
User inputs an audio file to the system to analyze. 

\subsection{Uses}
N/A

\subsection{Syntax}

\subsubsection{Exported Constants}
N/A

\subsubsection{Exported Access Programs}

\begin{center}
\begin{tabular}{p{2cm} p{4cm} p{4cm} p{2cm}}
\hline
\textbf{Name} & \textbf{In} & \textbf{Out} & \textbf{Exceptions}\\
\hline%by results, at this stage, we mean the output of the final program (could include features, includes recs)
On Input Button Press &Audio File &Collection of song reference(s) &Invalid File Type\\

\hline
\end{tabular}
\end{center}

\subsection{Semantics}

\subsubsection{State Variables}
\begin{itemize}
  \item Collection of track reference(s)
\end{itemize}

\subsubsection{Environment Variables}
N/A 

\subsubsection{Assumptions}
\begin{itemize}
  \item User has a properly named Audio File.
  \item User audio file input is actually a song. 
\end{itemize}

\subsubsection{Access Routine Semantics}

\noindent \wss{accessProg}():
\begin{itemize}
\item transition: \wss{if appropriate} 
\item output: \wss{if appropriate} 
\item exception: \wss{if appropriate} 
\end{itemize}

\wss{A module without environment variables or state variables is unlikely to
  have a state transition.  In this case a state transition can only occur if
  the module is changing the state of another module.}

\wss{Modules rarely have both a transition and an output.  In most cases you
  will have one or the other.}

\subsubsection{Local Functions}

\wss{As appropriate} \wss{These functions are for the purpose of specification.
  They are not necessarily something that is going to be implemented
  explicitly.  Even if they are implemented, they are not exported; they only
  have local scope.}

%subinput 2
\section{MIS of Search Query Module} 

\subsection{Search Query Module}
User inputs a song and that is turned into a spotify search query where the top 10 matches are available for user to select

\subsection{Uses}
N/A

\subsection{Syntax}

\subsubsection{Exported Constants}
N/A

\subsubsection{Exported Access Programs}

\begin{center}
\begin{tabular}{p{2cm} p{4cm} p{4cm} p{2cm}}
\hline
\textbf{Name} & \textbf{In} & \textbf{Out} & \textbf{Exceptions}\\
\hline%by results, at this stage, we mean the output of the final program (could include features, includes recs)
Search Query Request &text input &top 10 matches from spotify query search &N/A\\
Output result selection &user selection &Collection containing track reference &N/A\\
\hline
\end{tabular}
\end{center}

\subsection{Semantics}

\subsubsection{State Variables}
\begin{itemize}
  \item Collection containing track reference
\end{itemize}

\subsubsection{Environment Variables}
\begin{itemize}
  \item Spotify Client ID
  \item Spotify Client Secret
\end{itemize}

\subsubsection{Assumptions}
N/A

\subsubsection{Access Routine Semantics}

\noindent \wss{accessProg}():
\begin{itemize}
\item transition: \wss{if appropriate} 
\item output: \wss{if appropriate} 
\item exception: \wss{if appropriate} 
\end{itemize}

\wss{A module without environment variables or state variables is unlikely to
  have a state transition.  In this case a state transition can only occur if
  the module is changing the state of another module.}

\wss{Modules rarely have both a transition and an output.  In most cases you
  will have one or the other.}

\subsubsection{Local Functions}

\wss{As appropriate} \wss{These functions are for the purpose of specification.
  They are not necessarily something that is going to be implemented
  explicitly.  Even if they are implemented, they are not exported; they only
  have local scope.}

%server compute
\section{MIS of Client Communication Module} 

\subsection{Client Communication Module}
User inputs a song and that is turned into a spotify search query where the top 10 matches are available for user to select

\subsection{Uses}
N/A

\subsection{Syntax}

\subsubsection{Exported Constants}
N/A

\subsubsection{Exported Access Programs}

\begin{center}
\begin{tabular}{p{2cm} p{4cm} p{4cm} p{2cm}}
\hline
\textbf{Name} & \textbf{In} & \textbf{Out} & \textbf{Exceptions}\\
\hline%by results, at this stage, we mean the output of the final program (could include features, includes recs)
Search Query Request &text input &top 10 matches from spotify query search &N/A\\
Output result selection &user selection &Collection containing track reference &N/A\\
\hline
\end{tabular}
\end{center}

\subsection{Semantics}

\subsubsection{State Variables}
\begin{itemize}
  \item Collection containing track reference
\end{itemize}

\subsubsection{Environment Variables}
\begin{itemize}
  \item Spotify Client ID
  \item Spotify Client Secret
\end{itemize}

\subsubsection{Assumptions}
N/A

\subsubsection{Access Routine Semantics}

\noindent \wss{accessProg}():
\begin{itemize}
\item transition: \wss{if appropriate} 
\item output: \wss{if appropriate} 
\item exception: \wss{if appropriate} 
\end{itemize}

\wss{A module without environment variables or state variables is unlikely to
  have a state transition.  In this case a state transition can only occur if
  the module is changing the state of another module.}

\wss{Modules rarely have both a transition and an output.  In most cases you
  will have one or the other.}

\subsubsection{Local Functions}

\wss{As appropriate} \wss{These functions are for the purpose of specification.
  They are not necessarily something that is going to be implemented
  explicitly.  Even if they are implemented, they are not exported; they only
  have local scope.}

%separator
\section{MIS of Server Communication Module} 

\subsection{Server Communication Module}
User inputs a song and that is turned into a spotify search query where the top 10 matches are available for user to select

\subsection{Uses}
N/A

\subsection{Syntax}

\subsubsection{Exported Constants}
N/A

\subsubsection{Exported Access Programs}

\begin{center}
\begin{tabular}{p{2cm} p{4cm} p{4cm} p{2cm}}
\hline
\textbf{Name} & \textbf{In} & \textbf{Out} & \textbf{Exceptions}\\
\hline%by results, at this stage, we mean the output of the final program (could include features, includes recs)
Search Query Request &text input &top 10 matches from spotify query search &N/A\\
Output result selection &user selection &Collection containing track reference &N/A\\
\hline
\end{tabular}
\end{center}

\subsection{Semantics}

\subsubsection{State Variables}
\begin{itemize}
  \item Collection containing track reference
\end{itemize}

\subsubsection{Environment Variables}
\begin{itemize}
  \item Spotify Client ID
  \item Spotify Client Secret
\end{itemize}

\subsubsection{Assumptions}
N/A

\subsubsection{Access Routine Semantics}

\noindent \wss{accessProg}():
\begin{itemize}
\item transition: \wss{if appropriate} 
\item output: \wss{if appropriate} 
\item exception: \wss{if appropriate} 
\end{itemize}

\wss{A module without environment variables or state variables is unlikely to
  have a state transition.  In this case a state transition can only occur if
  the module is changing the state of another module.}

\wss{Modules rarely have both a transition and an output.  In most cases you
  will have one or the other.}

\subsubsection{Local Functions}

\wss{As appropriate} \wss{These functions are for the purpose of specification.
  They are not necessarily something that is going to be implemented
  explicitly.  Even if they are implemented, they are not exported; they only
  have local scope.}

%separator
\section{MIS of Driver Module} 

\subsection{Driver Module}
User inputs a song and that is turned into a spotify search query where the top 10 matches are available for user to select

\subsection{Uses}
N/A

\subsection{Syntax}

\subsubsection{Exported Constants}
N/A

\subsubsection{Exported Access Programs}

\begin{center}
\begin{tabular}{p{2cm} p{4cm} p{4cm} p{2cm}}
\hline
\textbf{Name} & \textbf{In} & \textbf{Out} & \textbf{Exceptions}\\
\hline%by results, at this stage, we mean the output of the final program (could include features, includes recs)
Search Query Request &text input &top 10 matches from spotify query search &N/A\\
Output result selection &user selection &Collection containing track reference &N/A\\
\hline
\end{tabular}
\end{center}

\subsection{Semantics}

\subsubsection{State Variables}
\begin{itemize}
  \item Collection containing track reference
\end{itemize}

\subsubsection{Environment Variables}
\begin{itemize}
  \item Spotify Client ID
  \item Spotify Client Secret
\end{itemize}

\subsubsection{Assumptions}
N/A

\subsubsection{Access Routine Semantics}

\noindent \wss{accessProg}():
\begin{itemize}
\item transition: \wss{if appropriate} 
\item output: \wss{if appropriate} 
\item exception: \wss{if appropriate} 
\end{itemize}

\wss{A module without environment variables or state variables is unlikely to
  have a state transition.  In this case a state transition can only occur if
  the module is changing the state of another module.}

\wss{Modules rarely have both a transition and an output.  In most cases you
  will have one or the other.}

\subsubsection{Local Functions}

\wss{As appropriate} \wss{These functions are for the purpose of specification.
  They are not necessarily something that is going to be implemented
  explicitly.  Even if they are implemented, they are not exported; they only
  have local scope.} 

% MIS of Featurizer Module
\section{MIS of Featurizer Module}

\subsection{Featurizer Module}
The Featurizer Module is responsible for extracting 9 distinct feature values from audio files:
\begin{itemize}
    \item Tempo
    \item Key and Scale
    \item Instrument Type
    \item Vocal Gender
    \item Dynamic Range
    \item Instrumentalness
    \item Contour
    \item Mood
    \item Genre
\end{itemize}

The module invokes sub-feature modules to compute these feature values. It consolidates the results into a single \texttt{FeatureValues} object and returns it to the Driver Module.

%MIS of Feature Extraction algorithms
%MIS of Tempo(BPM) extraction
\section{MIS of Tempo (BPM) Feature Extraction Module} 

\subsection{Tempo (BPM) Feature Extraction Module}

\subsection{Uses}
N/A

\subsection{Syntax}

\subsubsection{Exported Constants}
\textcolor{red}{N/A} 

\subsubsection{Exported Access Programs}

\begin{center}
\begin{tabular}{p{2cm} p{4cm} p{4cm} p{2cm}}
\hline
\textbf{Name} & \textbf{In} & \textbf{Out} & \textbf{Exceptions}\\
\hline%by results, at this stage, we mean the output of the final program (could include features, includes recs)
Extract Tempo &Audio time series (\texttt{np.ndarray}) &Song Tempo $\in \mathbb{R}$ &N/A \\
\hline
\end{tabular}
\end{center}

\subsection{Semantics}

\subsubsection{State Variables}
\textcolor{red}{N/A}

\subsubsection{Environment Variables}
N/A

\subsubsection{Assumptions}
Valid audio file with coherent song information.

\subsubsection{Access Routine Semantics}

\noindent \texttt{ExtractTempo}():
\begin{itemize}
\item transition: \textcolor{red}{N/A}
\item output: \texttt{SongTempo} : = \texttt{ExtractTempo(Audio\textunderscore Time\textunderscore Series)}
\item exception: \textcolor{red}{N/A}
\end{itemize}

\subsubsection{Local Functions}
N/A



%Key and Scale Feature Extraction Module
\section{MIS of Key and Scale Feature Extraction Module} 

\subsection{Key and Scale Feature Extraction Module}

\subsection{Uses}
N/A

\subsection{Syntax}

\subsubsection{Exported Constants}
\textcolor{red}{N/A}

\subsubsection{Exported Access Programs}

\begin{center}
\begin{tabular}{p{2cm} p{4cm} p{4cm} p{2cm}}
\hline
\textbf{Name} & \textbf{In} & \textbf{Out} & \textbf{Exceptions}\\
\hline%by results, at this stage, we mean the output of the final program (could include features, includes recs)
Extract Key \& Scale &Audio time series (\texttt{np.ndarray}) & Song Key, Scale $\in \mathbb{Z}^2$ & N/A \\
\hline
\end{tabular}
\end{center}

\subsection{Semantics}

\subsubsection{State Variables}
\textcolor{red}{N/A}

\subsubsection{Environment Variables}
N/A

\subsubsection{Assumptions}
Valid audio file with coherent song information.

\subsubsection{Access Routine Semantics}

\noindent \texttt{Extract\textunderscore Key\textunderscore Scale}():
\begin{itemize}
\item transition: \textcolor{red}{N/A}
\item output: \texttt{SongKey, SongScale} : = \texttt{Extract\textunderscore Key\textunderscore Scale(Audio\textunderscore Time\textunderscore Series)}
\item exception: \textcolor{red}{N/A}
\end{itemize}

\subsubsection{Local Functions}
N/A


%Instrument Type Feature
\section{MIS of Instrument Type Feature Extraction Module} 

\subsection{Instrument Type Feature Extraction Module}

\subsection{Uses}
N/A

\subsection{Syntax}

\subsubsection{Exported Constants}
\textcolor{red}{N/A}

\subsubsection{Exported Access Programs}

\begin{center}
\begin{tabular}{p{2cm} p{4cm} p{4cm} p{2cm}}
\hline
\textbf{Name} & \textbf{In} & \textbf{Out} & \textbf{Exceptions}\\
\hline%by results, at this stage, we mean the output of the final program (could include features, includes recs)
Extract Instrument Type &Audio time series (\texttt{np.ndarray}) & Instrument Type $\in \mathbb{Z}^k$ & N/A \\
\hline
\end{tabular}
\end{center}

\subsection{Semantics}

\subsubsection{State Variables}
\textcolor{red}{N/A}

\subsubsection{Environment Variables}
N/A

\subsubsection{Assumptions}
Valid audio file with coherent song information.

\subsubsection{Access Routine Semantics}

\noindent \texttt{Extract\textunderscore Instrument\textunderscore Type}():
\begin{itemize}
\item transition: \textcolor{red}{N/A} 
\item output: \texttt{InstrumentType} : = \texttt{Extract\textunderscore Instrument\textunderscore Type(Audio\textunderscore Time\textunderscore Series)}
\item exception: \textcolor{red}{N/A} 
\end{itemize}

\subsubsection{Local Functions}
\textcolor{red}{N/A}


%MIS Vocal Gender Feature
\section{MIS of Vocal Gender Feature Extraction Module} 

\subsection{MIS of Vocal Gender Feature Extraction Module}
This feature seeks to quantify whether the voices features in the inputted audio file 
are largely more feminine or masculine sounding. This is represented by a float with a range between
0 and 1 where 0 means only "masculine" sound signatures are contained and 1 means only "feminine" sounds,
where values in-between represent a blend. 
%IMO this conceptions is the easiest way to do it. If you did a 0 or 1 integer the representation wouldn't make
%useful sense for songs where there are vocalists of different genders. 

\subsection{Uses}
N/A

\subsection{Syntax}

\subsubsection{Exported Constants}
N/A

\subsubsection{Exported Access Programs}

\begin{center}
\begin{tabular}{p{2cm} p{4cm} p{4cm} p{2cm}}
\hline
\textbf{Name} & \textbf{In} & \textbf{Out} & \textbf{Exceptions}\\
\hline%by results, at this stage, we mean the output of the final program (could include features, includes recs)
Extract Vocal Gender &Audio time series (\texttt{np.ndarray}) &Vocal Gender $\in \mathbb{R}$ &N/A\\
\hline
\end{tabular}
\end{center}

\subsection{Semantics}

\subsubsection{State Variables}
\textcolor{red}{N/A}

\subsubsection{Environment Variables}
N/A

\subsubsection{Assumptions}
Valid audio file with coherent song information.

\subsubsection{Access Routine Semantics}

\noindent \texttt{Extract\textunderscore Vocal\textunderscore Gender}():
\begin{itemize}
\item transition: \textcolor{red}{N/A}
\item output: \texttt{VocalGender} : = \texttt{Extract\textunderscore Vocal\textunderscore Gender(Audio\textunderscore Time\textunderscore Series)}
\item exception: \textcolor{red}{N/A}
\end{itemize}

\subsubsection{Local Functions}
\textcolor{red}{N/A}

%Dynamic Range Feature
\section{MIS of Dynamic Range Feature Extraction Module} 

\subsection{Dynamic Range Feature Extraction Module}
Feature extracts the range of sounds (difference between peak and through) of the audio signal.

\subsection{Uses}
N/A

\subsection{Syntax}

\subsubsection{Exported Constants}
N/A

\subsubsection{Exported Access Programs}

\begin{center}
\begin{tabular}{p{2cm} p{4cm} p{4cm} p{2cm}}
\hline
\textbf{Name} & \textbf{In} & \textbf{Out} & \textbf{Exceptions}\\
\hline%by results, at this stage, we mean the output of the final program (could include features, includes recs)
Extract Dynamic Range &Audio time series (\texttt{np.ndarray}) &Dynamic Range (decibels) $\in \mathbb{R}$ &N/A\\
\hline
\end{tabular}
\end{center}

\subsection{Semantics}

\subsubsection{State Variables}
\textcolor{red}{N/A}

\subsubsection{Environment Variables}
N/A

\subsubsection{Assumptions}
Valid audio file with coherent song information.

\subsubsection{Access Routine Semantics}

\noindent \texttt{Extract\textunderscore Dynamic\textunderscore Range}():
\begin{itemize}
\item transition: N/A
\item output: \texttt{DynamicRange} : = \texttt{Extract\textunderscore Dynamic\textunderscore Range(Audio\textunderscore Time\textunderscore Series)}
\item exception: N/A
\end{itemize}

\subsubsection{Local Functions}
N/A

%Instrumentalness
\section{MIS of Instrumentalness Feature Extraction Module} 

\subsection{Instrumentalness Feature Extraction Module}
Extracts the how prominent instrumental sounds are within the song. Represented by a float
variable where the range is between 0 and 1, where higher values mean more instrumental sounds
and lower means less. Eg, 0 would mean an acapella piece of music, 1 would be something that purely
features instruments.  

\subsection{Uses}
N/A

\subsection{Syntax}

\subsubsection{Exported Constants}
N/A

\subsubsection{Exported Access Programs}

\begin{center}
\begin{tabular}{p{2cm} p{4cm} p{4cm} p{2cm}}
\hline
\textbf{Name} & \textbf{In} & \textbf{Out} & \textbf{Exceptions}\\
\hline%by results, at this stage, we mean the output of the final program (could include features, includes recs)
Extract Instrumentalness &Audio time series (\texttt{np.ndarray}) &Instrumentalness $\in \mathbb{R}$ &N/A\\
\hline
\end{tabular}
\end{center}

\subsection{Semantics}

\subsubsection{State Variables}
\textcolor{red}{N/A}

\subsubsection{Environment Variables}
N/A

\subsubsection{Assumptions}
Valid audio file with coherent song information.

\subsubsection{Access Routine Semantics}

\noindent \texttt{Extract\textunderscore Instrumentalness}():
\begin{itemize}
\item transition: N/A
\item output: \texttt{Instrumentalness} : = \texttt{Extract\textunderscore Instrumentalness(Audio\textunderscore Time\textunderscore Series)}
\item exception: N/A
\end{itemize}

\subsubsection{Local Functions}
N/A

%Contour Extraction
\section{MIS of Contour Feature Extraction Module} 

\subsection{Contour Feature Extraction Module}

\subsection{Uses}
N/A

\subsection{Syntax}

\subsubsection{Exported Constants}
N/A

\subsubsection{Exported Access Programs}

\begin{center}
\begin{tabular}{p{2cm} p{4cm} p{4cm} p{2cm}}
\hline
\textbf{Name} & \textbf{In} & \textbf{Out} & \textbf{Exceptions}\\
\hline%by results, at this stage, we mean the output of the final program (could include features, includes recs)
Extract Melodic Contour &Audio time series (\texttt{np.ndarray}) &output &N/A\\
\hline
\end{tabular}
\end{center}

\subsection{Semantics}

\subsubsection{State Variables}
\textcolor{red}{N/A}

\subsubsection{Environment Variables}
N/A

\subsubsection{Assumptions}
Valid audio file with coherent song information.

\subsubsection{Access Routine Semantics}

\noindent \texttt{Extract\textunderscore Melodic\textunderscore Contour()}:
\begin{itemize}
\item transition: N/A 
\item output: \texttt{variable name} : = \texttt{Extract\textunderscore Melodic\textunderscore Contour(Audio\textunderscore Time\textunderscore Series)}
\item exception: N/A
\end{itemize}

\subsubsection{Local Functions}
N/A

%Mood extraction
\section{MIS of Mood Feature Extraction Module} 

\subsection{Mood Feature Extraction Module}

\subsection{Uses}
N/A

\subsection{Syntax}

\subsubsection{Exported Constants}
N/A

\subsubsection{Exported Access Programs}

\begin{center}
\begin{tabular}{p{2cm} p{4cm} p{4cm} p{2cm}}
\hline
\textbf{Name} & \textbf{In} & \textbf{Out} & \textbf{Exceptions}\\
\hline%by results, at this stage, we mean the output of the final program (could include features, includes recs)
Extract Mood &Audio time series (\texttt{np.ndarray}) &Mood $\in{\mathbb{Z}}$ &N/A\\
\hline
\end{tabular}
\end{center}

\subsection{Semantics}

\subsubsection{State Variables}
\textcolor{red}{N/A}

\subsubsection{Environment Variables}
N/A

\subsubsection{Assumptions}
Valid audio file with coherent song information.

\subsubsection{Access Routine Semantics}

\noindent \texttt{Extract\textunderscore Mood()}:
\begin{itemize}
\item transition: N/A 
\item output: \texttt{Mood} : = \texttt{Extract\textunderscore Mood(Audio\textunderscore Time\textunderscore Series)}
\item exception: N/A
\end{itemize}

\subsubsection{Local Functions}
N/A

%Recommendation module
\section{MIS of Recommendation Module} 

\subsection{Recommendation Module}
%for features input I assume it is an array of features (which is an array itself) and then the 
%recs are generated this way. I'm not 100% certain this is how, just placeholder for now. 

%also, we need to determine how the song recommendations are represented. Should they just be names? or spotify url links? etc. 
%i've currently left this as abstract as I think it is possible - just as a set of songs where the type is nebulous

\subsection{Uses}
\begin{itemize}
  \item Tempo (BPM) Feature Extraction Module
  \item Key and Scale Feature Extraction Module
  \item Instrument Type Feature Extraction Module
  \item Vocal Gender Feature Extraction Module
  \item Dynamic Range Feature Extraction Module
  \item Instrumentalness Feature Extraction Module
  \item Contour Feature Extraction Module
  \item Mood Feature Extraction Module
  \item Driver Module
  \item Spotify API
\end{itemize}

\subsection{Syntax}

\subsubsection{Exported Constants}
N/A

\subsubsection{Exported Access Programs}

\begin{center}
\begin{tabular}{p{2cm} p{4cm} p{4cm} p{2cm}}
\hline
\textbf{Name} & \textbf{In} & \textbf{Out} & \textbf{Exceptions}\\
\hline
Generate Recommendations &Song \textunderscore Features (\texttt{np.ndarray[Features])}) &Song List &N/A\\
\hline
\end{tabular}
\end{center}

\subsection{Semantics}

\subsubsection{State Variables}
\textcolor{red}{N/A}

\subsubsection{Environment Variables}
N/A

\subsubsection{Assumptions}
N/A

\subsubsection{Access Routine Semantics}

\noindent \wss{accessProg}():
\begin{itemize}
\item transition: N/A
\item output: \texttt{Recommended\textunderscore Songs} : = \texttt{GenerateRecommendations(Song\textunderscore Features)} 
\item exception: N/A
\end{itemize}

\subsubsection{Local Functions}

\wss{As appropriate} \wss{These functions are for the purpose of specification.
  They are not necessarily something that is going to be implemented
  explicitly.  Even if they are implemented, they are not exported; they only
  have local scope.}



%MIS-20 Results Display Interface
\section{MIS of Program Results Interface Module} 

\subsection{Program Results Interface Module}
%should just be used to display the generated recommendations. need to decide how that is represented, is it links? do we show 
%the features? etc. 

\subsection{Uses}
\begin{itemize}
  \item Recommendation Module
\end{itemize}

\subsection{Syntax}

\subsubsection{Exported Constants}
N/A

\subsubsection{Exported Access Programs}

\begin{center}
\begin{tabular}{p{2cm} p{4cm} p{4cm} p{2cm}}
\hline
\textbf{Name} & \textbf{In} & \textbf{Out} & \textbf{Exceptions}\\
\hline
Display Results &input &output &N/A\\
\hline
\end{tabular}
\end{center}

\subsection{Semantics}

\subsubsection{State Variables}
\textcolor{red}{N/A}

\subsubsection{Environment Variables}
N/A

\subsubsection{Assumptions}
N/A

\subsubsection{Access Routine Semantics}

\noindent \wss{accessProg}():
\begin{itemize}
\item transition: \wss{if appropriate} 
\item output: \wss{if appropriate} 
\item exception: \wss{if appropriate} 
\end{itemize}

\wss{A module without environment variables or state variables is unlikely to
  have a state transition.  In this case a state transition can only occur if
  the module is changing the state of another module.}

\wss{Modules rarely have both a transition and an output.  In most cases you
  will have one or the other.}

\subsubsection{Local Functions}

\wss{As appropriate} \wss{These functions are for the purpose of specification.
  They are not necessarily something that is going to be implemented
  explicitly.  Even if they are implemented, they are not exported; they only
  have local scope.}

\newpage

\bibliographystyle {plainnat}
\bibliography {../../../refs/References}

\newpage

\section{Appendix} \label{Appendix}

\wss{Extra information if required}

\newpage{}

\section*{Appendix --- Reflection}

\wss{Not required for CAS 741 projects}

The information in this section will be used to evaluate the team members on the
graduate attribute of Problem Analysis and Design.

The purpose of reflection questions is to give you a chance to assess your own
learning and that of your group as a whole, and to find ways to improve in the
future. Reflection is an important part of the learning process.  Reflection is
also an essential component of a successful software development process.  

Reflections are most interesting and useful when they're honest, even if the
stories they tell are imperfect. You will be marked based on your depth of
thought and analysis, and not based on the content of the reflections
themselves. Thus, for full marks we encourage you to answer openly and honestly
and to avoid simply writing ``what you think the evaluator wants to hear.''

Please answer the following questions.  Some questions can be answered on the
team level, but where appropriate, each team member should write their own
response:


\begin{enumerate}
  \item What went well while writing this deliverable? 
  \item What pain points did you experience during this deliverable, and how
    did you resolve them?
  \item Which of your design decisions stemmed from speaking to your client(s)
  or a proxy (e.g. your peers, stakeholders, potential users)? For those that
  were not, why, and where did they come from?
  \item While creating the design doc, what parts of your other documents (e.g.
  requirements, hazard analysis, etc), it any, needed to be changed, and why?
  \item What are the limitations of your solution?  Put another way, given
  unlimited resources, what could you do to make the project better? (LO\_ProbSolutions)
  \item Give a brief overview of other design solutions you considered.  What
  are the benefits and tradeoffs of those other designs compared with the chosen
  design?  From all the potential options, why did you select the documented design?
  (LO\_Explores)
\end{enumerate}


\end{document}