\documentclass{article}

\usepackage{booktabs}
\usepackage{tabularx}
\usepackage{hyperref}

%% Comments

\usepackage{color}

\newif\ifcomments\commentstrue %displays comments
%\newif\ifcomments\commentsfalse %so that comments do not display

\ifcomments
\newcommand{\authornote}[3]{\textcolor{#1}{[#3 ---#2]}}
\newcommand{\todo}[1]{\textcolor{red}{[TODO: #1]}}
\else
\newcommand{\authornote}[3]{}
\newcommand{\todo}[1]{}
\fi

\newcommand{\wss}[1]{\authornote{blue}{SS}{#1}} 
\newcommand{\plt}[1]{\authornote{magenta}{TPLT}{#1}} %For explanation of the template
\newcommand{\an}[1]{\authornote{cyan}{Author}{#1}}

%% Common Parts

\newcommand{\progname}{Software Engineering} % PUT YOUR PROGRAM NAME HERE
\newcommand{\authname}{Team 8 -- Rhythm Rangers\\
\\ Ansel Chen
\\ Muhammad Jawad
\\ Mohamad-Hassan Bahsoun
\\ Matthew Baleanu
\\ Ahmed Al-Hayali} % AUTHOR NAMES                  

\usepackage{hyperref}
    \hypersetup{colorlinks=true, linkcolor=blue, citecolor=blue, filecolor=blue,
                urlcolor=blue, unicode=false}
    \urlstyle{same}
                                


\title{User Guide\\\progname}

\author{\authname}

\date{2025-04-03}

\begin{document}

\begin{table}[hp]
\caption{Revision History} \label{TblRevisionHistory}
\begin{tabularx}{\textwidth}{llX}
\toprule
\textbf{Date} & \textbf{Developer(s)} & \textbf{Change}\\
\midrule
Date1 & Name(s) & Description of changes\\
Date2 & Name(s) & Description of changes\\
... & ... & ...\\
\bottomrule
\end{tabularx}  
\end{table}

\newpage

\maketitle
\tableofcontents
\newpage

\section{Introduction}
This user guide provides instructions for setting up and launching the GenreGuru application.
The frontend is built using react and node js, and is launched by using \texttt{npm start} and communicates with a backend service to provide recommendations. 
This guide assumes the user has already cloned the project repository.

\section{Installation Requirements}
Before launching the frontend, ensure that all dependencies are installed correctly.

\subsection{Python Backend Dependencies}
The backend dependencies are specified in \texttt{setup.py}. To install them, navigate to the backend root directory and run:

\begin{verbatim}
pip install -e .
\end{verbatim}

\subsection{Frontend Dependencies}
Navigate to the frontend directory (\texttt{cd src/client/genre-guru}) and run \texttt{npm install}

\subsection{Audit and Fix Security Issues}
Use \texttt{npm audit fix} in order to resolve the issues that arise during \texttt{npm install}. 

\subsection{Setup a .env.local file}
First, use spotify's developer dashboard in order to generate your client id and client secret.
Then, navigate to \texttt{cd src/client/genre-guru} and create a .env.local file, with two lines corresponding to\texttt{REACT\_APP\_SPOTIFY\_CLIENT\_ID, REACT\_APP\_SPOTIFY\_CLIENT\_SECRET},
and set them equal to your client id and client secret.

\section{Launching the Frontend Application}
Once the dependencies are installed, you can start the frontend application by running the following commands:

\begin{verbatim}
cd src/client/genre-gur
npm start
\end{verbatim}

This will launch the frontend in your default web browser at \url{http://localhost:3000}.

\section{Troubleshooting}
\begin{itemize}
  \item If there is 500 error when receiving recommendations, verify that the backend is running and accessible.
  \item If there is 400 error when making API calls, verify that the .env.local file is correct and/or the client id and client secret are correct.
  \item If modules fail to install via \texttt{npm install}, try deleting \texttt{node\_modules/} and \texttt{package-lock.json}, then reinstall.
\end{itemize}

\section{Recorded User Guide}

Finally, here is a video recording of the user guide: \href{https://drive.google.com/file/d/1wVYMzri1Ggy7P0_b0EQj4uQOwrtJOPV2/view?usp=drive_link}{link.}

\end{document}