% THIS DOCUMENT IS FOLLOWS THE VOLERE TEMPLATE BY Suzanne Robertson and James Robertson
% ONLY THE SECTION HEADINGS ARE PROVIDED
%
% Initial draft from https://github.com/Dieblich/volere
%
% Risks are removed because they are covered by the Hazard Analysis
\documentclass[12pt]{article}
\usepackage[normalem]{ulem} % Keeps \emph in italics
\usepackage{booktabs}
\usepackage{tabularx}
\usepackage{xcolor}
\usepackage{hyperref}
\usepackage{graphicx}
\usepackage{array}
\graphicspath{ {./diagrams/} }

\hypersetup{
    bookmarks=true,         % show bookmarks bar?
      colorlinks=true,      % false: boxed links; true: colored links
    linkcolor=red,          % color of internal links (change box color with linkbordercolor)
    citecolor=green,        % color of links to bibliography
    filecolor=magenta,      % color of file links
    urlcolor=cyan           % color of external links
}

\newcommand{\lips}{\textit{Insert your content here.}}

\input{../Comments}
%% Common Parts

\newcommand{\progname}{GenreGuru} % PUT YOUR PROGRAM NAME HERE
\newcommand{\authname}{Team 8 -- Rhythm Rangers\\
\\ Ansel Chen
\\ Muhammad Jawad
\\ Mohamad-Hassan Bahsoun
\\ Matthew Baleanu
\\ Ahmed Al-Hayali} % AUTHOR NAMES                  

\usepackage{hyperref}
    \hypersetup{colorlinks=true, linkcolor=blue, citecolor=blue, filecolor=blue,
                urlcolor=blue, unicode=false}
    \urlstyle{same}
                                


\begin{document}

\title{Software Requirements Specification for \progname: subtitle describing software} 
\author{\authname}
\date{October 11, 2024}
	
\maketitle

~\newpage

\pagenumbering{roman}

\tableofcontents

~\newpage

\section*{Revision History}

\begin{tabularx}{\textwidth}{p{3cm}p{2cm}X}
\toprule {\textbf{Date}} & {\textbf{Version}} & {\textbf{Notes}}\\
\midrule
2024-10-11 & 0.0 & Revision 0\\
\bottomrule
\end{tabularx}

~\\

~\newpage
\section{Purpose of the Project}
\subsection{User Business}
\textcolor{red}{\sout{Experimentation in music production is a process driven by intuition, i.e., lacking a core systematic structure, limited by a producer's experience and exposure to complex tools and techniques. \emph{GenreGuru} strives to greatly reduce the effort involved in \emph{methodically} attaining exposure to the use of tools and techniques in other songs. \emph{GenreGuru}, by extension, democratizes access to experimentation in music production to less experienced producers, hobbyist musicians, and novices in music production.}}

\textcolor{red}{Discovering new music that aligns with an individual's unique taste can be overwhelming, especially in todays vast music industry. 
GenreGuru simplifies this process by leveraging a content-based recommendation system. 
Rather than relying on user ratings or external popularity metrics, GenreGuru analyzes the intrinsic features of each song, 
such as spectral characteristics, tempo, and key, to identify and suggest tracks that closely match a user's current favorite. 
This approach empowers casual listeners, hobbyists, and music enthusiasts to effortlessly explore new music that resonates with their preferences, 
transforming the music discovery into a more engaging experience.}

\subsection{Goals of the Project}

\begin{itemize}
  \item \textcolor{red}{\sout{\emph{featurize} -- produce tabular features corresponding to characteristics of input songs;}}
  \item \textcolor{red}{\sout{\emph{recommend} -- produce a collection of songs similar to input songs;}}
  \item \textcolor{red}{\sout{\emph{generate} -- produce an audio artifact similar to input reference songs.}}  
\end{itemize}


\textcolor{red}{The goal of the project is to develop a fast, accurate, and intuitive music recommendation system that leverages detailed audio features to suggest 
tracks closely aligned with a listener’s unique taste. By analyzing intrinsic song characteristics such as spectral properties, tempo, and key, 
GenreGuru delivers personalized recommendations through a user-friendly interface that requires minimal learning effort. 
If successful, this system will enhance personalized music discovery for a broad range of users, fostering deeper engagement with their 
favorite genres.}

\section{Stakeholders}
\subsection{Client} \label{2.1-client}
The project is academic in nature, hence has no formal clients beyond the supervisor, who will be consulted periodically to direct project effort.
\subsection{Customer} \label{2.2-customer}
Please refer to \ref{2.3-other-stakeholders} and \ref{2.4-hands-on-users} for the current characterization of candidate customers. Section \ref{2.5-personas} will more succinctly specify archetypal customers after candidate customer interviews are carried out.
\subsection{Other Stakeholders} \label{2.3-other-stakeholders}
\begin{itemize}
  \item \textcolor{red}{\sout{Subject Matter Experts (SMEs) -- \emph{To protect the privacy of our stakeholders, SMEs are completely deidentified with the exception of the user group they fall into, if applicable. Were this to be a commercial project, we acknowledge that we would have to, under jurisdiction of the Office of the Privacy Commissioner of Canada, abide by The \href{https://www.priv.gc.ca/en/privacy-topics/privacy-laws-in-canada/the-personal-information-protection-and-electronic-documents-act-pipeda/}{Personal Information Protection and Electronic Documents Act (PIPEDA)}. Further, we have elected, for the interim, not to include a formal conflict resolution agreement as stakeholders' interests will be considered only under discretion of the development team.}}}
  
  \begin{itemize}
    \item \textcolor{red}{\sout{Music Producers \& Sound Engineers (subsequently ``producers'')}}
    \begin{itemize}
      \item \textcolor{red}{\sout{\emph{Target subject matter knowledge:} description of current process and/or approach used to guide recording artists to explore or experiment with a new \textcolor{red}{sound}. For example, ``\emph{while recording, if I get an idea, I play a song with a specific cool feature (take tempo for example) and iteratively incorporate it into the current song being recorded, guiding the artist to adjust (e.g., incorporating a different cadence) across different attempts to incorporate the experimental feature}''.}} 
      \item \textcolor{red}{\sout{\emph{Extent of project involvement:} minimal, i.e., no more than three interviews per producer.}}
      \item \textcolor{red}{\sout{\emph{Influence on project:} moderate-low -- technology-keen producers may be more likely to already have a process into which \emph{GenreGuru} can be integrated, i.e., song featurization can quickly, and in large volumes, summarize music that the producer's target audience listens to, allowing the producer to better tailor their music output. Such a producer's insights can inform and guide development, but at the discretion of the development team.}}
    \end{itemize}
    
    \item \textcolor{red}{\sout{Musicians}} 
    \begin{itemize}
      \item \textcolor{red}{\sout{\emph{Target subject matter knowledge:} description of current process and/or approach used to generate novel ideas for unrecorded songs or experimenting with different ideas for already recorded songs. \emph{We must be cautious so as to only consider the experimentation component of the musician's workflow, not the music creation in its core.}}}
      \item \textcolor{red}{\sout{\emph{Extent of project involvement:} minimal, i.e., no more than three interviews per musician.}} 
      \item \textcolor{red}{\sout{\emph{Influence on project:} moderate -- like technology-keen producers, we suspect musicians may already have a process into which \emph{GenreGuru} can be integrated, i.e., song recommendation can quickly, and in large volumes, expose the musician to songs with desirable features as they explore how to create their own song. The musician's insights can inform and guide development, but again, at the discretion of the development team.}}
    \end{itemize}
    
    \item \textcolor{red}{\sout{Music Theorists}}
    \begin{itemize}
      \item \textcolor{red}{\sout{\emph{Target subject matter knowledge:} description of current process and/or approach used to generate novel ideas for composing new songs or experimenting with different ideas for arranging existing songs. \emph{Yet again, we must be cautious so as to only consider the experimentation component of the theorist's workflow, not the composition or arrangement process in its core.}}}
      \item \textcolor{red}{\sout{\emph{Extent of project involvement:} minimal, i.e., no more than three interviews per theorist -- \emph{though, we currently do not have any candidate music theorists.}}}
      \item \textcolor{red}{\sout{\emph{Influence on project:} moderate-low -- music theorists may already have a process into which \emph{GenreGuru} can be integrated, similar to producers, i.e., song featurization can quickly, and in large volumes, summarize music from a catalogue of songs of interest to identify similarities and differences in their sound properties based on their composition and arrangement. At the discretion of the development team, the music theorist's insights can inform and guide development geared for very musically literate users.}}
    \end{itemize}
    
    \item \textcolor{red}{\sout{Music Educators}}
    \begin{itemize}
      \item \textcolor{red}{\sout{\emph{Target subject matter knowledge:} description of current process and/or approach used to introduce students to novel music concepts through experimentation or experimenting with different ideas for previously-learned (composite) concepts. \emph{We must be cautious so as to only consider the experimentation component of the teacher's workflow, not the teaching practice or philosophy in its core.}}}
      \item \textcolor{red}{\sout{\emph{Extent of project involvement:} minimal, i.e., no more than three interviews per teacher -- \emph{though, we currently do not have any candidate music teachers.}}}
      \item \textcolor{red}{\sout{\emph{Influence on project:} low -- music teachers may already have a process into which \emph{GenreGuru} can be integrated, i.e., song generation can (relatively) quickly, and in (relatively) large volumes, produce sound artifacts that introduce novel music concepts or demonstrate alternative use of one or more previously-learned concepts. Like other stakeholders, at the discretion of the development team, the music teacher's insights can inform and guide development geared for \emph{shared} music experimentation settings.}}
    \end{itemize}
    
  \end{itemize}
  
  \item \textcolor{red}{\sout{Affiliated corporation staff -- \emph{out of scope}}}
  \begin{itemize}
    \item \textcolor{red}{\sout{Label staff -- \emph{publishers, marketers, lawyers, \& executives}}}
    \item \textcolor{red}{\sout{Production studio staff -- \emph{studio managers, instrument maintainers, \& sound designers}}}
  \end{itemize}
  
  \item \textcolor{red}{\sout{Development team -- \emph{exclusively involves team members, so out of scope}.}}
  \item \textcolor{red}{\sout{Maintenance team -- \emph{exclusively involves team members, so out of scope}.}}
  \item \textcolor{red}{\sout{Music regulators -- \emph{song licensing laws to abide by when acquiring training data is the only applicable concern, otherwise out of scope. For the interim, API documentation and metadata dictionaries suffice as a resource.}}}
\end{itemize}

\begin{itemize}
  \item \textbf{Subject Matter Experts (SMEs)}\\
    Experts in music and audio analysis who can offer insights into effective recommendation practices while ensuring stakeholder privacy. For a commercial project, compliance with data protection regulations (such as PIPEDA) would be required.
    
  \item \textbf{Music Producers \& Sound Engineers}\\
    These stakeholders can use recommendations to quickly identify songs with similar production qualities or sonic characteristics, aiding in benchmarking and creative decision-making. Their involvement will be minimal (e.g., up to three interviews) but valuable for informing system integration into professional workflows.
    
  \item \textbf{Musicians}\\
    Musicians can leverage the recommendation system to explore music that aligns with their style or inspires new creative directions. As with producers, their feedback (via a few interviews) will help refine the system’s usability and relevance.
    
  \item \textbf{Music Theorists}\\
    Although their direct involvement is expected to be minimal (e.g., up to three interviews), theorists can provide insights into how the system might reveal relationships among songs based on shared musical structures or characteristics.
    
  \item \textbf{Music Educators}\\
    Educators can utilize the recommendation system to expose students to a broader range of music with similar attributes, enhancing learning and creative exploration. Their feedback will help tailor the system for educational purposes.
    
  \item \textbf{Affiliated Corporate and Label Staff}\\
    These groups (including publishers, marketers, and studio managers) are acknowledged as potential external stakeholders but are considered out of scope for the current academic project.
    
  \item \textbf{Development and Maintenance Teams}\\
    Both teams consist exclusively of project members and are considered internal; therefore, they are not detailed further in this section.
    
  \item \textbf{Music Regulators}\\
    Regulatory bodies concerned with song licensing and data usage provide context for compliance when acquiring training data, though these aspects are managed via API documentation and metadata dictionaries.
\end{itemize}


\subsection{Hands-On Users of the Project} \label{2.4-hands-on-users}
The first four stakeholders of section \ref{2.3-other-stakeholders} are the users of concern. To maximize project reach, we do not distinguish between separate user groups with regards to some characteristics, i.e., experience level in the subject matter or technology, attitude toward technology, and physical location. A user can be any combination of: a beginner, novice, intermediate, advanced, or expert in the subject matter or technology, they may be timid to use technology or a technology fanatic, and they can be located anywhere that is within reach of our service area. What varies between user groups are their relevant responsibilities, outlined below.
\begin{itemize}
  \item \textcolor{red}{\sout{Music Producers \& Sound Engineers -- Edit, mix, and master live \& recorded audio; facilitate experimentation with instruments, audio effects, and lyrics.}}  
  \item \textcolor{red}{\sout{Musicians -- Play instruments and/or sing in live \& recorded settings; experiment with instruments and vocals.}}
  \item \textcolor{red}{\sout{Music Theorists -- Compose new pieces of music; arrange existing music compositions.}}  
  \item \textcolor{red}{\sout{Music Educators -- Conduct personal and group instruction sessions to present novel music concepts; reintroduce previously-learned music concepts used in a novel setting; present combinations of previously-learned music concepts.}}
\end{itemize}

\begin{itemize}
  \item \textcolor{red}{\textbf{Music Producers \& Sound Engineers:} Professionals who leverage the recommendation system to quickly identify and integrate similar production techniques and sonic characteristics into their work.}
  \item \textcolor{red}{\textbf{Musicians:} Artists who utilize the system as a creative tool to discover new music that resonates with their style and inspires fresh ideas.}
  \item \textcolor{red}{\textbf{Music Theorists:} Scholars and analysts who explore musical relationships by examining shared structures and patterns between songs.}
  \item \textcolor{red}{\textbf{Music Educators:} Instructors who incorporate the recommendation system to expose students to a diverse range of music, thereby fostering creative learning and experimentation.}
\end{itemize}
\subsection{Personas} \label{2.5-personas}
\begin{itemize}
  \item Music Producers \& Sound Engineers
  \begin{itemize}
    \item Fictitious name -- \emph{Brianna Barboza}
    \item Fictitious age -- \emph{31}
    \item Relevant job -- \emph{Accountant}
    \item Relevant hobbies -- \emph{Disc jockeying}
    \item Relevant music genres -- \emph{pop \& hip-hop}
    \item \textcolor{red}{\sout{Relevant likes/dislikes -- TBD after interviews}}
    \item \textcolor{red}{Relevant likes/dislikes -- Like tracks that are energetic, well-produced, and dynamic, but dislikes music that is unpolished.}
    \item Technology attitude -- \emph{comfortable using digital tools, but prefers analog when it comes to audio.}
  \end{itemize}
  \item Musicians
  \begin{itemize}
    \item Fictitious name -- \emph{Luis Braga}
    \item Fictitious age -- \emph{24}
    \item Relevant job -- \emph{N/A, studying for a MSc in Chemistry and Biochemistry from UWaterloo}
    \item Relevant hobbies -- \emph{Breakdancing}
    \item Relevant music genres -- \emph{Samba \& Bossa Nova}
    \item \textcolor{red}{\sout{Relevant likes/dislikes -- TBD after interviews}}
    \item \textcolor{red}{Relevant likes/dislikes -- Enjoys creative, inspiring tunes with a unique vibe, and avoids repetitive or generic music.}
    \item Technology attitude --  \emph{very proficient, he grew up spending his free time in an internet café before starting university.}
  \end{itemize}
  \item Music Theorists
  \begin{itemize}
    \item Fictitious name -- \emph{Goran Kodeski}
    \item Fictitious age -- \emph{31}
    \item Relevant job -- \emph{Consulting}
    \item Relevant hobbies -- \emph{Collecting LP vinyl records}
    \item Relevant music genres -- \emph{Folk \& Jazz}
    \item \textcolor{red}{\sout{Relevant likes/dislikes -- TBD after interviews}}
    \item \textcolor{red}{Relevant likes/dislikes -- Appreciates music with complex structures that challenge my thinking. Stays away from overly simplistic, commercial tracks.}
    \item Technology attitude --  \emph{vehemently anti-digital, owns a flip-phone without a SIM card, and only uses VoIP.}
  \end{itemize}
  \item Music Educators
  \begin{itemize}
    \item Fictitious name -- \emph{Tumanako "Tui" Teka}
    \item Fictitious age -- \emph{44}
    \item Relevant job -- \emph{Music teacher}
    \item Relevant hobbies -- \emph{Swimming in Lake Waikaremoana}
    \item Relevant music genres -- \emph{Pūoro Māori}
    \item \textcolor{red}{\sout{Relevant likes/dislikes -- TBD after interviews}}
    \item \textcolor{red}{Relevant likes/dislikes -- Likes culturally rich music, avoids anything that offers little in terms of teaching opportunities.}
    \item Technology attitude --  \emph{complete beginner, and he only ever goes to the studio to record something he's performed a few times prior.}
  \end{itemize}
\end{itemize}
\subsection{Priorities Assigned to Users}
This section builds on \ref{2.4-hands-on-users}, appointing \emph{music producers} \& \emph{musicians} key users, then music theorists \& music educators secondary users. \textcolor{red}{\sout{These priorities may change as interviews are conducted and different user groups become more concrete.}}
\subsection{User Participation}
Further extending \ref{2.4-hands-on-users}, all users will be notified that they will be involved in no more than 3 interviews as mentioned in the extent of project involvement in \ref{2.3-other-stakeholders}. Should a user be willing to further contribute to the project after three interviews, they will be contacted as sparingly or generously as they outline. Asynchronous communication via e-mails and text are unrestricted, but expected to be within reason and not to cause a disturbance to its recipient.
\subsection{Maintenance Users and Service Technicians}
The maintenance team exclusively involves the team members, thus is considered out of scope and will not be explored in detail.

\section{Mandated Constraints}
\subsection{Solution Constraints}
\begin{itemize}

  \item \textcolor{red}{\sout{The service uses a music dataset}}
  \\ \textcolor{red}{\sout{\textbf{Rationale:} A dataset for an AI project is necessary as some form of training data must be used 
  in order to train the AI generative, analysis and recommendation systems. }}
  \\ \textcolor{red}{\sout{\textbf{Fit Criterion:} The machine learning algorithms use a music dataset as the training set. }}

  \item \textcolor{red}{\sout{The service uses a machine learning algorithm in order to generate song recommendations and snippets. }}
  \\ \textcolor{red}{\sout{\textbf{Rationale:} The general gist of this project is a leveraging of different signal processing and machine learning
  algorithms in order to provide an end user with a better experience for consuming music. We believe that using a machine 
  learning algorithm for these ends would both be interesting in terms of implementing and training a model, but also practically
  useful for the end user to provide better recommendations by leveraging training over a very large dataset in order to produce results. }}
  \\\textcolor{red}{\sout{\textbf{Fit Criterion:} The recommendation and generation components utilize trained machine learning algorithm. }}

  \item \textcolor{red}{\sout{The service features integration with an existing music streaming provider's platform. }}
  \\ \textcolor{red}{\sout{\textbf{Rationale:} A music service provider (such as Spotify) would allow the service to bypass the need to have
  music inputted, instead being able to use references to a track. In addition, through API calls, providers such as spotify
  already have a large amount of useful labels for invidiual track(s) that can be used as features for the machine learning
  components of the service.  }}
  \\ \textcolor{red}{\sout{\textbf{Fit Criterion:} The service has components that make uses of features (such as API calls) 
  that belong to a music service provider's platform. }}

  \item \textcolor{red}{\sout{The service uses a server to process the user requests separately from the front end.}}
  \\ \textcolor{red}{\sout{\textbf{Rationale:} Ideally, our service would use an on-premise deployed ubuntu server in order to process the user 
  analysis, recommendation, and generation components of the project, as this would allow a more flexible front end design
  (such as through a web application). }}
  \\ \textcolor{red}{\sout{\textbf{Fit Criterion:} An on-premise server is deployed for the purposes of handling the analysis, generation and recommendations
  systems separately from the front end. }}


  \textcolor{red}{\textbf{Technology Stack}} \\
  \\ \textcolor{red}{\textbf{Description:} The system shall be developed using a high-level programming language for efficient core processing 
  and a robust, secure language for backend logic, while the frontend will utilize modern web technologies (e.g., React and CSS) 
  for a responsive, intuitive interface.}
  \\ \textcolor{red}{\textbf{Rationale:} A high-level language on the backend enables rapid development and efficient handling of critical 
  processing tasks, and modern web technologies on the frontend ensure the interface is user-friendly, maintainable, and secure.}
  \\ \textcolor{red}{\textbf{Fit Criterion:} The final product shall meet established performance benchmarks for processing tasks and deliver a responsive, 
  secure user interface across supported devices.}
  \\ \\

  \textcolor{red}{\textbf{Open-Source Libraries}} \\
  \\ \textcolor{red}{\textbf{Description:} The app shall use open-source libraries and frameworks where applicable to save both time and money.}
  \\ \textcolor{red}{\textbf{Rationale:} Utilizing open-source libraries is cost-effective and leverages community-driven improvements and support.}
  \\ \textcolor{red}{\textbf{Fit Criterion:} All third-party libraries used in the project must be licensed under recognized open-source agreements
  (e.g., MIT, GPL, or Apache) to ensure they are free to use.}
  \\ \\

  \textcolor{red}{\textbf{Integration with a Music Streaming Provider’s Platform}} \\
  \\ \textcolor{red}{\textbf{Description:} The service shall integrate with an existing music streaming provider’s platform.}
  \\ \textcolor{red}{\textbf{Rationale:} By integrating with a provider (such as Spotify, or Deezer), the service can use track references and 
  leverage existing metadata—eliminating the need to manually input music—and take advantage of the rich labels that 
  the provider supplies.}
  \\ \textcolor{red}{\textbf{Fit Criterion:} The service must include components that successfully make API calls to a music streaming 
  provider’s platform to retrieve track features.}
  \\ \\

  \textcolor{red}{\textbf{Server-Based Processing of User Requests}} \\
  \\ \textcolor{red}{\textbf{Description:} The service shall utilize a dedicated server to process user requests separately from the front end.}
  \\ \textcolor{red}{\textbf{Rationale:} Deploying an on-premise Ubuntu server for backend processing allows for flexible front-end design 
  (e.g., via a web application) while efficiently handling analysis, recommendations, and other processing tasks.}
  \\ \textcolor{red}{\textbf{Fit Criterion:} An on-premise server must be deployed to handle the backend processing, ensuring that 
  recommendation functions operate independently from the user interface.}
  \\ \\
\end{itemize}

\subsection{Implementation Environment of the Current System}
\includegraphics[scale=0.552]{3_2_environment_diagram.pdf}

\subsection{Partner or Collaborative Applications}
\includegraphics[scale=0.72]{3_3_partner_constraints_figure}

\subsection{Off-the-Shelf Software}
There are several existing solutions that could serve as part of the music generation and recommendation system. These include:

\begin{itemize}
    \item \textbf{Spotify API}: Provides access to a vast library of music, including song previews and metadata, which can be leveraged for generating recommendations.
    \item \textbf{Librosa Library}: An open-source Python package for analyzing and processing music files, suitable for extracting features from songs and facilitating generative components.
    \item \textbf{TensorFlow and PyTorch Pre-trained Models}: Both frameworks offer pre-trained models that could be adapted for music generation tasks. These solutions provide a basis for deep learning models without having to build and train from scratch.
    \item \textbf{OpenAI Jukebox}: A generative model that is capable of producing music, which could potentially be adapted and integrated into our system.
\end{itemize}

These off-the-shelf software solutions provide a foundation upon which we can build our custom features, significantly reducing the development time and leveraging existing technologies to enhance the functionality of our platform.

\subsection{Anticipated Workplace Environment}
\begin{itemize}
  \item Home Usage
  \\\textbf{Rationale:} This environment is more appropriate for casual music listeners. 
  This means users will most likely be using it in a form of a casual environment. This can 
  include such as during a public commute on the train, where the user might try to interact 
  with the service while there is an unreliable internet connection on a mobile device. This
  means the front end of the service must be easy to use and access on all types of devices, 
  and preferrably, not a large amount of data is streamed between the user and the service. 
  This means that preferrably, the input would have references to pieces of music instead of 
  having say, full .MP3 files as the input. These users are also more likely interested in the 
  recommendation system to attempt to find new pieces of music to listen to, and might be more
  willing to use the generation system for "fun", so they might be satifisfied even with a 
  generation component that produces odd results.
 
  \item Studio Usage
  \\\textbf{Rationale:} This environment is more professional and what is most likely what 
  creative professionals will be using. The expectation is that they will be using this on their 
  computers, thus they are more likely have a reliable internet connection in addition to being
  willing to input a lot more music to the service. This means that the service needs to faciliate 
  entering a large amount of music, and be able to generate results in a rapid and efficient 
  manner. Because the users of this type of environment would likely be inviduals engaged in some form
  of creative endeavors, they will generally have higher standards of what the service provides, and
  might be more interested in the analysis component, such as being able to find new labels for their
  existing work to get new perspectives. This also means that they most likely expect the music generation
  component to give them something more "listenable", something they could directly use as a sample or inspiration.  

\end{itemize}

\subsection{Schedule Constraints}
This project was started in the 2024 Fall Term and is expected to be completed by the 
2025 winter term at McMaster university. Some of the key deadlines are: 
\begin{itemize}
  \item Proof of Concept Demonstration Plan, November 11-22
  \\This deadline accounts for 5\% of the mark. The group should have identified the most significant
  risks involving the project and come up with plans on how to migitate said risks. If this is not possible, then the project 
  can be redefined. The group also needs to predict and note other concerns about potential problems or difficulties that could arise
  during development, such as testing difficulties or software portability. If this deadline is not met, then it means that our group
   does not fully understand the potential pitfalls of our project and we need to redefine certain aspects of the project. 

  \item Final Demonstration, March 24-30
  \\This deadline accounts for 20\% of the mark. This deadline involves a demonstration of a finalized 
  version of the project to supervisors before the public EXPO happening at a TBD date. By this deadline, 
  the service should be in a completed state ready for use and demonstration. If this deadline is not met then
  the project can be considered a "failure". 

  \item Final Documentation, April 2
  \\This deadline accounts for 30\% of the mark. It involves the finalized documentation of plans pertaining to 
  the project and the actual working program/service. Any final revisions and reflections pertaining to the project
  should have been completed by this deadline as no further changes to the project will be possible. Whatever documentation
  or code that is not completed by this deadline would be considered permanently unfinished. 
\end{itemize}

\subsection{Budget Constraints}
The budget limit as stated by the capstone outline is 750\$ CAD. Potential additional
costs might include API calls, software liscensing, account fees for cloud services. 
For the purposes of the demonstration they should not exceed the 750\$ limit. 

\subsection{Enterprise Constraints}
There are no specific enterprise constraints as we do not have outside investors
for this project. 

\section{Naming Conventions and Terminology}
\subsection{Glossary of All Terms, Including Acronyms, Used by Stakeholders
involved in the Project}
\begin{itemize}
  \item Track is full song, i.e., a completed audio file published by a musician. 
  \item A snippet is a fragment of a track, i.e., an incomplete audio file. Such audio files can be in a \emph{lossy} compressed format, e.g., 
  \texttt{.mp3}, a \emph{lossless} compressed format, e.g., \texttt{.FLAC}, or an uncompressed format, e.g., \texttt{.WAV}. While the file formats of audio files are known, 
  the source of a song or song snippet can be anything. A concrete example is a 30-second snippet from Spotify's API being an MP3 file delivered through the Spotify content 
  delivery network, \texttt{scdn}, e.g., \href{https://p.scdn.co/mp3-preview/c05a687254dbdf50a9ab4879d85e54a7594e367f?cid=945b9ed74f4c43b49bcf26e7dc388df0}{Jack Harlow's ``First Class''}.
  \item Model: Statistical Model (Not software module)
  \item Module: Software Component
  \item AI: Artificial Intelligence
  \item ML: Machine Learning Model
  \item Music Metadata: labels obtained through streaming music API (eg spotify) that are
  like features (eg genre, 'danceability')
\end{itemize}

\section{Relevant Facts And Assumptions}
\subsection{Relevant Facts}
\begin{itemize}
  \item Music contains the following core features \begin{itemize}
    \item Tempo
    \item Key signature
    \item Time signature
    \item Pitch
    \item Timbre
  \end{itemize}
  \item Song files have metadata that contains information such as: \begin{itemize}
    \item Song title
    \item Artist
    \item Release date
  \end{itemize}
  \item Most songs can be classified into multiple genres
\end{itemize}
\subsection{Business Rules}
\begin{itemize}
  \item The user should be able to generate their own music
  \item The user should be able to figure out what musical features a song contains 
  \item The user should be able to ask for similar songs
  \item the user should be able to interact with the system without any external installation
\end{itemize}
\subsection{Assumptions}
\begin{itemize}
  \item Users will have at least some familiarity of music theory
  \item The analysis and recommendation systems will use as many well-established musical 
  features as possible
  \item All API inputs will be easily accessible and reliable enough to support the
  recommendation and analysis systems
  \item The system will be written in a language that all developers are familiar with
  \item The system will use a local server to handle the processing of
  the machine learning model and large datasets
  \item Handling of niche features and cover art are designed to enhance the user experience, but 
  these will not be a part of the core functionality of the system
  \item The generative system will be completed by the POC demo date
  \item The recommendation and analysis systems will be completed by the Revision 0 date
\end{itemize}

\section{The Scope of the Work}
\subsection{The Current Situation}
\begin{itemize}
  \item \textbf{Music Analysis Component}
  \\\textbf{Manual Process}: Currently, this is usually manually done by the artist self-labelling their work
  or third parties labelling the music. This can end up being time consuming. 
  \\\textbf{Automated Process}: Most automation on this regard (retrieving song information from a music 
  provider's api) is mostly useful for retreiving already labelled information, there is not a lot of automation 
  that is be done for labelling, but a lot that is possible retreiving said labels. 
  \\What we wish to do with this component of the service is to gather and present said information that can be
  gleamed from the music service provider's API to the user in a more presentable manner (such as audio visualizations)
   but also featurize an individual song for the other components in our service. Our service should also generate new
   labels using ML algorithms, which allows for faster automation of song labelling and more features for our other 
   components. 
  \\\includegraphics[width=\textwidth]{6_1_b.pdf}
  \item \textbf{Music Recommendation Component}
  \\\textbf{Manual Process}: Generally, this is mostly done through music curators should as DJs, playlist makers,
  etc, where curators manually select songs and make recommendations based on popular trends, new artists releases,
  or shifts in the genre. The main issue here is that it requires a large amount of manual work, as curators need to 
  listen to new music before adding them to some form of playlist or recommending them to other people, which greatly
  limits their potential recommendations. 
  \\\textbf{Automated Process}: Most music service providers such as spotify, apple music or youtube already offer 
  some form of music recommendation service use algorithms to recommend users new songs based on their listening history.
  These generally already work decently well, but they end up usually being heavily weighted towards more popular songs 
  (for example, sabrina carpenter's espresso being automatically recommended through smart shuffle no matter what your prior listening history was like), 
  as record labels often have deals with these music service platforms so it is not unusual to get recommendations that 
  do not necessarily suit the user's music tastes. 
  \\The main benefit our service should provide is the ability to recommend songs that are more niche, more accurate based
  on the user input while still using an automated system powered by machine learning, as we would have more features due 
  to the analysis component for the ML algorithm to process, in addition to not being heavily weighted towards more popular,
  trendy songs or some form of likely record label deals meddling. 
  \\\includegraphics[width=\textwidth]{6_1_a.pdf}

  \item \textbf{Music Generation Component}
  \\\textbf{Manual Process}: Currently, generating music usually requires years of dedication to the craft and involves the
  labour of musicians, producers, and composers. This means that there is a large amount of manpower that is necessary
  in order to create new unique pieces. 
  
  We wish to have our service provide a new avenue to generate new types of music. We do not envision this service as 
  something that replaces the traditional music creation process, but rather as something that can potentially enhance it. 

  \includegraphics[width=\textwidth]{6_1_c.pdf}

\end{itemize}

\subsection{The Context of the Work}
\includegraphics[width=\textwidth]{6_2_context_diagram.pdf}

\subsection{Work Partitioning}
\includegraphics[width=\textwidth]{6_3_work_partitioning_table.pdf}

\subsection{Specifying a Business Use Case (BUC)}
\begin{itemize}
  \item \textbf{User Requests Track Recommendations}
    \\Primary Actor: App User
    \\Trigger: User intiates a request with the service to generate some new songs to listen to. 
    \\Preconditions: 
    \begin{itemize}
    \item User has valid references to tracks for the service to accept
    \item ML generation components are currently active
    \end{itemize}

    Main Success Scenario: 
    \begin{itemize}
    \item user submits request for track recommendation
    \item user inputs track(s) to the service
    \item the service's analyzation component analyzes the tracks
    \item analyzation component passes on tracks \& analysis to recommendation component
    \item recommendation component generates tracks
    \item user receives new tracks to listen to
    \end{itemize}
  
  \item \textbf{User Requests Track Analysis}
    \\Primary Actor: App User
    \\Trigger: User intiates a request with the service to generate some new songs to listen to. 
    \\Preconditions: 
    \begin{itemize}
    \item User has valid references to tracks for the service to accept
    \item ML generation components are currently active
    \end{itemize}

    Main Success Scenario: 
    \begin{itemize}
    \item user submits request for track recommendation
    \item user inputs track(s) to the service
    \item the service's analyzation component analyzes the tracks
    \item analyzation component generates presentations and visualizations based on the inputted track(s)
    \item the service returns generated presentations and visualizations to the user
    \end{itemize}
  
  \item \textbf{User Requests Snippet Generation}
    \\Primary Actor: App User
    \\Trigger: User intiates a request with the service to generate some new songs to listen to. 
    \\Preconditions: 
    \begin{itemize}
    \item User has valid references to tracks for the service to accept
    \item ML generation components are currently active
    \end{itemize}

    Main Success Scenario: 
    \begin{itemize}
    \item user submits request for snippet generation
    \item user inputs track(s) to the service
    \item the service's analyzation component analyzes the tracks
    \item analyzation component passes on tracks \& analysis to generation component
    \item generation component generates tracks
    \item user receives a snippet to listen to
    \end{itemize}
\end{itemize}



\section{Business Data Model and Data Dictionary}
\subsection{Business Data Model}
\includegraphics[width=\textwidth]{7_1_business_data_model.png}
\subsection{Data Dictionary}
\subsubsection{Relations}
Please refer to table \ref{tbl:rel-data-dict}.
\begin{table}[h!]
  \centering
  \begin{tabular}{ p{.25\linewidth} || p{.65\linewidth} }
    \textbf{Relation Name} & \textbf{Relation Description} \\
    \toprule
    \emph{Track} & Primary relation containing track name, track length, \& references to other relations. \\
    \midrule
    \emph{TrackArtists} & Intermediate relation to retain information about which artists appear on which tracks. \\
    \midrule
    \emph{TracksAlbum} & Intermediate relation to retain information about which tracks appear on which albums, if applicable. \\
    \midrule
    \emph{TrackSource} & Secondary relation containing information about the source of a track, e.g., Spotify, Deezer, Amazon music, BandCamp, or local. \\
    \midrule
    \emph{TrackFeatures} & Secondary relation contianing information about the features of a track, following the \href{https://en.wikipedia.org/wiki/Entity%E2%80%93attribute%E2%80%93value_model}{entity-attribute-value (EAV) model}. \\
    \midrule
    \emph{TrackCover} & Abstract relation containing information about the cover art of the track. \\
    \midrule
    \emph{Artist} & Secondary relation containing information about artists. \\
    \midrule
    \emph{Album} & Secondary relation containing information about albums. \\
    \midrule
    \textcolor{red}{Audio Access System} & Abstract storage system containing information about the track audio. \\
  \end{tabular}
  \label{tbl:rel-data-dict}
  \caption{Relations Data Dictionary}
\end{table}
\subsubsection{Attributes}
Please refer to table \ref{tbl:attr-data-dict}.
\begin{table}[h!]
  \centering
  \begin{tabular}{ p{.35\linewidth} || p{.55\linewidth} }
    \textbf{Attribute Name} & \textbf{Attribute Description} \\
    \toprule
    \textbf{track\_id} & Unique track identifier. \\
    \midrule
    \textbf{artist\_id} & Unique artist identifier. \\
    \midrule
    \textbf{album\_id} & Unique album identifier. \\
    \midrule
    \textbf{audio\_id} & Unique track audio reference identifier. \\
    \midrule
    \textbf{track\_source\_id} & Unique track source reference identifier. \\
    \midrule
    \textbf{track\_features\_id} & Unique track features reference identifier. \\
    \midrule
    \textbf{track\_cover\_id} & Unique track cover art reference identifier. \\
    \midrule
    \emph{Track}.name & Track name. \\
    \midrule
    \emph{Track}.length & Track length. \\
    \midrule
    \emph{Artist}.name & Artist name. \\
    \midrule
    \emph{Album}.title & Album title \\
    \midrule
    \emph{TrackSource}.track\_source & Track source name. \\
    \midrule
    \emph{TrackSource}.track\_loc\_type & Track source location type, e.g., ``URL'' or ``path''. \\
    \midrule
    \emph{TrackSource}.track\_loc & Track location, i.e., a URL or path. \\
    \midrule
    \emph{TrackFeatures}.feature\_name & Track feature name, e.g., popularity. \\
    \midrule
    \emph{TrackFeatures}.feature\_value & Track feature corresponding value, e.g., 88. \\
    \midrule
    \emph{TrackCover}.cover\_loc\_type & Track cover art location type, e.g., ``URL'' or ``path''. \\
    \midrule
    \emph{TrackCover}.cover\_loc & Track cover art location, i.e., a URL or path.
  \end{tabular}
  \label{tbl:attr-data-dict}
  \caption{Attributes Data Dictionary}
\end{table}
\subsubsection{Relationships}
Please refer to table \ref{tbl:relsh-data-dict}. Please consult \href{https://en.wikipedia.org/wiki/Entity%E2%80%93relationship_model#Crow's_foot_notation}{Entity Relationship Model - Crow's Foot Notation} as a notation reference if necessary - it is mostly useful for the connector multiplicities.
\begin{table}[htbp]
  \centering
  \begin{tabular}{ p{.175\linewidth} || p{.175\linewidth} || p{.55\linewidth} }
    \textbf{Relation 1} & \textbf{Relation 2} & \textbf{Relationship Description} \\
    \toprule
    \emph{Track} & \emph{TrackArtists} & A track can correspond to multiple artists, and vice versa. \\
    \emph{Track} & \emph{TracksAlbum} & A track can correspond to multiple albums (a single and a long-playing (LP) album), and vice versa. \\
    \emph{Track} & \textcolor{red}{Audio Access System} & Multiple tracks can correspond to a single instance of an audio file, i.e., a single and an LP release of a song point to the same audio file. \\
    \emph{Track} & \emph{TrackSource} & A track should only belong to one and only one source to avoid duplication. \\
    \emph{Track} & \emph{TrackFeatures} & A track can correspond to 0 or more features, i.e., a song with no features has not been featurized and a song with multiple features has. This allows modular and evolvable data storage. \\
    \emph{Track} & \emph{TrackCover} & A track can correspond to at most one cover, i.e., it can have no cover art. The cover art of an album can also apply to its constitutent tracks. \\
    \emph{TrackArtists} & \emph{Artist} & A mapping relationship, where an artist can exist on multiple tracks. \\
    \emph{TracksAlbum} & \emph{Album} & A mapping relationship, where an album can contain multiple tracks. \\
    \emph{Artist} & \emph{Album} & An artist can have 0 or more albums, but an album must have at least one artist. 
  \end{tabular}
  \label{tbl:relsh-data-dict}
  \caption{Relationships Data Dictionary}
\end{table}

\section{The Scope of the Product}
\subsection{Product Boundary}
\includegraphics[width=\textwidth]{product_boundary.png} \\
\subsection{Product Use Case Table}
\begin{table}[!ht]
    \centering
    \resizebox{\textwidth}{!}{
    \begin{tabular}{|l|l|l|l|}
    \hline
        \textbf{PUC No} & \textbf{PUC Name} & \textbf{Actor/s} & \textbf{Input \& Output} \\ \hline
        1 & UI Interaction & User & User Actions (click, swipe, drag) (in) \newline System Response (out) \\ \hline
        2 & Song Recommendation Based on Features & User & User's desired features (in) \newline Collection of reference songs (out) \\ \hline
        3 & Music Generation Based on Input & User & Reference song(s) and/or song snippet(s) (in) \newline Generated song or song snippet (out) \\ \hline
        4 & Analyze Music & User & Reference song or song snippet (in) \newline Collection of features and visualizations (out) \\ \hline
        5 & Song Recommendation Based on Input & User & Reference song(s) (in) \newline Collection of reference songs (out) \\ \hline
        6 & Server Interaction for Music Generation & Server & Reference song(s) and/or song snippet(s) (in) \newline Generated song or song snippet (out) \\ \hline
        7 & Server Interaction for Song Recommendation & Server & User's desired features or reference song(s) and/or snippet(s) (in) \newline Collection of reference songs (out) \\ \hline
        8 & Server Interaction for Music Analysis & Server & Reference song or song snippet (in) \newline Collection of features and visualizations (out) \\ \hline
    \end{tabular}
    }
    \caption{Product Use Case Table}
    \label{tab:PUCtable}
\end{table}

\subsection{Individual Product Use Cases (PUC's)}
\textbf{1. Product Use Case Name:} UI Interaction \\
\textbf{Trigger:} User commits some action (e.g. clicking, swiping, dragging) \\
\textbf{Preconditions:} User has successfully accessed GenreGuru, or is already in GenreGuru \\
\textbf{Interested Stakeholders:} All \\
\textbf{Actor/s:} User \\
\textbf{Activity Diagram:} \\
\includegraphics[width=\textwidth]{UI.png} \\
\textbf{Outcome:} The user will commit an action like swiping or pressing and the system will react depending on the action.

\vspace{1cm}

\noindent \textbf{2. Product Use Case Name:} Song Recommendation Based on Features \\
\textbf{Trigger:} User picks features, and indicates they want to search for recommendations \\
\textbf{Preconditions:} User must have GenreGuru open, the user has selected features to search for \\
\textbf{Interested Stakeholders:} Casual Music Listeners, Hobbyist Musicians \\
\textbf{Actor/s:} User \\
\textbf{Activity Diagram:} \\
\includegraphics[width=\textwidth]{recommendation_feature.png} \\
\textbf{Outcome:} The user will select or manually enter features they are looking for in a song, and the system will first check to see if the features they selected/entered are valid, and the system will return a collection of reference songs that match those features.

\vspace{1cm}

\noindent \textbf{3. Product Use Case Name:} Music Generation Based on Input \\
\textbf{Trigger:} User inputs reference song(s) and/or song snippet(s), and indicates they want to generate a song \\
\textbf{Preconditions:} User must have GenreGuru open, and the user has provided a valid input(s) \\
\textbf{Interested Stakeholders:} Music producers, Hobbyist Musicians \\
\textbf{Actor/s:} User \\
\textbf{Activity Diagram:} \\
\includegraphics[width=\textwidth]{generate_song.png} \\
\textbf{Outcome:} The user will enter song(s) and/or song snippet(s) and indicate to the system that they want to generate music, the system will check that these inputs are valid (correct format) and then will generate a song and return it to the user.

\vspace{1cm}

\noindent \textbf{4. Product Use Case Name:} Analyze Music \\
\textbf{Trigger:} User inputs a reference song or song snippet and indicates they want to analyze the music \\
\textbf{Preconditions:} User must have GenreGuru open, and the user has provided a valid input \\
\textbf{Interested Stakeholders:} Music Producers, Audio Engineers, Music Educators \\
\textbf{Actor/s:} User \\
\textbf{Activity Diagram:} \\
\includegraphics[width=\textwidth]{analyze_song.png} \\
\textbf{Outcome:} The user will input a reference song or song snippet and indicate they want to analyze the song, the system will validate the input and return a set of features and visualizations.

\vspace{1cm}

\noindent \textbf{5. Product Use Case Name:} Song Recommendation Based on Input \\
\textbf{Trigger:} User inputs reference song(s) and/or snippet(s), and indicates they want to search for recommendations \\
\textbf{Preconditions:} User must have GenreGuru open, and the user has provided a valid input(s) \\
\textbf{Interested Stakeholders:} Casual Music Listeners, Hobbyist Musicians \\
\textbf{Actor/s:} User \\
\textbf{Activity Diagram:} \\
\includegraphics[width=\textwidth]{recommend_song.png} \\
\textbf{Outcome:} The users will input reference song(s) and/or snippet(s), the system will first check to see if the inputs are valid. Then the system will return a collection of reference songs.

\vspace{1cm}

\textbf{6. Product Use Case Name:} Server Interaction for Music Generation \\
\textbf{Trigger:} User submits a reference song and/or snippet and requests music generation \\
\textbf{Preconditions:} User has provided a valid input through, and the server is operational \\
\textbf{Interested Stakeholders:} Music Producers, Hobbyist Musicians \\
\textbf{Actor/s:} Server \\
\textbf{Activity Diagram:} \\
\includegraphics[width=\textwidth]{server_song_gen.png} \\
\textbf{Outcome:} The server processes the input, generates music, and returns the generated song to the user

\vspace{1cm}

\textbf{7. Product Use Case Name:} Server Interaction for Song Recommendation \\
\textbf{Trigger:} User submits desired features or reference songs/snippets and requests song recommendations \\
\textbf{Preconditions:} User has provided valid input, and the server is available \\
\textbf{Interested Stakeholders:} Casual Music Listeners, Hobbyist Musicians \\
\textbf{Actor/s:} Server \\
\textbf{Activity Diagram:} \\
\includegraphics[width=\textwidth]{server_song_rec.png} \\
\textbf{Outcome:} The server processes the input and returns a collection of recommended songs based on the input features or reference songs/snippets.

\vspace{1cm}

\textbf{8. Product Use Case Name:} Server Interaction for Music Analysis \\
\textbf{Trigger:} User submits a reference song or snippet and requests music analysis \\
\textbf{Preconditions:} User has provided a valid input, and the server is ready to analyze \\
\textbf{Interested Stakeholders:} Music Producers, Audio Engineers, Music Educators \\
\textbf{Actor/s:} Server \ \textbf{Activity Diagram:} \\
\includegraphics[width=\textwidth]{server_song_analysis.png} \\
\textbf{Outcome:} The server analyzes the song or snippet and returns a collection of features and visualizations to the user.

\section{Functional Requirements}
\subsection{Functional Requirements}
\hspace{14pt}

\fbox{
\begin{minipage}{\textwidth}
\textbf{Requirement \# 1} \\
\textbf{Requirement Type:} 9 \\
\textbf{Event/Use Case \#:} (need to figure this out later)\\
\textbf{Description:} The client application should allow the user to provide a reference link to a song to initiate a request from the system \\
\textbf{Rationale:} This provides a straightforward way for users to interact with the system and start a process of their choosing \\
\textbf{Originator:} Requirement Analyst \\
\textbf{Fit Criterion:} The user can paste a link, then make a request of the system \\
\textbf{Customer Satisfaction:} 5 \\
\textbf{Customer Dissatisfaction:} 0 \\
\textbf{Priority:} High \\
\textbf{Conflicts:} None \\
\textbf{Supporting Material:} None \\
\textbf{History:} Created November 1, 2024
\end{minipage}
}


\vspace{10pt}

\fbox{
\begin{minipage}{\textwidth}
\textbf{Requirement \# 2} \\
\textbf{Requirement Type:} 9 \\
\textbf{Event/Use Case \#:} (need to figure this out later)\\
\textbf{Description:} The system should establish a connection between the client application, server, and external APIs to handle data requests and responses \\
\textbf{Rationale:} Want to ensure data flow across the system, from client inputs to external API requests \\
\textbf{Originator:} Requirement Analyst \\
\textbf{Fit Criterion:} The client’s request will trigger network communication with various components of the system, which will be confirmed by successful data retrieval and submission. \\
\textbf{Customer Satisfaction:} 5 \\
\textbf{Customer Dissatisfaction:} 0 \\
\textbf{Priority:} High \\
\textbf{Conflicts:} None \\
\textbf{Supporting Material:} None \\
\textbf{History:} Created November 1, 2024
\end{minipage}
}


\vspace{10pt}

\fbox{
\begin{minipage}{\textwidth}
\textbf{Requirement \# 3} \\
\textbf{Requirement Type:} 9 \\
\textbf{Event/Use Case \#:} (need to figure this out later)\\
\textbf{Description:} The system should process feature extraction requests \\
\textbf{Rationale:} Want to provide efficient and reliable processing  \\
\textbf{Originator:} Requirement Analyst \\
\textbf{Fit Criterion:} For each valid request, the server completes feature extraction within a specified time \\
\textbf{Customer Satisfaction:} 5 \\
\textbf{Customer Dissatisfaction:} 0 \\
\textbf{Priority:} High \\
\textbf{Conflicts:} None \\
\textbf{Supporting Material:} None \\
\textbf{History:} Created November 1, 2024
\end{minipage}
}


\vspace{10pt}

\fbox{
\begin{minipage}{\textwidth}
\textbf{Requirement \# 4} \\
\textbf{Requirement Type:} 9 \\
\textbf{Event/Use Case \#:} (need to figure this out later)\\
\textbf{Description:} When a reference link is provided, the system shall use External Service APIs to retrieve song data \\
\textbf{Rationale:} Using external APIs to retrieve song data will reduce input requirements from the user \\
\textbf{Originator:} Requirement Analyst \\
\textbf{Fit Criterion:} A valid reference link triggers data retrieval from the external API \\
\textbf{Customer Satisfaction:} 5 \\
\textbf{Customer Dissatisfaction:} 0 \\
\textbf{Priority:} High \\
\textbf{Conflicts:} None \\
\textbf{Supporting Material:} None \\
\textbf{History:} Created November 1, 2024
\end{minipage}
}


\vspace{10pt}

\fbox{
\begin{minipage}{\textwidth}
\textbf{Requirement \# 5} \\
\textbf{Requirement Type:} 9 \\
\textbf{Event/Use Case \#:} (need to figure this out later)\\
\textbf{Description:} The system should analyze the provided reference song and extract a set of features (e.g. tempo, pitch, rhythm, genre, etc.) \\
\textbf{Rationale:} Users want a way to break down songs for a more detailed analysis \\
\textbf{Originator:} Requirement Analyst \\
\textbf{Fit Criterion:} With every request they system will return a consistent set of features \\
\textbf{Customer Satisfaction:} 5 \\
\textbf{Customer Dissatisfaction:} 0 \\
\textbf{Priority:} High \\
\textbf{Conflicts:} None \\
\textbf{Supporting Material:} None \\
\textbf{History:} Created November 1, 2024
\end{minipage}
}


\vspace{10pt}

\fbox{
\begin{minipage}{\textwidth}
\textbf{Requirement \# 6} \\
\textbf{Requirement Type:} 9 \\
\textbf{Event/Use Case \#:} (need to figure this out later)\\
\textbf{Description:} The system should provide error feedback for each component \\
\textbf{Rationale:} Clear error messaging supports debugging and ensures users receive actionable feedback \\
\textbf{Originator:} Requirement Analyst \\
\textbf{Fit Criterion:} Each component generates specific error messages, guiding users and/or developers \\
\textbf{Customer Satisfaction:} 5 \\
\textbf{Customer Dissatisfaction:} 0 \\
\textbf{Priority:} High \\
\textbf{Conflicts:} None \\
\textbf{Supporting Material:} None \\
\textbf{History:} Created November 1, 2024
\end{minipage}
}


\vspace{10pt}

\fbox{
\begin{minipage}{\textwidth}
\textbf{Requirement \# 7} \\
\textbf{Requirement Type:} 9 \\
\textbf{Event/Use Case \#:} (need to figure this out later)\\
\textbf{Description:} The client application should display the output of system components \\
\textbf{Rationale:} To provide a way of interpreting/visualizing the output to the user \\
\textbf{Originator:} Requirement Analyst \\
\textbf{Fit Criterion:} The client application displays the output for the user to see \\
\textbf{Customer Satisfaction:} 5 \\
\textbf{Customer Dissatisfaction:} 0 \\
\textbf{Priority:} High \\
\textbf{Conflicts:} None \\
\textbf{Supporting Material:} None \\
\textbf{History:} Created November 1, 2024
\end{minipage}
}


\vspace{10pt}

\fbox{
\begin{minipage}{\textwidth}
\textbf{Requirement \# 8} \\
\textbf{Requirement Type:} 9 \\
\textbf{Event/Use Case \#:} (need to figure this out later)\\
\textbf{Description:} The system will validate user inputs to ensure they are correct \\
\textbf{Rationale:} Prevents errors and ensures the system processes valid data. \\
\textbf{Originator:} Requirement Analyst \\
\textbf{Fit Criterion:} The system will display an error message if the input is invalid, or will let the user proceed if the input is valid. \\
\textbf{Customer Satisfaction:} 5 \\
\textbf{Customer Dissatisfaction:} 0 \\
\textbf{Priority:} High \\
\textbf{Conflicts:} None \\
\textbf{Supporting Material:} None \\
\textbf{History:} Created November 1, 2024
\end{minipage}
}

\vspace{10pt}

\fbox{
\begin{minipage}{\textwidth}
\textbf{Requirement \# 9} \\
\textbf{Requirement Type:} 9 \\
\textbf{Event/Use Case \#:} (need to figure this out later)\\
\textbf{Description:} The system should feature a client application for user interaction \\
\textbf{Rationale:} To allow users to interact with the system efficiently and intuitively \\
\textbf{Originator:} Requirement Analyst \\
\textbf{Fit Criterion:} The client application allows users to initiate interactions such as submitting song links, requesting analysis, etc.  \\
\textbf{Customer Satisfaction:} 5 \\
\textbf{Customer Dissatisfaction:} 0 \\
\textbf{Priority:} High \\
\textbf{Conflicts:} None \\
\textbf{Supporting Material:} None \\
\textbf{History:} Created November 1, 2024
\end{minipage}
}


\vspace{10pt}

\section{Look and Feel Requirements}
\subsection{Appearance Requirements}
\textbf{APR1.} The system shall present usable options without clutter. \\
\textbf{APR2.} The system shall be visually accessible by using highly contrasting elements. \\
\textbf{APR3.} The system shall use an easily navigable, professional design.
\subsection{Style Requirements}
\textbf{STR1.} The system shall be responsive to various screen sizes.

\section{Usability and Humanity Requirements}
\subsection{Ease of Use Requirements}
\textbf{EUR1.} The system shall show tooltips pop-ups upon users hovering on an interactable element.
\subsection{Personalization and Internationalization Requirements}
\textbf{PIR1.} The system shall support different display colour themes.
\subsection{Learning Requirements}
\textbf{LR1.} The system shall provide a first-time user guide to use features upon first launch. \\
\textbf{LR2.} The user shall be able to complete the tutorial within \emph{10} minutes.
\subsection{Understandability and Politeness Requirements}
\textbf{UPR1.} The system shall not generate vulgar content.
\textbf{UPR2.} The system shall only use standardized, inoffensive iconography wherever icons are needed.
\subsection{Accessibility Requirements}
\textbf{ACR1.} The system shall only use accessible fonts. \\
\textbf{ACR2.} The system shall offer a color-blind mode.
\section{Performance Requirements}
\subsection{Speed and Latency Requirements}
\emph{N/A -- while there is a server and networking in the system, such requirements are considered out of scope.}
\subsection{Safety-Critical Requirements}
\emph{N/A -- there are no safety-affiliated components affiliated with the project.}
\subsection{Precision or Accuracy Requirements}
\textbf{PAR1.} Query request times are displayed to within 1 minute of precision. \\
\textbf{PAR2.} Displayed percentages will be rounded to the nearest two decimals.
\subsection{Robustness or Fault-Tolerance Requirements}
\textbf{RAR1.} The system shall maintain 97.5\% uptime, i.e., an average downtime of 18 hours every month.
\subsection{Capacity Requirements}
\textbf{CR1.} The system shall handle concurrent usage by up to four users. \\
\textbf{CR2.} The system shall store a cached record of songs and corresponding information.
\subsection{Scalability or Extensibility Requirements}
\emph{N/A -- while there is a server in the system, scaling is not in scope.}
\subsection{Longevity Requirements}
\emph{N/A -- maintenance and longevity of the project is not in scope, at least not until after the POC demonstration.}
\section{Operational and Environmental Requirements}
\subsection{Expected Physical Environment}
\textbf{EPER1.} The system server shall run on a modified Dell OptiPlex 3050. \\
\subsection{Wider Environment Requirements}
\emph{N/A}
\subsection{Requirements for Interfacing with Adjacent Systems}
\emph{N/A}
\subsection{Productization Requirements}
\textbf{PRR1.} The system front-end shall be accessible from a browser across different devices.
\subsection{Release Requirements}
\emph{NA -- the project is academic in nature, so a release cycle is unnecessary.}

\section{Maintainability and Support Requirements}
\subsection{Maintenance Requirements}
\textbf{MR1.} The system shall facilitate switching between music providers overnight.
\subsection{Supportability Requirements}
\emph{NA -- the project is academic in nature, and its lifespan is unlikely to extend such that it needs support.}
\subsection{Adaptability Requirements}
\emph{N/A -- a web-app is the most portable format for the specified target users.}

\section{Security Requirements}
\subsection{Access Requirements}
\textbf{ACR1.} The user shall only be able to access songs they upload or ones that are licensed for their use.
\subsection{Integrity Requirements}
\emph{\textbf{IR1.} The server shall perform a weekly backup of its database to prevent data loss in the event of a catastrophic failure.} \\
\emph{\textbf{IR2.} The server shall implement deduplication measures to guarantee no data redundancy.}
\subsection{Privacy Requirements}
\textbf{PR1.} The system shall only expose query requests to the user that made them.
\subsection{Audit Requirements}
\textbf{AUR1.} The system shall maintain a history of user query requests.
\subsection{Immunity Requirements}
\emph{N/A -- again, the project is academic in nature, and its purpose is not to be robust for attacks on a commercial software.}
\section{Cultural Requirements}
\subsection{Cultural Requirements}
\textbf{CUR1.} The system shall not contain any (potentially) culturally offensive iconography or language.
\section{Compliance Requirements}
\subsection{Legal Requirements}
\textbf{LGR1.} The system shall not make use of any copyrighted material without express permission. \\
\textbf{LGR2.} The data collection process must obey all potential API developer rules.
\subsection{Standards Compliance Requirements}
\emph{N/A -- out of project scope.}

\section{Open Issues}
\begin{itemize}
  \item Generated output use regulation, \#1
  \\\textbf{Summary:} The service EULA agreement might need to consider how and where the generated snippet by the service is allowed
  to be used. 
  
  This is important because the training data or inputted track(s) into the service
  might have copyright laws \& rules regulating commerical use associated with them, thus we need to consider whether those rules \& regulations
   also apply with the generated snippet.  
  
  For example, the EULA might need to state that generated snippets cannot be used for commercial purposes if the input contains song from a 
  specific artist belonging to some specific record label. This means that a creative professional's ability to actually use 
  this portion of the service could be greatly limited depending on what tracks they
  are inputting, and/or what the machine learning algorithm was trained on. 

  \item Dataset Bias, \#2
  \\\textbf{Summary:} We are currently not exactly certain about what dataset we will be using for the algorithm training. 
  For example, if we are using some form of non-copyrighted music, there's a likely chance it is heavily biased towards electronic dance 
  music as a genre. This could skew the algorithm's ability to interact with music that is not of that genre, thus the service 
  might not be able to properly process requests users who perhaps listen to more niche genres of music. This means we might need to train the 
  machine learning algorithms on multiple different datasets, or we need to manually modify an existing dataset to integrate more niche genres 
  of music as the training source. 

\end{itemize}

\section{Off-the-Shelf Solutions}
\subsection{Ready-Made Products}
\lips
\subsection{Reusable Components}
\lips
\subsection{Products That Can Be Copied}
\lips

\section{New Problems}

\subsection{Effects on the Current Environment}

The platform's high computational requirements, especially during the training and usage of generative models, may put a strain on existing computational resources. This could lead to slower performance for other applications that share these resources. Additionally, increased data transmission due to the integration with APIs such as Spotify and Deezer could lead to higher network bandwidth utilization, potentially affecting the performance of other network-dependent systems. Mitigating these effects may require infrastructure scaling, such as increasing server capacity or optimizing data processing, to ensure that the existing systems are not disrupted.

\subsection{Effects on the Installed Systems}

The new platform will interact with various installed systems, leading to the introduction of new dependencies. For example, using third-party APIs like Spotify or integrating machine learning frameworks such as TensorFlow may require changes to the current software stack. These dependencies could introduce compatibility issues, particularly if third-party providers update their APIs or if new versions of required libraries are released. Such updates might necessitate timely system changes to maintain functionality and avoid disruptions. Ensuring that installed systems remain compatible requires diligent version management, comprehensive testing, and a structured process for managing library and API updates.

\subsection{Potential User Problems}

The platform is designed to offer advanced features, which may present usability challenges for some users. For example, users with limited technical skills may struggle to understand the process of modifying musical features like key, rhythm, or tempo, which are essential for customizing music generation. Furthermore, the quality of generated outputs is inherently subjective, and users may feel frustrated if the system-generated music does not meet their expectations. Another potential problem is system latency, especially when resource-intensive operations like model inference are being performed. Long waiting times could negatively impact user experience, making the platform feel less responsive, which may deter regular use.

\subsection{Limitations in the Anticipated Implementation Environment That May Inhibit the New Product}

The platform is expected to be implemented on a local server, which presents certain limitations. The local server environment will need sufficient processing power to handle computationally intensive tasks, such as model training and music generation. If the server infrastructure is under-resourced, performance issues such as latency and reduced processing speed could arise. Additionally, reliance on a local server means that the platform's effectiveness is tied directly to server availability and maintenance. Any server downtime could render the platform inaccessible to users. If we decide to extend the platform's implementation onto personal devices, new limitations arise. Personal devices, especially those with limited hardware capabilities, may struggle with the computational requirements of generative music models, leading to significant performance bottlenecks. This could make the platform inaccessible or difficult to use for some users. Moreover, the reliance on internet connectivity for accessing external APIs is another major limitation. Users in areas with unreliable internet or low bandwidth may face difficulties, particularly with real-time features like music recommendations or generating new songs based on streaming data. These limitations could restrict the user base to only those with high-performance devices and stable internet connections.

\subsection{Follow-Up Problems}

Once the platform is deployed, several follow-up issues will need to be addressed to maintain and improve the product. One of the primary challenges will be keeping up with third-party API changes, such as modifications to Spotify or Deezer APIs, which could lead to broken features if not handled promptly. The generative models used by the platform will also require regular updates and retraining to stay relevant to emerging musical styles and user preferences. Another follow-up problem is related to usability: feedback from users might reveal unforeseen pain points or desired improvements. Addressing these concerns will require an iterative development approach, incorporating user feedback into future versions of the platform to enhance usability and performance.


\lips
\subsection{Follow-Up Problems}
\lips

\section{Tasks}
\begin{itemize}
  \item Revise requirements document
  \item Design the systems structure
  \item Build prototype and get feedback
  \item Integrate Spotify API for song data
  \item Develop music recommendation module
  \item Work on music generation  module
  \item Create song analysis moudle
  \item Test the platform and gather feedback
  \item Revise based on feedback
  \item Write user guide and documentation
  \item Finalize revisions to documentation
\end{itemize}

\section{Migration to the New Product}
\subsection{Requirements for Migration to the New Product}
There are no migration requirements as this project is not a replacement or upgrade of a previous project
\subsection{Data That Has to be Modified or Translated for the New System}
Similarly, there currently is no data that needs to be modified

\section{Costs}
The monetary cost estimate of the project is \$0 CAD. All of the necessary equipment is
owned by at least one group member.\\

\noindent
The total time cost estimate of the project is 8 months (September 2024 - April 2025).\\

\noindent
The function point cost estimate is 12. This is derived from the above sections, mainly the functional
requirements, non-functional requirements, and business rules
\section{User Documentation and Training}

\subsection{User Documentation Requirements}

The music featurization feature is heavily API-driven, and as such, detailed documentation will primarily be covered within the API reference section. This approach ensures that developers and advanced users can understand the feature's capabilities without needing additional user guides.

To ensure that users can effectively interact with the music generation and recommendation platform, the following user documentation will be provided:

\begin{itemize}
  \item \textbf{Quick Start Guide}: A concise guide aimed at helping new users get started with the basics of generating and recommending music.
  \item \textbf{API Reference and Technical Specifications}: Detailed documentation of the platform’s API, including available endpoints, request/response formats, and example queries. This reference is crucial for developers and advanced users who want to integrate the platform with other applications or automate tasks.
  \item \textbf{Installation Guide}: A step-by-step guide for installing the platform on local servers, including system requirements, installation commands, and troubleshooting common setup issues.
  \item \textbf{FAQs and Troubleshooting}: A list of frequently asked questions and troubleshooting tips to help users solve common issues independently.
  \item \textbf{Video Tutorials}: Step-by-step video guides that visually demonstrate key features and workflows, including setting up the platform, using the API, and generating music.
\end{itemize}

These documents will be designed for users of varying technical backgrounds to ensure they can fully utilize the platform's capabilities. The documentation will be created and maintained primarily by the development team, ensuring accuracy and alignment with the latest platform features. However, feedback from user groups will be actively sought to improve clarity and address any documentation gaps. Updates to the API reference and technical specifications will be managed as part of the regular software update cycle.
\subsection{Training Requirements}

To provide users with sufficient knowledge to operate the platform effectively, the following training resources will be developed:

\begin{itemize}
  \item \textbf{Video Tutorials}: Developed by the development team, these tutorials will cover various aspects of the platform, including API usage, generating music, and using advanced features. 
  \item \textbf{Online Training Modules}: If additional resources become available, online training modules could be developed to provide users with a structured learning path. However, due to current resource constraints, we do not plan to offer live training sessions.
\end{itemize}

These training requirements aim to encourage users to explore the full potential of the platform, regardless of their prior experience in music production or technology.


\section{Waiting Room}
\begin{enumerate}
  \item The recommendation system will be able to recommend songs from less popular music genres (jazz, blues, etc.)
  \item The analysis system will be able to extract musical features from less popular music genres (jazz, blues, etc.)
  \item The generative system will be able to generate songs from less popular music genres (jazz, blues, etc.)
  \item The recommendation system will be able to recommend songs from unpopular music genres (gnawa, libyan funk, etc.)
  \item The analysis system will be able to extract musical features from less popular music genres (gnawa, libyan funk, etc.)
  \item The generative system will be able to generate songs from less popular music genres (gnawa, libyan funk, etc.)
  \item The recommendation system will be able to search for songs with cover art similar to an input song
  \item The generative system will be able to generate new cover art for a newly generated song, based on the user's
  input criteria
  \item The generative system will be able to generate new covert art for an existing song
\end{enumerate}

\section{Ideas for Solution}

\begin{itemize}
    \item \textbf{Hybrid Recommendation System}: 
    A hybrid recommendation system combines content-based filtering and collaborative filtering techniques to provide a more personalized experience for users. Content-based filtering analyzes song features, such as genre, key, and rhythm, to suggest similar tracks. Collaborative filtering uses user preferences and historical listening patterns to suggest music. By combining these approaches, the system can offer users personalized suggestions while also helping them discover new genres and music styles.

    \item \textbf{Generative Music Model}: 
    To enable the creation of new music, a generative model will be used. This model could be based on techniques such as a Generative Adversarial Network (GAN) or Recurrent Neural Network (RNN). A GAN would allow for the generation of realistic music by having the generator and discriminator work together to produce convincing compositions. An RNN, on the other hand, would be well-suited for learning the sequential nature of music, generating new melodies based on learned patterns. This solution provides users with an innovative way to create new music based on their inputs and preferences.

    \item \textbf{Feature Manipulation Interface}: 
    This interface will allow users to interact directly with song features, such as tempo, key, and rhythm, enabling them to create customized versions of existing tracks or generate entirely new compositions. By adjusting different musical parameters, users can personalize their musical experience and experiment with creative variations, providing a high level of control over the output.

    \item \textbf{Integration with Existing Platforms}: 
    Integrating the system with existing music platforms, such as Spotify, will allow users to easily access and analyze a large library of songs. Users will be able to input their favorite tracks from these platforms and generate variations or receive recommendations. This integration ensures a smooth user experience, allowing seamless interaction between existing music libraries and the platform's generative capabilities.

\end{itemize}


\newpage{}
\section*{Appendix --- Reflection}

The information in this section will be used to evaluate the team members on the
graduate attribute of Lifelong Learning.  Please answer the following questions:

\begin{enumerate}
  \item What knowledge and skills will the team collectively need to acquire to
  successfully complete this capstone project?  Examples of possible knowledge
  to acquire include domain specific knowledge from the domain of your
  application, or software engineering knowledge, mechatronics knowledge or
  computer science knowledge.  Skills may be related to technology, or writing,
  or presentation, or team management, etc.  You should look to identify at
  least one item for each team member.
  \\ \\ 
  As a team, the Rhythm Rangers, we need to acquire a diverse range of knowledge and skills from various domains, including software development, 
  music generation, and collaborative teamwork, to successfully complete our capstone project. Given the scope of this task, it is essential 
  for each team member to focus on specific areas of expertise that align with their skills, passions, roles, and responsibilities, 
  as well as learn new skills and gain new knowledge. Outlined below is the knowledge and skills the team will collectively need to acquire
  to successfully complete this capstone project: 
  \\ \\
  \textbf{Music Analysis and Signal Processing:} This capstone project involves developing expertise in audio signal processing to analyze sound data and extract 
  valuable insights for music recommendation and generation systems. The team will learn to implement machine learning models for tasks such as genre 
  classification and feature extraction. Proficiency in Python libraries for audio analysis and model training is essential. This will deepen the 
  teams understanding of music theory and the connections between song features and genres.
  \\ \\ 
  \textbf{Frontend or Backend Development:} The team will need to understand backend frameworks for building and managing the recommendation system's 
  infrastructure. The will be integrating external APIs to access song previews and features. They will also gain knowledge in database 
  management for storing and organizing song data and user preferences. Furthermore, this involves learning how to scale and efficiently handle data
  for a local server-based system.
  \\ \\
  \textbf{UI/UX and Design:} The team will need to design user-friendly interfaces that ensure smooth interaction with the music recommendation and 
  generation systems. UI/UX design skills will need refinement and utilizization of frontend development frameworks will be needed to craft the systems user 
  interface. They will also learn to connect frontend components with backend APIs for real-time updates, such as delivering song recommendations.
  \\ \\
  \textbf{Music Generation:} An understanding of generative models to create music snippets from input tracks or references will be a huge component for this project. 
  The team will delve into music feature engineering, transforming audio data into usable features for machine learning applications. Familiarity with music data 
  will assist in generating new content and making recommendations.
  \\ \\ 
  \textbf{Team Management and Infrastructure:} For this project to be a success improving project management and team coordination skills will foster effective communication, 
  sprint planning, and task assignment. We will need to learn how to establish and maintain local server infrastructure for efficient hosting and operation of the platform. 
  Understanding security best practices to safeguard user data and ensure the system's resilience against vulnerabilities is critical. Mastery of version control and Git management 
  will promote seamless collaboration and code review among the team.
  \\ \\
  \item For each of the knowledge areas and skills identified in the previous
  question, what are at least two approaches to acquiring the knowledge or
  mastering the skill?  Of the identified approaches, which will each team
  member pursue, and why did they make this choice?
\end{enumerate}

\end{document}