% THIS DOCUMENT IS FOLLOWS THE VOLERE TEMPLATE BY Suzanne Robertson and James Robertson
% ONLY THE SECTION HEADINGS ARE PROVIDED
%
% Initial draft from https://github.com/Dieblich/volere
%
% Risks are removed because they are covered by the Hazard Analysis
\documentclass[12pt]{article}

\usepackage{booktabs}
\usepackage{tabularx}
\usepackage{hyperref}
\hypersetup{
    bookmarks=true,         % show bookmarks bar?
      colorlinks=true,      % false: boxed links; true: colored links
    linkcolor=red,          % color of internal links (change box color with linkbordercolor)
    citecolor=green,        % color of links to bibliography
    filecolor=magenta,      % color of file links
    urlcolor=cyan           % color of external links
}

\newcommand{\lips}{\textit{Insert your content here.}}

%% Comments

\usepackage{color}

\newif\ifcomments\commentstrue %displays comments
%\newif\ifcomments\commentsfalse %so that comments do not display

\ifcomments
\newcommand{\authornote}[3]{\textcolor{#1}{[#3 ---#2]}}
\newcommand{\todo}[1]{\textcolor{red}{[TODO: #1]}}
\else
\newcommand{\authornote}[3]{}
\newcommand{\todo}[1]{}
\fi

\newcommand{\wss}[1]{\authornote{blue}{SS}{#1}} 
\newcommand{\plt}[1]{\authornote{magenta}{TPLT}{#1}} %For explanation of the template
\newcommand{\an}[1]{\authornote{cyan}{Author}{#1}}

%% Common Parts

\newcommand{\progname}{Software Engineering} % PUT YOUR PROGRAM NAME HERE
\newcommand{\authname}{Team 8 -- Rhythm Rangers\\
\\ Ansel Chen
\\ Muhammad Jawad
\\ Mohamad-Hassan Bahsoun
\\ Matthew Baleanu
\\ Ahmed Al-Hayali} % AUTHOR NAMES                  

\usepackage{hyperref}
    \hypersetup{colorlinks=true, linkcolor=blue, citecolor=blue, filecolor=blue,
                urlcolor=blue, unicode=false}
    \urlstyle{same}
                                


\begin{document}

\title{Software Requirements Specification for \progname: subtitle describing software} 
\author{\authname}
\date{\today}
	
\maketitle

~\newpage

\pagenumbering{roman}

\tableofcontents

~\newpage

\section*{Revision History}

\begin{tabularx}{\textwidth}{p{3cm}p{2cm}X}
\toprule {\textbf{Date}} & {\textbf{Version}} & {\textbf{Notes}}\\
\midrule
Date 1 & 1.0 & Notes\\
Date 2 & 1.1 & Notes\\
\bottomrule
\end{tabularx}

~\\

~\newpage
\section{Purpose of the Project}
\subsection{User Business}
\lips
\subsection{Goals of the Project}
\lips
\section{Stakeholders}
\subsection{Client}
\lips
\subsection{Customer}
\lips
\subsection{Other Stakeholders}
\lips
\subsection{Hands-On Users of the Project}
\lips
\subsection{Personas}
\lips
\subsection{Priorities Assigned to Users}
\lips
\subsection{User Participation}
\lips
\subsection{Maintenance Users and Service Technicians}
\lips

\section{Mandated Constraints}
\subsection{Solution Constraints}
\lips
\subsection{Implementation Environment of the Current System}
\lips
\subsection{Partner or Collaborative Applications}
\lips
\subsection{Off-the-Shelf Software}


There are several existing solutions that could serve as part of the music generation and recommendation system. These include:

\begin{itemize}
    \item \textbf{Spotify API}: Provides access to a vast library of music, including song previews and metadata, which can be leveraged for generating recommendations.
    \item \textbf{Librosa Library}: An open-source Python package for analyzing and processing music files, suitable for extracting features from songs and facilitating generative components.
    \item \textbf{TensorFlow and PyTorch Pre-trained Models}: Both frameworks offer pre-trained models that could be adapted for music generation tasks. These solutions provide a basis for deep learning models without having to build and train from scratch.
    \item \textbf{OpenAI Jukebox}: A generative model that is capable of producing music, which could potentially be adapted and integrated into our system.
\end{itemize}

These off-the-shelf software solutions provide a foundation upon which we can build our custom features, significantly reducing the development time and leveraging existing technologies to enhance the functionality of our platform.

\lips
\subsection{Anticipated Workplace Environment}
\lips
\subsection{Schedule Constraints}
\lips
\subsection{Budget Constraints}
\lips
\subsection{Enterprise Constraints}
\lips

\section{Naming Conventions and Terminology}
\subsection{Glossary of All Terms, Including Acronyms, Used by Stakeholders
involved in the Project}
\lips

\section{Relevant Facts And Assumptions}
\subsection{Relevant Facts}
\lips
\subsection{Business Rules}
\lips
\subsection{Assumptions}
\lips

\section{The Scope of the Work}
\subsection{The Current Situation}
\lips
\subsection{The Context of the Work}
\lips
\subsection{Work Partitioning}
\lips
\subsection{Specifying a Business Use Case (BUC)}
\lips

\section{Business Data Model and Data Dictionary}
\subsection{Business Data Model}
\lips
\subsection{Data Dictionary}
\lips

\section{The Scope of the Product}
\subsection{Product Boundary}
\lips
\subsection{Product Use Case Table}
\lips
\subsection{Individual Product Use Cases (PUC's)}
\lips

\section{Functional Requirements}
\subsection{Functional Requirements}
\lips

\section{Look and Feel Requirements}
\subsection{Appearance Requirements}
\lips
\subsection{Style Requirements}
\lips

\section{Usability and Humanity Requirements}
\subsection{Ease of Use Requirements}
\lips
\subsection{Personalization and Internationalization Requirements}
\lips
\subsection{Learning Requirements}
\lips
\subsection{Understandability and Politeness Requirements}
\lips
\subsection{Accessibility Requirements}
\lips

\section{Performance Requirements}
\subsection{Speed and Latency Requirements}
\lips
\subsection{Safety-Critical Requirements}
\lips
\subsection{Precision or Accuracy Requirements}
\lips
\subsection{Robustness or Fault-Tolerance Requirements}
\lips
\subsection{Capacity Requirements}
\lips
\subsection{Scalability or Extensibility Requirements}
\lips
\subsection{Longevity Requirements}
\lips

\section{Operational and Environmental Requirements}
\subsection{Expected Physical Environment}
\lips
\subsection{Wider Environment Requirements}
\lips
\subsection{Requirements for Interfacing with Adjacent Systems}
\lips
\subsection{Productization Requirements}
\lips
\subsection{Release Requirements}
\lips

\section{Maintainability and Support Requirements}
\subsection{Maintenance Requirements}
\lips
\subsection{Supportability Requirements}
\lips
\subsection{Adaptability Requirements}
\lips

\section{Security Requirements}
\subsection{Access Requirements}
\lips
\subsection{Integrity Requirements}
\lips
\subsection{Privacy Requirements}
\lips
\subsection{Audit Requirements}
\lips
\subsection{Immunity Requirements}
\lips

\section{Cultural Requirements}
\subsection{Cultural Requirements}
\lips

\section{Compliance Requirements}
\subsection{Legal Requirements}
\lips
\subsection{Standards Compliance Requirements}
\lips

\section{Open Issues}
\begin{itemize}
  \item Generated output use regulation, \#1
  \\\textbf{Summary:} The service EULA agreement might need to consider how and where the generated snippet by the service is allowed
  to be used. 
  
  This is important because the training data or inputted track(s) into the service
  might have copyright laws \& rules regulating commerical use associated with them, thus we need to consider whether those rules \& regulations
   also apply with the generated snippet.  
  
  For example, the EULA might need to state that generated snippets cannot be used for commercial purposes if the input contains song from a 
  specific artist belonging to some specific record label. This means that a creative professional's ability to actually use 
  this portion of the service could be greatly limited depending on what tracks they
  are inputting, and/or what the machine learning algorithm was trained on. 

  \item Dataset Bias, \#2
  \\\textbf{Summary:} We are currently not exactly certain about what dataset we will be using for the algorithm training. 
  For example, if we are using some form of non-copyrighted music, there's a likely chance it is heavily biased towards electronic dance 
  music as a genre. This could skew the algorithm's ability to interact with music that is not of that genre, thus the service 
  might not be able to properly process requests users who perhaps listen to more niche genres of music. This means we might need to train the 
  machine learning algorithms on multiple different datasets, or we need to manually modify an existing dataset to integrate more niche genres 
  of music as the training source. 

\end{itemize}

\section{Off-the-Shelf Solutions}
\subsection{Ready-Made Products}
\lips
\subsection{Reusable Components}
\lips
\subsection{Products That Can Be Copied}
\lips

\section{New Problems}
\subsection{Effects on the Current Environment}
\lips
\subsection{Effects on the Installed Systems}
\lips
\subsection{Potential User Problems}
\lips
\subsection{Limitations in the Anticipated Implementation Environment That May
Inhibit the New Product}
\lips
\subsection{Follow-Up Problems}
\lips

\section{Tasks}
\subsection{Project Planning}
\lips
\subsection{Planning of the Development Phases}
\lips

\section{Migration to the New Product}
\subsection{Requirements for Migration to the New Product}
There are no migration requirements as this project is not a replacement or upgrade of a previous project
\subsection{Data That Has to be Modified or Translated for the New System}
Similarly, there currently is no data that needs to be modified

\section{Costs}
\lips
\section{User Documentation and Training}
\subsection{User Documentation Requirements}
\lips
\subsection{Training Requirements}
\lips

\section{Waiting Room}
\lips

\section{Ideas for Solution}

\begin{itemize}
    \item \textbf{Hybrid Recommendation System}: 
    A hybrid recommendation system combines content-based filtering and collaborative filtering techniques to provide a more personalized experience for users. Content-based filtering analyzes song features, such as genre, key, and rhythm, to suggest similar tracks. Collaborative filtering uses user preferences and historical listening patterns to suggest music. By combining these approaches, the system can offer users personalized suggestions while also helping them discover new genres and music styles.

    \item \textbf{Generative Music Model}: 
    To enable the creation of new music, a generative model will be used. This model could be based on techniques such as a Generative Adversarial Network (GAN) or Recurrent Neural Network (RNN). A GAN would allow for the generation of realistic music by having the generator and discriminator work together to produce convincing compositions. An RNN, on the other hand, would be well-suited for learning the sequential nature of music, generating new melodies based on learned patterns. This solution provides users with an innovative way to create new music based on their inputs and preferences.

    \item \textbf{Feature Manipulation Interface}: 
    This interface will allow users to interact directly with song features, such as tempo, key, and rhythm, enabling them to create customized versions of existing tracks or generate entirely new compositions. By adjusting different musical parameters, users can personalize their musical experience and experiment with creative variations, providing a high level of control over the output.

    \item \textbf{Integration with Existing Platforms}: 
    Integrating the system with existing music platforms, such as Spotify, will allow users to easily access and analyze a large library of songs. Users will be able to input their favorite tracks from these platforms and generate variations or receive recommendations. This integration ensures a smooth user experience, allowing seamless interaction between existing music libraries and the platform's generative capabilities.

\end{itemize}


\newpage{}
\section*{Appendix --- Reflection}

The information in this section will be used to evaluate the team members on the
graduate attribute of Lifelong Learning.  Please answer the following questions:

\begin{enumerate}
  \item What knowledge and skills will the team collectively need to acquire to
  successfully complete this capstone project?  Examples of possible knowledge
  to acquire include domain specific knowledge from the domain of your
  application, or software engineering knowledge, mechatronics knowledge or
  computer science knowledge.  Skills may be related to technology, or writing,
  or presentation, or team management, etc.  You should look to identify at
  least one item for each team member.
  \item For each of the knowledge areas and skills identified in the previous
  question, what are at least two approaches to acquiring the knowledge or
  mastering the skill?  Of the identified approaches, which will each team
  member pursue, and why did they make this choice?
\end{enumerate}

\end{document}