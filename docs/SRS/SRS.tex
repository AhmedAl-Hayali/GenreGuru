% THIS DOCUMENT IS FOLLOWS THE VOLERE TEMPLATE BY Suzanne Robertson and James Robertson
% ONLY THE SECTION HEADINGS ARE PROVIDED
%
% Initial draft from https://github.com/Dieblich/volere
%
% Risks are removed because they are covered by the Hazard Analysis
\documentclass[12pt]{article}

\usepackage{booktabs}
\usepackage{tabularx}
\usepackage{hyperref}
\hypersetup{
    bookmarks=true,         % show bookmarks bar?
      colorlinks=true,      % false: boxed links; true: colored links
    linkcolor=red,          % color of internal links (change box color with linkbordercolor)
    citecolor=green,        % color of links to bibliography
    filecolor=magenta,      % color of file links
    urlcolor=cyan           % color of external links
}

\newcommand{\lips}{\textit{Insert your content here.}}

%% Comments

\usepackage{color}

\newif\ifcomments\commentstrue %displays comments
%\newif\ifcomments\commentsfalse %so that comments do not display

\ifcomments
\newcommand{\authornote}[3]{\textcolor{#1}{[#3 ---#2]}}
\newcommand{\todo}[1]{\textcolor{red}{[TODO: #1]}}
\else
\newcommand{\authornote}[3]{}
\newcommand{\todo}[1]{}
\fi

\newcommand{\wss}[1]{\authornote{blue}{SS}{#1}} 
\newcommand{\plt}[1]{\authornote{magenta}{TPLT}{#1}} %For explanation of the template
\newcommand{\an}[1]{\authornote{cyan}{Author}{#1}}

%% Common Parts

\newcommand{\progname}{Software Engineering} % PUT YOUR PROGRAM NAME HERE
\newcommand{\authname}{Team 8 -- Rhythm Rangers\\
\\ Ansel Chen
\\ Muhammad Jawad
\\ Mohamad-Hassan Bahsoun
\\ Matthew Baleanu
\\ Ahmed Al-Hayali} % AUTHOR NAMES                  

\usepackage{hyperref}
    \hypersetup{colorlinks=true, linkcolor=blue, citecolor=blue, filecolor=blue,
                urlcolor=blue, unicode=false}
    \urlstyle{same}
                                


\begin{document}

\title{Software Requirements Specification for \progname: subtitle describing software} 
\author{\authname}
\date{\today}
	
\maketitle

~\newpage

\pagenumbering{roman}

\tableofcontents

~\newpage

\section*{Revision History}

\begin{tabularx}{\textwidth}{p{3cm}p{2cm}X}
\toprule {\textbf{Date}} & {\textbf{Version}} & {\textbf{Notes}}\\
\midrule
Date 1 & 1.0 & Notes\\
Date 2 & 1.1 & Notes\\
\bottomrule
\end{tabularx}

~\\

~\newpage
\section{Purpose of the Project}
\subsection{User Business}
\lips
\subsection{Goals of the Project}
\lips
\section{Stakeholders}
\subsection{Client}
\lips
\subsection{Customer}
\lips
\subsection{Other Stakeholders}
\lips
\subsection{Hands-On Users of the Project}
\lips
\subsection{Personas}
\lips
\subsection{Priorities Assigned to Users}
\lips
\subsection{User Participation}
\lips
\subsection{Maintenance Users and Service Technicians}
\lips

\section{Mandated Constraints}
\subsection{Solution Constraints}
\lips
\subsection{Implementation Environment of the Current System}
\lips
\subsection{Partner or Collaborative Applications}
\lips
\subsection{Off-the-Shelf Software}
\lips
\subsection{Anticipated Workplace Environment}
\lips
\subsection{Schedule Constraints}
\lips
\subsection{Budget Constraints}
\lips
\subsection{Enterprise Constraints}
\lips

\section{Naming Conventions and Terminology}
\subsection{Glossary of All Terms, Including Acronyms, Used by Stakeholders
involved in the Project}
\lips

\section{Relevant Facts And Assumptions}
\subsection{Relevant Facts}
\lips
\subsection{Business Rules}
\lips
\subsection{Assumptions}
\lips

\section{The Scope of the Work}
\subsection{The Current Situation}
\lips
\subsection{The Context of the Work}
\lips
\subsection{Work Partitioning}
\lips
\subsection{Specifying a Business Use Case (BUC)}
\lips

\section{Business Data Model and Data Dictionary}
\subsection{Business Data Model}
\lips
\subsection{Data Dictionary}
\lips

\section{The Scope of the Product}
\subsection{Product Boundary}
\lips
\subsection{Product Use Case Table}
\lips
\subsection{Individual Product Use Cases (PUC's)}
\lips

\section{Functional Requirements}
\subsection{Functional Requirements}
\hspace{14pt}
\fbox{
\begin{minipage}{\textwidth}
\textbf{Requirement \# 1} \\
\textbf{Requirement Type:} 9 \\
\textbf{Event/Use Case \#:} 1 \\
\textbf{Description:} The system should respond to user actions (e.g. swipe, tap). \\
\textbf{Rationale:} To allow users to interact with the system efficiently and intuitively. \\
\textbf{Originator:} Requirement Analyst \\
\textbf{Fit Criterion:} User performs an action (e.g. swipe, tap) and system responds within 2 seconds. \\
\textbf{Customer Satisfaction:} 5 \\
\textbf{Customer Dissatisfaction:} 0 \\
\textbf{Priority:} High \\
\textbf{Conflicts:} None \\
\textbf{Supporting Material:} None \\
\textbf{History:} Created October 6, 2024
\end{minipage}
}

\vspace{10pt}

\fbox{
\begin{minipage}{\textwidth}
\textbf{Requirement \# 2} \\
\textbf{Requirement Type:} 9 \\
\textbf{Event/Use Case \#:} 2 \\
\textbf{Description:} The system allows the users to select song features (e.g. tempo, genre), and it returns recommendations. \\
\textbf{Rationale:} User wants song recommendations based on desired features. \\
\textbf{Originator:} Requirement Analyst \\
\textbf{Fit Criterion:} User can select features and does receive song recommendations. \\
\textbf{Customer Satisfaction:} 5 \\
\textbf{Customer Dissatisfaction:} 0 \\
\textbf{Priority:} High \\
\textbf{Conflicts:} None \\
\textbf{Supporting Material:} None \\
\textbf{History:} Created October 6, 2024
\end{minipage}
}

\vspace{10pt}

\fbox{
\begin{minipage}{\textwidth}
\textbf{Requirement \# 3} \\
\textbf{Requirement Type:} 9 \\
\textbf{Event/Use Case \#:} 3 \\
\textbf{Description:} The system generates a song based on reference song(s) or snippet(s) received from the user as input. \\
\textbf{Rationale:} Users need an easy way to create music, without prior knowledge, that is similar to their input songs. \\
\textbf{Originator:} Requirement Analyst \\
\textbf{Fit Criterion:} Music is generated from the input reference song(s) or snippet(s). \\
\textbf{Customer Satisfaction:} 5 \\
\textbf{Customer Dissatisfaction:} 0 \\
\textbf{Priority:} High \\
\textbf{Conflicts:} None \\
\textbf{Supporting Material:} None \\
\textbf{History:} Created October 6, 2024
\end{minipage}
}

\vspace{10pt}

\fbox{
\begin{minipage}{\textwidth}
\textbf{Requirement \# 4} \\
\textbf{Requirement Type:} 9 \\
\textbf{Event/Use Case \#:} 4 \\
\textbf{Description:} The system will analyze a reference song or snippet, and provide its features. \\
\textbf{Rationale:} Users, more particularly music producers and educators, need a way to break down songs for a more detailed analysis of their features. \\
\textbf{Originator:} Requirement Analyst \\
\textbf{Fit Criterion:} User is provided with various features and visualizations showing an accurate breakdown of their song or snippet. \\
\textbf{Customer Satisfaction:} 5 \\
\textbf{Customer Dissatisfaction:} 0 \\
\textbf{Priority:} High \\
\textbf{Conflicts:} None \\
\textbf{Supporting Material:} None \\
\textbf{History:} Created October 11, 2024
\end{minipage}
}

\vspace{10pt}

\fbox{
\begin{minipage}{\textwidth}
\textbf{Requirement \# 5} \\
\textbf{Requirement Type:} 9 \\
\textbf{Event/Use Case \#:} 4 \\
\textbf{Description:} The system will analyze a reference song or snippet, and provide its visualizations. \\
\textbf{Rationale:} Users, more particularly music producers and educators, need a way to visualize the breakdown of songs for a better understading and analysis. \\
\textbf{Originator:} Requirement Analyst \\
\textbf{Fit Criterion:} User is provided with various visualizations that accurately represent their song or snippet. \\
\textbf{Customer Satisfaction:} 5 \\
\textbf{Customer Dissatisfaction:} 0 \\
\textbf{Priority:} High \\
\textbf{Conflicts:} None \\
\textbf{Supporting Material:} None \\
\textbf{History:} Created October 11, 2024
\end{minipage}
}

\vspace{10pt}

\fbox{
\begin{minipage}{\textwidth}
\textbf{Requirement \# 6} \\
\textbf{Requirement Type:} 9 \\
\textbf{Event/Use Case \#:} 5 \\
\textbf{Description:} Users want recommendations based on reference song(s) and/or snippet(s). \\
\textbf{Rationale:} Users want to discover new music or want music similar to the ones they are listening to. \\
\textbf{Originator:} Requirement Analyst \\
\textbf{Fit Criterion:} The system returns a list of recommendations. \\
\textbf{Customer Satisfaction:} 5 \\
\textbf{Customer Dissatisfaction:} 0 \\
\textbf{Priority:} High \\
\textbf{Conflicts:} None \\
\textbf{Supporting Material:} None \\
\textbf{History:} Created October 6, 2024
\end{minipage}
}

\vspace{10pt}

\fbox{
\begin{minipage}{\textwidth}
\textbf{Requirement \# 7} \\
\textbf{Requirement Type:} 9 \\
\textbf{Event/Use Case \#:} 2,3,4,5 \\
\textbf{Description:} The system will validate user inputs to ensure they are correct. \\
\textbf{Rationale:} Prevents errors and ensures the system processes valid data. \\
\textbf{Originator:} Requirement Analyst \\
\textbf{Fit Criterion:} The system will display an error message if the input is invalid, or will let the user proceed if the input is valid. \\
\textbf{Customer Satisfaction:} 5 \\
\textbf{Customer Dissatisfaction:} 0 \\
\textbf{Priority:} High \\
\textbf{Conflicts:} None \\
\textbf{Supporting Material:} None \\
\textbf{History:} Created October 6, 2024
\end{minipage}
}

\section{Look and Feel Requirements}
\subsection{Appearance Requirements}
\lips
\subsection{Style Requirements}
\lips

\section{Usability and Humanity Requirements}
\subsection{Ease of Use Requirements}
\lips
\subsection{Personalization and Internationalization Requirements}
\lips
\subsection{Learning Requirements}
\lips
\subsection{Understandability and Politeness Requirements}
\lips
\subsection{Accessibility Requirements}
\lips

\section{Performance Requirements}
\subsection{Speed and Latency Requirements}
\lips
\subsection{Safety-Critical Requirements}
\lips
\subsection{Precision or Accuracy Requirements}
\lips
\subsection{Robustness or Fault-Tolerance Requirements}
\lips
\subsection{Capacity Requirements}
\lips
\subsection{Scalability or Extensibility Requirements}
\lips
\subsection{Longevity Requirements}
\lips

\section{Operational and Environmental Requirements}
\subsection{Expected Physical Environment}
\lips
\subsection{Wider Environment Requirements}
\lips
\subsection{Requirements for Interfacing with Adjacent Systems}
\lips
\subsection{Productization Requirements}
\lips
\subsection{Release Requirements}
\lips

\section{Maintainability and Support Requirements}
\subsection{Maintenance Requirements}
\lips
\subsection{Supportability Requirements}
\lips
\subsection{Adaptability Requirements}
\lips

\section{Security Requirements}
\subsection{Access Requirements}
\lips
\subsection{Integrity Requirements}
\lips
\subsection{Privacy Requirements}
\lips
\subsection{Audit Requirements}
\lips
\subsection{Immunity Requirements}
\lips

\section{Cultural Requirements}
\subsection{Cultural Requirements}
\lips

\section{Compliance Requirements}
\subsection{Legal Requirements}
\lips
\subsection{Standards Compliance Requirements}
\lips

\section{Open Issues}
\lips

\section{Off-the-Shelf Solutions}
\subsection{Ready-Made Products}
\lips
\subsection{Reusable Components}
\lips
\subsection{Products That Can Be Copied}
\lips

\section{New Problems}
\subsection{Effects on the Current Environment}
\lips
\subsection{Effects on the Installed Systems}
\lips
\subsection{Potential User Problems}
\lips
\subsection{Limitations in the Anticipated Implementation Environment That May
Inhibit the New Product}
\lips
\subsection{Follow-Up Problems}
\lips

\section{Tasks}
\subsection{Project Planning}
\lips
\subsection{Planning of the Development Phases}
\lips

\section{Migration to the New Product}
\subsection{Requirements for Migration to the New Product}
\lips
\subsection{Data That Has to be Modified or Translated for the New System}
\lips

\section{Costs}
\lips
\section{User Documentation and Training}
\subsection{User Documentation Requirements}
\lips
\subsection{Training Requirements}
\lips

\section{Waiting Room}
\lips

\section{Ideas for Solution}
\lips

\newpage{}
\section*{Appendix --- Reflection}

The information in this section will be used to evaluate the team members on the
graduate attribute of Lifelong Learning.  Please answer the following questions:

\begin{enumerate}
  \item What knowledge and skills will the team collectively need to acquire to
  successfully complete this capstone project?  Examples of possible knowledge
  to acquire include domain specific knowledge from the domain of your
  application, or software engineering knowledge, mechatronics knowledge or
  computer science knowledge.  Skills may be related to technology, or writing,
  or presentation, or team management, etc.  You should look to identify at
  least one item for each team member.
  \item For each of the knowledge areas and skills identified in the previous
  question, what are at least two approaches to acquiring the knowledge or
  mastering the skill?  Of the identified approaches, which will each team
  member pursue, and why did they make this choice?
  \\ \\ 
  \textbf{Music Analysis and Signal Processing:} To acquire knowledge and skills in Music Analysis and Signal Processing, we can take online courses focused 
  on audio signal processing and machine learning for music, building a solid theoretical foundation. We can also engage in discussions with university professors 
  knowledgeable in this field. Additionally, we will take a look into analyzing audio data and implementing models using libraries like Librosa.
  \\ \\
  \textbf{Frontend or Backend Development:} For knowledge and skills in Frontend or Backend Development, we will look at documentation for backend 
  frameworks like Django or Flask to learn how to build and manage the recommendation system. We will collaborate with teammates to share knowledge and enhance 
  our skills.
  \\ \\ 
  \textbf{UI/UX and Design:} To acquire knowledge and skills in UI/UX and Design, we can read books and articles on UI/UX best practices to deepen our 
  understanding of design principles. We can conduct user testing sessions with prototypes to gather feedback and iterate on the design.
  \\ \\ 
  \textbf{Music Generation and AI:} To acquire knowledge and skills in Music Generation and AI, we can will read research papers on generative models in music to 
  understand their applications. Using Python libraries, we can test various algorithms to gain practical experience.
  \\ \\
  \textbf{Team Management and Infrastructure:} For Team Management and Infrastructure, the team will read books and articles on effective 
  team leadership and management strategies, allowing us to understand different team dynamics and communication styles. We will also draw on our past experiences with team 
  managers from co-op terms. Additionally, studying documentation on local server management and security best practices will help to establish a strong foundation for 
  the project’s infrastructure.
  \\
\end{enumerate}

\end{document}