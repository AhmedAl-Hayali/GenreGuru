\documentclass{article}

\usepackage{float}
\restylefloat{table}

\usepackage{booktabs}

\title{Team Contributions: Rev 0\\\progname}

\author{\authname}

\date{}

%% Comments

\usepackage{color}

\newif\ifcomments\commentstrue %displays comments
%\newif\ifcomments\commentsfalse %so that comments do not display

\ifcomments
\newcommand{\authornote}[3]{\textcolor{#1}{[#3 ---#2]}}
\newcommand{\todo}[1]{\textcolor{red}{[TODO: #1]}}
\else
\newcommand{\authornote}[3]{}
\newcommand{\todo}[1]{}
\fi

\newcommand{\wss}[1]{\authornote{blue}{SS}{#1}} 
\newcommand{\plt}[1]{\authornote{magenta}{TPLT}{#1}} %For explanation of the template
\newcommand{\an}[1]{\authornote{cyan}{Author}{#1}}

%% Common Parts

\newcommand{\progname}{Software Engineering} % PUT YOUR PROGRAM NAME HERE
\newcommand{\authname}{Team 8 -- Rhythm Rangers\\
\\ Ansel Chen
\\ Muhammad Jawad
\\ Mohamad-Hassan Bahsoun
\\ Matthew Baleanu
\\ Ahmed Al-Hayali} % AUTHOR NAMES                  

\usepackage{hyperref}
    \hypersetup{colorlinks=true, linkcolor=blue, citecolor=blue, filecolor=blue,
                urlcolor=blue, unicode=false}
    \urlstyle{same}
                                


\begin{document}

\maketitle

This document summarizes the contributions of each team member for the Rev 0
Demo.  The time period of interest is the time between the POC demo and the Rev
0 demo.

\section{Demo Plans}

\begin{itemize}
    \item Demonstrate working client program front-end: the user shall be able to input a plain-text title of a song, receive Spotify preview options, select the desired song from amongst them, then be displayed a collection of Spotify preview songs as recommendation output. \textcolor{red}{May also demonstrate audio file input}.
    \item \emph{Optionally}, show database schema and a sample of database query results.
    \item \emph{Optionally}, briefly explain how a subset of the features are extracted, and potentially relevant visualizations. A full list of features follows:
    \begin{itemize}
        \item Dynamic Range.
        \item Tempo (Beats Per Minute).
        \item Frequency Range.
        \item Root Mean Square (Average Loudness).
        \item Key \& Scale.
        \item Timbre.
        \item Vocal Gender.
        \item Instrumentalness.
        \item \emph{Latent (derived) features}.
        \item \emph{Contour (as a derived feature)}.
    \end{itemize}
\end{itemize}

\section{Team Meeting Attendance}

\begin{table}[H]
\centering
\begin{tabular}{ll}
\toprule
\textbf{Student} & \textbf{Meetings}\\
\midrule
Total & 26\\
Ansel Chen & 23\\
Muhammad Jawad & 19\\
Mohamad-Hassan Bahsoun & 23\\
Matthew Baleanu & 24\\
Ahmed Al-Hayali & 26\\
\bottomrule
\end{tabular}
\end{table}

\emph{No meetings were missed by any of the team members since the POC team contribution report. The only meeting that may appear so was between Ansel \& Ahmed just before the POC alignment meeting, which was not necessary.}

\section{Supervisor/Stakeholder Meeting Attendance}

\begin{table}[H]
\centering
\begin{tabular}{ll}
\toprule
\textbf{Student} & \textbf{Meetings}\\
\midrule
Total & 1\\
Ansel Chen & 1\\
Muhammad Jawad & 1\\
Mohamad-Hassan Bahsoun & 1\\
Matthew Baleanu & 1\\
Ahmed Al-Hayali & 1\\
\bottomrule
\end{tabular}
\end{table}

\section{Lecture Attendance}

\begin{table}[H]
\centering
\begin{tabular}{ll}
\toprule
\textbf{Student} & \textbf{Lectures}\\
\midrule
Total & 13\\
Ansel Chen & 11\\
Muhammad Jawad & 10\\
Mohamed-Hassan Bahsoun & 12\\
Matthew Baleanu & 10\\
Ahmed Al-Hayali & 12\\
\bottomrule
\end{tabular}
\end{table}

\emph{No lectures were missed by any of the team members since the POC team contribution report.}

\section{TA Document Discussion Attendance}

\begin{table}[H]
\centering
\begin{tabular}{ll}
\toprule
\textbf{Student} & \textbf{Lectures}\\
\midrule
Total & 4\\
Ansel Chen & 4\\
Muhammad Jawad & 4\\
Mohamed-Hassan Bahsoun & 4\\
Matthew Baleanu & 4\\
Ahmed Al-Hayali & 4\\
\bottomrule
\end{tabular}
\end{table}

\section{Commits}

\wss{For each team member how many commits to the main branch have been made
over the time period of interest.  The total is the total number of commits for
the entire team since the beginning of the term.  The percentage is the
percentage of the total commits made by each team member.}

\begin{table}[H]
\centering
\begin{tabular}{lll}
\toprule
\textbf{Student} & \textbf{Commits} & \textbf{Percent}\\
\midrule
Total & Num & 100\% \\
Name 1 & Num & \% \\
Name 2 & Num & \% \\
Name 3 & Num & \% \\
Name 4 & Num & \% \\
Name 5 & Num & \% \\
\bottomrule
\end{tabular}
\end{table}

\wss{If needed, an explanation for the counts can be provided here.  For
instance, if a team member has more commits to unmerged branches, these numbers
can be provided here.  If multiple people contribute to a commit, git allows for
multi-author commits.}

\section{Issue Tracker}

\wss{For each team member how many issues have they authored (including open and
closed issues (O+C)) and how many have they been assigned (only counting closed
issues (C only)) over the time period of interest.}

\begin{table}[H]
\centering
\begin{tabular}{lll}
\toprule
\textbf{Student} & \textbf{Authored (O+C)} & \textbf{Assigned (C only)}\\
\midrule
Name 1 & Num & Num \\
Name 2 & Num & Num \\
Name 3 & Num & Num \\
Name 4 & Num & Num \\
Name 5 & Num & Num \\
\bottomrule
\end{tabular}
\end{table}

\wss{If needed, an explanation for the counts can be provided here.}

\section{CICD}

\wss{Say how CICD is used in your project}

\wss{If your team has additional metrics of productivity, please feel free to
add them to this report.}

\end{document}