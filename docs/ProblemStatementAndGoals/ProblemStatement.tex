\documentclass{article}

\usepackage{tabularx}
\usepackage{booktabs}

\title{Problem Statement and Goals\\ GenreGuru}

\author{\authname}

\date{}

%% Comments

\usepackage{color}

\newif\ifcomments\commentstrue %displays comments
%\newif\ifcomments\commentsfalse %so that comments do not display

\ifcomments
\newcommand{\authornote}[3]{\textcolor{#1}{[#3 ---#2]}}
\newcommand{\todo}[1]{\textcolor{red}{[TODO: #1]}}
\else
\newcommand{\authornote}[3]{}
\newcommand{\todo}[1]{}
\fi

\newcommand{\wss}[1]{\authornote{blue}{SS}{#1}} 
\newcommand{\plt}[1]{\authornote{magenta}{TPLT}{#1}} %For explanation of the template
\newcommand{\an}[1]{\authornote{cyan}{Author}{#1}}

%% Common Parts

\newcommand{\progname}{Software Engineering} % PUT YOUR PROGRAM NAME HERE
\newcommand{\authname}{Team 8 -- Rhythm Rangers\\
\\ Ansel Chen
\\ Muhammad Jawad
\\ Mohamad-Hassan Bahsoun
\\ Matthew Baleanu
\\ Ahmed Al-Hayali} % AUTHOR NAMES                  

\usepackage{hyperref}
    \hypersetup{colorlinks=true, linkcolor=blue, citecolor=blue, filecolor=blue,
                urlcolor=blue, unicode=false}
    \urlstyle{same}
                                


\begin{document}

\maketitle

\begin{table}[hp]
\caption{Revision History} \label{TblRevisionHistory}
\begin{tabularx}{\textwidth}{llX}
\toprule
\textbf{Date} & \textbf{Developer(s)} & \textbf{Change}\\
\midrule
Date1 & Name(s) & Description of changes\\
Date2 & Name(s) & Description of changes\\
... & ... & ...\\
\bottomrule
\end{tabularx}
\end{table}

\section{Problem Statement}

\wss{You should check your problem statement with the
\href{https://github.com/smiths/capTemplate/blob/main/docs/Checklists/ProbState-Checklist.pdf}
{problem statement checklist}.} 

\wss{You can change the section headings, as long as you include the required
information.}

\subsection{Problem}

\subsection{Inputs and Outputs}

\wss{Characterize the problem in terms of ``high level'' inputs and outputs.  
Use abstraction so that you can avoid details.}\\

There are two main components of this project. They are both generative systems, one being a recommendation
system, the other being a pseudo-music generator. There is also a song analysis system, necessary in
order to get the recommendation system and generative systems working as they will most likely operate on 
the data that the song analysis system provides. \\

Inputs:\\
A "track" is defined as a full song. A snippet is defined as a fragment of a track. These are objects that 
contain a sound file and some labels describing the song. 
High-Level: 
For system 1 (recommendation): a list of track(s). \\
For system 2 (music generation): a reference track or snippet. \\
For system 3 (analysis): a singular reference track. \\

Explanation: The main idea behind the inputs is essentially to make them all related to each other as they are
fundamentally connected within the system. This means that roughly, all the files are either lists containing
the track as an object and a snippet is functionally under the hood a track, simply labelled differently for
categorization purposes. This allows the system to share behaviors along the sub systems for 
faster implementation and design. As for format specifically, we can pull these from the spotify API song previews,
API song features. \\

Outputs: \\
For System 1 (recommendation): A list of track(s), indexed with a similarity score. \\
For System 2 (music generation): a snippet. \\
For System 3 (analysis): a breakdown of the main aspects of the track.\\

Explanation: for system 1, it should output a list of recommended tracks and some form of a similarity score
as a way to justify itself. This score could be derived on factors such as genre, sound signature, artist, etc.

For system 2, the snippest is an object containing the sound file. 

For system 3, a score that categorizes the main features of the song, such as the musical range, genre, rythm, etc. \\

\subsection{Stakeholders}

\subsection{Environment}

\wss{Hardware and software environment}

\section{Goals}
% Song Analysis Section
\subsubsection*{Song Analysis}
\begin{itemize}
    \item \textbf{Explanation:} GenreGurus will analyze a song or snippet and extract important musical features. The data gathered from this analysis will then be fed into the recommendation and generation systems, ensuring they work with accurate data about each song's musical features.
    \item \textbf{Reasoning:} By providing a detailed analysis, users can better understand the components of a given song. The analysis also ensures that the recommendations and generated music are based on actual musical data, which in turn improves the system’s accuracy and output quality.
\end{itemize}

% Music Generation Section
\subsubsection*{Music Generation}
\begin{itemize}
    \item \textbf{Explanation:} GenreGurus will allow users to input one or more reference songs or snippets and generate new music based on the features of these inputs. Users can adjust certain musical characteristics, and the system will produce an original track reflecting these changes.
    \item \textbf{Reasoning:} Users will be able to create and customize music through AI without needing extensive musical knowledge, making the platform more accessible while still appealing to expert musicians.
\end{itemize}

% User Customizable Recommendations Section
\subsubsection*{User Customizable Recommendations}
\begin{itemize}
    \item \textbf{Explanation:} GenreGurus will allow users to adjust the musical features of the input clip/song. Based on these adjustments, the system will update its recommendations in real-time.
    \item \textbf{Reasoning:} Customizable recommendations give users more control over the output, increasing user engagement and user satisfaction.
\end{itemize}

% User Centric Design and Interface Section
\subsubsection*{User Centric Design and Interface}
\begin{itemize}
    \item \textbf{Explanation:} GenreGurus will include a clean, intuitive interface where users can easily access the music recommendation, generation, and analysis features. The UI will be designed to require minimal understanding of how music adjustments work for a better user experience.
    \item \textbf{Reasoning:} An accessible and simple interface will appeal to a broader audience, from casual listeners to professionals.
\end{itemize}

% Supportive of Popular Music Genres Section
\subsubsection*{Supportive of Many Music Genres}
\begin{itemize}
    \item \textbf{Explanation:} GenreGurus will include a variety of popular genres of music.
    \item \textbf{Reasoning:} This will allow users to customize and explore their favorite genres of music.
\end{itemize}


\section{Stretch Goals}
% Machine Learning Cover Art Generation Section
\subsubsection*{Machine Learning Cover Art Generation}
\begin{itemize}
    \item \textbf{Explanation:} GenreGurus will use AI models to generate custom art based on the features of a generated song or user preferences. The art will visually reflect the song's mood, genre, style, etc.
    \item \textbf{Reasoning:} Music encompasses more than just using one's auditory senses; it is also a visual and emotional experience. By generating art that matches the music, the system offers more connection for creators, appealing to both their auditory and visual senses.
\end{itemize}

% Supportive of Niche Music Genres Section
\subsubsection*{Supportive of Many Music Genres}
\begin{itemize}
    \item \textbf{Explanation:} GenreGurus will explore a variety niche genres.
    \item \textbf{Reasoning:} This will allow users to explore genres they’ve never heard or experienced before. Which will engage the users more.
\end{itemize}

\section{Challenge Level and Extras}

\wss{State your expected challenge level (advanced, general or basic).  The
challenge can come through the required domain knowledge, the implementation or
something else.  Usually the greater the novelty of a project the greater its
challenge level.  You should include your rationale for the selected level.
Approval of the level will be part of the discussion with the instructor forx
approving the project.  The challenge level, with the approval (or request) of
the instructor, can be modified over the course of the term.}

\wss{Teams may wish to include extras as either potential bonus grades, or to
make up for a less advanced challenge level.  Potential extras include usability
testing, code walkthroughs, user documentation, formal proof, GenderMag
personas, Design Thinking, etc.  Normally the maximum number of extras will be
two.  Approval of the extras will be part of the discussion with the instructor
for approving the project.  The extras, with the approval (or request) of the
instructor, can be modified over the course of the term.}

\newpage{}

\section*{Appendix --- Reflection}

\wss{Not required for CAS 741}

The purpose of reflection questions is to give you a chance to assess your own
learning and that of your group as a whole, and to find ways to improve in the
future. Reflection is an important part of the learning process.  Reflection is
also an essential component of a successful software development process.  

Reflections are most interesting and useful when they're honest, even if the
stories they tell are imperfect. You will be marked based on your depth of
thought and analysis, and not based on the content of the reflections
themselves. Thus, for full marks we encourage you to answer openly and honestly
and to avoid simply writing ``what you think the evaluator wants to hear.''

Please answer the following questions.  Some questions can be answered on the
team level, but where appropriate, each team member should write their own
response:


\begin{enumerate}
    \item What went well while writing this deliverable? 
    \item What pain points did you experience during this deliverable, and how
    did you resolve them?
    
\emph{Ansel}
The main disagreements we had were the project selection. To resolve this, we used a project selection matrix 
where we rated the proposed projects by a large amount of criteria and then discussed which projects we felt
were the most interesting, feasible, or doable. We ended up eliminating some projects due to this. We then proceeded to use a strawpoll in order to sort out which
remaining projects were our top choices, and which ones were projects where there was 1 team member who simply was not
interested in at all. This lead us to select the GenreGuru project. The documentation allowed us to neatly organize our
thoughts on each project and to compare them in the most fair and honest method possible. 

We also predicted that editing a latex
document and then resolving conflict through github could be potentially an issue, which we resolved by having each member 
edit one section of the document and then using pull requests to review our changes before having one person handling merging
all our edits into the final .tex file. This helped us stay organized and not cause headaches with merge conflicts. 

    did you resolve them? \\ \\
    \emph{Bahsoun}
    During this deliverable, we anticipated challenges, so we approached the task by dividing the workload among our team. Each member volunteered to focus on a specific section while contributing thoughts during collaborative discussions. After completing our individual sections and reviewing each other's work, we convened for a call to share our feedback on what we liked and what needed adjustments. Overall, while the deliverable went smoothly, we did encounter some pain points. One critical challenge was selecting the right technologies, particularly in determining which frameworks and machine learning libraries to use. To tackle this, each member conducted research to evaluate the strengths of various options, allowing us to narrow down our choices effectively. Another challenge arose from team members' busy schedules, making it difficult to coordinate in-person meetings. To overcome this, we prioritized communication throughout the week, ensuring everyone stayed informed. Instead of requiring everyone to attend every meeting, we arranged for a few members to attend and then relay key points to those who couldn't make it. This approach helped keep the entire team aligned and engaged.

    \item How did you and your team adjust the scope of your goals to ensure
    they are suitable for a Capstone project (not overly ambitious but also of
    appropriate complexity for a senior design project)?
\end{enumerate}  

\end{document}