\documentclass{article}

\usepackage{tabularx}
\usepackage{booktabs}

\title{Problem Statement and Goals\\\progname}

\author{\authname}

\date{}

%% Comments

\usepackage{color}

\newif\ifcomments\commentstrue %displays comments
%\newif\ifcomments\commentsfalse %so that comments do not display

\ifcomments
\newcommand{\authornote}[3]{\textcolor{#1}{[#3 ---#2]}}
\newcommand{\todo}[1]{\textcolor{red}{[TODO: #1]}}
\else
\newcommand{\authornote}[3]{}
\newcommand{\todo}[1]{}
\fi

\newcommand{\wss}[1]{\authornote{blue}{SS}{#1}} 
\newcommand{\plt}[1]{\authornote{magenta}{TPLT}{#1}} %For explanation of the template
\newcommand{\an}[1]{\authornote{cyan}{Author}{#1}}

%% Common Parts

\newcommand{\progname}{Software Engineering} % PUT YOUR PROGRAM NAME HERE
\newcommand{\authname}{Team 8 -- Rhythm Rangers\\
\\ Ansel Chen
\\ Muhammad Jawad
\\ Mohamad-Hassan Bahsoun
\\ Matthew Baleanu
\\ Ahmed Al-Hayali} % AUTHOR NAMES                  

\usepackage{hyperref}
    \hypersetup{colorlinks=true, linkcolor=blue, citecolor=blue, filecolor=blue,
                urlcolor=blue, unicode=false}
    \urlstyle{same}
                                


\begin{document}

\maketitle

\begin{table}[hp]
\caption{Revision History} \label{TblRevisionHistory}
\begin{tabularx}{\textwidth}{llX}
\toprule
\textbf{Date} & \textbf{Developer(s)} & \textbf{Change}\\
\midrule
Date1 & Name(s) & Description of changes\\
Date2 & Name(s) & Description of changes\\
... & ... & ...\\
\bottomrule
\end{tabularx}
\end{table}

\section{Problem Statement}

\wss{You should check your problem statement with the
\href{https://github.com/smiths/capTemplate/blob/main/docs/Checklists/ProbState-Checklist.pdf}
{problem statement checklist}.} 

\wss{You can change the section headings, as long as you include the required
information.}

\subsection{Problem}

\subsection{Inputs and Outputs}

\wss{Characterize the problem in terms of ``high level'' inputs and outputs.  
Use abstraction so that you can avoid details.}

\subsection{Stakeholders}

\subsection{Environment}
We strive to launch an on-premise server operating with a version of Ubuntu server, likely the most recent version, \href{https://ubuntu.com/download/server}{24.04.1}. The server shall respond to and process requests from a web-application front-end, making the service accessible to many different devices, but requiring a network.

\section{Goals}
\begin{table}[h]
    \centering
    \begin{tabular}{|| p{0.45\textwidth} | p{0.45\textwidth} ||}
        \hline
        \textbf{Goals} & \textbf{Importance} \\
        \hline
        The system shall \textcolor{red}{adequately} process and respond to 
        requests involving widely-published music genres, e.g., pop, hip-hop, 
        and rock. & Widely-published music genres have the largest corpus of 
        data that can be used to train the featurization and generation 
        mechanisms of the system, i.e., the system must perform favourably in 
        tasks that it is well-trained on. \\
        \hline
        The system shall generate tabular features that correspond to characteristics 
        of the input song (snippet), akin to those of Spotify, e.g., danceability, 
        instrumentalness, and energy. & Structured tabular data can be rapidly process, 
        making the task of song recommendation more efficient and song generation 
        more explainable. \\
        \hline
        The system shall produce a list of songs that are \textcolor{red}{similar} 
        to a single song provided or a collecetion of songs provided by the user. & 
        This is a core feature of the system. Its inclusion should facilitate users 
        to explore a music genre or \textcolor{red}{``sound''} of interest. \\
        \hline
    \end{tabular}
\end{table}

\section{Stretch Goals}
\begin{table}[h]
    \centering
    \begin{tabular}{|| p{0.45\textwidth} | p{0.45\textwidth} ||}
        \hline
        \textbf{Goals} & \textbf{Importance} \\
        \hline
        The system shall \textcolor{red}{adequately} process and respond to 
        requests involving \emph{not-as-widely}-published music genres, e.g., 
        jazz, funk, and blues. & Such music genres have a smaller corpus of 
        data that can be used for training, hence the system may not perform 
        as favourably in tasks that it is not very well-trained on, but the 
        inclusion of such genres would allow access to a larger user-group. \\
        \hline
        The system shall generate tabular features that correspond to 
        characteristics of the input song's \emph{cover art}. & Cover art tends 
        to capture, however abstractly, the mood, energy, and intent of a song 
        or album, thus may contain tacit information that can be accessed with 
        image processing. \\
        \hline
    \end{tabular}
\end{table}

\section{Challenge Level and Extras}

The project is of a \emph{general} challenge level.
\begin{itemize}
    \item It requires domain knowledge about signal (audio) processing, music
    theory, learning models, generative models, and infrastructure setup.
    \item Its implementation is non-trivial, incorporating algorithm implementations, 
    training and testing models, assessing their performance, automating the extraction-
    processing-storage workflow and the live-response workflow.
    \item The system is not particularly novel. Recommender systems are not new, but we are 
    attempting to find and use features to create a better recommender system. The generative 
    component has been done before with images and video, so scaling down to audio and frequency 
    should be attainable, especially as it is a field that was researched quite deeply even before 
    the advent of neural network-based generative techniques.
\end{itemize}

\noindent
Project will include extras like user \& API Documentation for ease of reference, usability testing 
for easy startup, and design thinking to build an intuitive user interface.

\newpage{}

\section*{Appendix --- Reflection}

\wss{Not required for CAS 741}

The purpose of reflection questions is to give you a chance to assess your own
learning and that of your group as a whole, and to find ways to improve in the
future. Reflection is an important part of the learning process.  Reflection is
also an essential component of a successful software development process.  

Reflections are most interesting and useful when they're honest, even if the
stories they tell are imperfect. You will be marked based on your depth of
thought and analysis, and not based on the content of the reflections
themselves. Thus, for full marks we encourage you to answer openly and honestly
and to avoid simply writing ``what you think the evaluator wants to hear.''

Please answer the following questions.  Some questions can be answered on the
team level, but where appropriate, each team member should write their own
response:


\begin{enumerate}
    \item What went well while writing this deliverable? 
    \item What pain points did you experience during this deliverable, and how
    did you resolve them?
    \item How did you and your team adjust the scope of your goals to ensure
    they are suitable for a Capstone project (not overly ambitious but also of
    appropriate complexity for a senior design project)?
\end{enumerate}  

\end{document}