\documentclass{article}

\usepackage{tabularx}
\usepackage{booktabs}
\usepackage{amsfonts}

\title{Problem Statement and Goals\\\progname}

\author{\authname}

\date{}

%% Comments

\usepackage{color}

\newif\ifcomments\commentstrue %displays comments
%\newif\ifcomments\commentsfalse %so that comments do not display

\ifcomments
\newcommand{\authornote}[3]{\textcolor{#1}{[#3 ---#2]}}
\newcommand{\todo}[1]{\textcolor{red}{[TODO: #1]}}
\else
\newcommand{\authornote}[3]{}
\newcommand{\todo}[1]{}
\fi

\newcommand{\wss}[1]{\authornote{blue}{SS}{#1}} 
\newcommand{\plt}[1]{\authornote{magenta}{TPLT}{#1}} %For explanation of the template
\newcommand{\an}[1]{\authornote{cyan}{Author}{#1}}

%% Common Parts

\newcommand{\progname}{Software Engineering} % PUT YOUR PROGRAM NAME HERE
\newcommand{\authname}{Team 8 -- Rhythm Rangers\\
\\ Ansel Chen
\\ Muhammad Jawad
\\ Mohamad-Hassan Bahsoun
\\ Matthew Baleanu
\\ Ahmed Al-Hayali} % AUTHOR NAMES                  

\usepackage{hyperref}
    \hypersetup{colorlinks=true, linkcolor=blue, citecolor=blue, filecolor=blue,
                urlcolor=blue, unicode=false}
    \urlstyle{same}
                                


\begin{document}

\maketitle

\begin{table}[hp]
\caption{Revision History} \label{TblRevisionHistory}
\begin{tabularx}{\textwidth}{llX}
\toprule
\textbf{Date} & \textbf{Developer(s)} & \textbf{Change}\\
\midrule
2024-09-21 & Muhammad Jawad & Updated Problem section with Problem Definition and Problem Importance subsections.\\
Date2 & Name(s) & Description of changes\\
... & ... & ...\\
\bottomrule
\end{tabularx}
\end{table}

\section{Problem Statement}

\subsection{Problem}

\subsubsection{Problem Definition}

Music is an effortless art to consume, but with a laborious and inaccessible creation process. It demands extensive proficiency in playing instruments, the use of complex music production software, e.g., FL Studio or Cakewalk, the use of mixing tools and techniques, e.g., level faders and equalization, among other barriers to entry. GenreGuru strives to democratize experimentation in music production, making it accessible to beginners and more streamlined for professionals by automatically generating songs or snippets inspired by user-inputted songs or snippets. In support of the generative system, a music analysis tool can be used to summarize songs using tabular features, akin to Spotify's ``danceability'', for example. Finally, a catalogue of songs and their corresponding features can be accessed to recommend users songs that share similar features to ones they input.

\subsubsection{Problem Importance}

This project holds significant value for a wide range of users. Music recommendation systems have millions of daily users, and improving the quality of recommendations can greatly enhance user satisfaction. Additionally, by providing tools that make music creation more accessible, the project opens creative opportunities for individuals who might otherwise be excluded due to the technical barriers involved in music production. The platform is poised to impact both the music industry and hobbyists, facilitating innovation and creativity in music generation and exploration.

\subsection{Inputs and Outputs}

The project can be characterized by three systems: \emph{music generation}, \emph{music analysis}, and \emph{music recommendation}. For completeness, a song is a full track, i.e., a completed audio file published by a musician, and a snippet is a fragment of a track, i.e., an incomplete audio file. Such audio files can be in a \emph{lossy} compressed format, e.g., \texttt{.mp3}, a \emph{lossless} compressed format, e.g., \texttt{.FLAC}, or an uncompressed format, e.g., \texttt{.WAV}. While the file formats of audio files are known, the source of a song or song snippet can be anything. A concrete example is a 30-second snippet from Spotify's API being an MP3 file delivered through the Spotify content delivery network, \texttt{scdn}, e.g., \href{https://p.scdn.co/mp3-preview/c05a687254dbdf50a9ab4879d85e54a7594e367f?cid=945b9ed74f4c43b49bcf26e7dc388df0}{Jack Harlow's ``First Class''}.


\subsubsection{Inputs}

\begin{itemize}
    \item \emph{Music generation:} reference song(s) and/or song snippet(s).
    \item \emph{Music analysis:} reference song(s) and/or song snippet(s).
    \item \emph{Music recommendation:} reference song(s).
\end{itemize}

\subsubsection{Outputs}
\begin{itemize}
    \item \emph{Music generation:} generated song or song snippet, likely \texttt{.MP3} files.
    \item \emph{Music analysis:}
    \begin{itemize}
        \item A collection $\in \mathbb{R}^k$ of features, e.g., pitch, timbre, or ``danceability''.
        \item Visualizations of song characteristics, e.g., spectrograms and spectral density estimates.
    \end{itemize}
    \item \emph{Music recommendation:} collection of references to songs, e.g., string names of songs or URLs to music providers. Potentially sortable by likeness to the input reference songs. Please note the distinction between \emph{references to songs} and \emph{reference songs}.
\end{itemize}

\subsection{Stakeholders}

The stakeholders for this project include:
\begin{itemize}
    \item \textbf{Music Producers}: Professionals who are looking to generate new ideas, experiment with different genres, and augment their existing works.
    \item \textbf{Hobbyist Musicians}: Individuals interested in exploring and tinkering with familiar sounds and experimenting with new, creative compositions.
    \item \textbf{Music Theorists}: Users who seek to study and analyze different musical elements (e.g., pitch, rhythm) as a part of their research or hobby.
    \item \textbf{Audio Engineers}: Audio experts who can use the system to better understand sound characteristics and possibly optimize their workflows.
    \item \textbf{Independent Artists and Musicians}: Artists looking for accessible and affordable tools to experiment with and create music.
    \item \textbf{Music Educators}: Teachers seeking innovative ways to introduce students to music theory and composition.
    \item \textbf{Content Creators and Streamers}: Creators in need of unique music or soundtracks for their content.
    \item \textbf{Sound Designers}: Professionals working in film, games, or other industries where custom soundscapes and music are needed.
    \item \textbf{DJs}: DJs who want to generate new tracks and remixes based on existing music for performances.
    \item \textbf{Music App Developers}: Developers looking to integrate music recommendation and generation into their own apps.
    \item \textbf{Music Therapy Practitioners}: Therapists using music to help clients and generate tailored soundscapes for treatment.
    \item \textbf{Casual Music Listeners}: Users who want to discover and generate music for personal enjoyment.
    \item \textbf{Record Labels}: Companies interested in discovering new trends and helping artists explore experimental genres.
\end{itemize}


\subsection{Environment}
We strive to launch an on-premise server operating with a version of Ubuntu server, likely the most recent version, \href{https://ubuntu.com/download/server}{24.04.1}. The server shall respond to and process requests from a web-application front-end, making the service accessible to many different devices, but requiring a network.

\section{Goals}
\begin{table}[h]
    \centering
    \begin{tabular}{|| p{0.45\textwidth} | p{0.45\textwidth} ||}
        \hline
        \textbf{Goals} & \textbf{Importance} \\
        \hline
        The system shall \textcolor{red}{adequately} process and respond to 
        requests involving widely-published music genres, e.g., pop, hip-hop, 
        and rock. & Widely-published music genres have the largest corpus of 
        data that can be used to train the featurization and generation 
        mechanisms of the system, i.e., the system must perform favourably in 
        tasks that it is well-trained on. \\
        \hline
        The system shall generate tabular features that correspond to characteristics 
        of the input song (snippet), akin to those of Spotify, e.g., danceability, 
        instrumentalness, and energy. & Structured tabular data can be rapidly process, 
        making the task of song recommendation more efficient and song generation 
        more explainable. \\
        \hline
        The system shall produce a list of songs that are \textcolor{red}{similar} 
        to a single song provided or a collecetion of songs provided by the user. & 
        This is a core feature of the system. Its inclusion should facilitate users 
        to explore a music genre or \textcolor{red}{``sound''} of interest. \\
        \hline
    \end{tabular}
\end{table}

% Song Analysis Section
\subsubsection*{Song Analysis}
\begin{itemize}
    \item \textbf{Explanation:} GenreGurus will analyze a song or snippet and extract important musical features. The data gathered from this analysis will then be fed into the recommendation and generation systems, ensuring they work with accurate data about each song's musical features.
    \item \textbf{Reasoning:} By providing a detailed analysis, users can better understand the components of a given song. The analysis also ensures that the recommendations and generated music are based on actual musical data, which in turn improves the system’s accuracy and output quality.
\end{itemize}

% Music Generation Section
\subsubsection*{Music Generation}
\begin{itemize}
    \item \textbf{Explanation:} GenreGurus will allow users to input one or more reference songs or snippets and generate new music based on the features of these inputs. Users can adjust certain musical characteristics, and the system will produce an original track reflecting these changes.
    \item \textbf{Reasoning:} Users will be able to create and customize music through AI without needing extensive musical knowledge, making the platform more accessible while still appealing to expert musicians.
\end{itemize}

% User Customizable Recommendations Section
\subsubsection*{User Customizable Recommendations}
\begin{itemize}
    \item \textbf{Explanation:} GenreGurus will allow users to adjust the musical features of the input clip/song. Based on these adjustments, the system will update its recommendations in real-time.
    \item \textbf{Reasoning:} Customizable recommendations give users more control over the output, increasing user engagement and user satisfaction.
\end{itemize}

% User Centric Design and Interface Section
\subsubsection*{User Centric Design and Interface}
\begin{itemize}
    \item \textbf{Explanation:} GenreGurus will include a clean, intuitive interface where users can easily access the music recommendation, generation, and analysis features. The UI will be designed to require minimal understanding of how music adjustments work for a better user experience.
    \item \textbf{Reasoning:} An accessible and simple interface will appeal to a broader audience, from casual listeners to professionals.
\end{itemize}

% Supportive of Popular Music Genres Section
\subsubsection*{Supportive of Many Music Genres}
\begin{itemize}
    \item \textbf{Explanation:} GenreGurus will include a variety of popular genres of music.
    \item \textbf{Reasoning:} This will allow users to customize and explore their favorite genres of music.
\end{itemize}


\section{Stretch Goals}
\begin{table}[h]
    \centering
    \begin{tabular}{|| p{0.45\textwidth} | p{0.45\textwidth} ||}
        \hline
        \textbf{Goals} & \textbf{Importance} \\
        \hline
        The system shall \textcolor{red}{adequately} process and respond to 
        requests involving \emph{not-as-widely}-published music genres, e.g., 
        jazz, funk, and blues. & Such music genres have a smaller corpus of 
        data that can be used for training, hence the system may not perform 
        as favourably in tasks that it is not very well-trained on, but the 
        inclusion of such genres would allow access to a larger user-group. \\
        \hline
        The system shall generate tabular features that correspond to 
        characteristics of the input song's \emph{cover art}. & Cover art tends 
        to capture, however abstractly, the mood, energy, and intent of a song 
        or album, thus may contain tacit information that can be accessed with 
        image processing. \\
        \hline
    \end{tabular}
\end{table}

% Machine Learning Cover Art Generation Section
\subsubsection*{Machine Learning Cover Art Generation}
\begin{itemize}
    \item \textbf{Explanation:} GenreGurus will use AI models to generate custom art based on the features of a generated song or user preferences. The art will visually reflect the song's mood, genre, style, etc.
    \item \textbf{Reasoning:} Music encompasses more than just using one's auditory senses; it is also a visual and emotional experience. By generating art that matches the music, the system offers more connection for creators, appealing to both their auditory and visual senses.
\end{itemize}

% Supportive of Niche Music Genres Section
\subsubsection*{Supportive of Many Music Genres}
\begin{itemize}
    \item \textbf{Explanation:} GenreGurus will explore a variety niche genres.
    \item \textbf{Reasoning:} This will allow users to explore genres they’ve never heard or experienced before. Which will engage the users more.
\end{itemize}

\section{Challenge Level and Extras}

The project is of a \emph{general} challenge level.
\begin{itemize}
    \item It requires domain knowledge about signal (audio) processing, music
    theory, learning models, generative models, and infrastructure setup.
    \item Its implementation is non-trivial, incorporating algorithm implementations, 
    training and testing models, assessing their performance, automating the extraction-
    processing-storage workflow and the live-response workflow.
    \item The system is not particularly novel. Recommender systems are not new, but we are 
    attempting to find and use features to create a better recommender system. The generative 
    component has been done before with images and video, so scaling down to audio and frequency 
    should be attainable, especially as it is a field that was researched quite deeply even before 
    the advent of neural network-based generative techniques.
\end{itemize}

\noindent
Project will include extras like user \& API Documentation for ease of reference, usability testing 
for easy startup, and design thinking to build an intuitive user interface.

\newpage{}

\section*{Appendix --- Reflection}

\wss{Not required for CAS 741}

The purpose of reflection questions is to give you a chance to assess your own
learning and that of your group as a whole, and to find ways to improve in the
future. Reflection is an important part of the learning process.  Reflection is
also an essential component of a successful software development process.  

Reflections are most interesting and useful when they're honest, even if the
stories they tell are imperfect. You will be marked based on your depth of
thought and analysis, and not based on the content of the reflections
themselves. Thus, for full marks we encourage you to answer openly and honestly
and to avoid simply writing ``what you think the evaluator wants to hear.''

Please answer the following questions.  Some questions can be answered on the
team level, but where appropriate, each team member should write their own
response:


\begin{enumerate}
    \item What went well while writing this deliverable? 
    \item What pain points did you experience during this deliverable, and how
    did you resolve them?
    
\emph{Ansel}
The main disagreements we had were the project selection. To resolve this, we used a project selection matrix 
where we rated the proposed projects by a large amount of criteria and then discussed which projects we felt
were the most interesting, feasible, or doable. We ended up eliminating some projects due to this. We then proceeded to use a strawpoll in order to sort out which
remaining projects were our top choices, and which ones were projects where there was 1 team member who simply was not
interested in at all. This lead us to select the GenreGuru project. The documentation allowed us to neatly organize our
thoughts on each project and to compare them in the most fair and honest method possible. 

We also predicted that editing a latex
document and then resolving conflict through github could be potentially an issue, which we resolved by having each member 
edit one section of the document and then using pull requests to review our changes before having one person handling merging
all our edits into the final .tex file. This helped us stay organized and not cause headaches with merge conflicts. 

    did you resolve them? \\ \\
    \emph{Bahsoun}
    During this deliverable, we anticipated challenges, so we approached the task by dividing the workload among our team. Each member volunteered to focus on a specific section while contributing thoughts during collaborative discussions. After completing our individual sections and reviewing each other's work, we convened for a call to share our feedback on what we liked and what needed adjustments. Overall, while the deliverable went smoothly, we did encounter some pain points. One critical challenge was selecting the right technologies, particularly in determining which frameworks and machine learning libraries to use. To tackle this, each member conducted research to evaluate the strengths of various options, allowing us to narrow down our choices effectively. Another challenge arose from team members' busy schedules, making it difficult to coordinate in-person meetings. To overcome this, we prioritized communication throughout the week, ensuring everyone stayed informed. Instead of requiring everyone to attend every meeting, we arranged for a few members to attend and then relay key points to those who couldn't make it. This approach helped keep the entire team aligned and engaged.

    \item How did you and your team adjust the scope of your goals to ensure
    they are suitable for a Capstone project (not overly ambitious but also of
    appropriate complexity for a senior design project)?
\end{enumerate}  

\end{document}