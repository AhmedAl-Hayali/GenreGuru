\documentclass{article}

\usepackage{tabularx}
\usepackage{booktabs}

\title{Problem Statement and Goals\\\progname}

\author{\authname}

\date{}

\input{../Comments}
%% Common Parts

\newcommand{\progname}{GenreGuru} % PUT YOUR PROGRAM NAME HERE
\newcommand{\authname}{Team 8 -- Rhythm Rangers\\
\\ Ansel Chen
\\ Muhammad Jawad
\\ Mohamad-Hassan Bahsoun
\\ Matthew Baleanu
\\ Ahmed Al-Hayali} % AUTHOR NAMES                  

\usepackage{hyperref}
    \hypersetup{colorlinks=true, linkcolor=blue, citecolor=blue, filecolor=blue,
                urlcolor=blue, unicode=false}
    \urlstyle{same}
                                


\begin{document}

\maketitle

\begin{table}[hp]
\caption{Revision History} \label{TblRevisionHistory}
\begin{tabularx}{\textwidth}{llX}
\toprule
\textbf{Date} & \textbf{Developer(s)} & \textbf{Change}\\
\midrule
2024-09-21 & Muhammad Jawad & Updated Problem section with Problem Definition and Problem Importance subsections.\\
Date2 & Name(s) & Description of changes\\
... & ... & ...\\
\bottomrule
\end{tabularx}
\end{table}

\section{Problem Statement}

\wss{You should check your problem statement with the
\href{https://github.com/smiths/capTemplate/blob/main/docs/Checklists/ProbState-Checklist.pdf}
{problem statement checklist}.} 

\wss{You can change the section headings, as long as you include the required
information.}

\subsection{Problem}

\subsubsection{Problem Definition}

Current music recommendation systems, such as Spotify, often lack personalization and depth, leading to suboptimal suggestions. Furthermore, the ability to create music typically requires extensive knowledge of instruments, production software, or technical tools, limiting accessibility. This project aims to bridge that gap by offering a platform that allows users to receive tailored music recommendations and generate music without needing specialized training. By simplifying the creation and recommendation process, this platform democratizes music production for both professionals and casual users.

\subsubsection{Problem Importance}

This project holds significant value for a wide range of users. Music recommendation systems have millions of daily users, and improving the quality of recommendations can greatly enhance user satisfaction. Additionally, by providing tools that make music creation more accessible, the project opens creative opportunities for individuals who might otherwise be excluded due to the technical barriers involved in music production. The platform is poised to impact both the music industry and hobbyists, facilitating innovation and creativity in music generation and exploration.

\subsection{Inputs and Outputs}

\wss{Characterize the problem in terms of ``high level'' inputs and outputs.  
Use abstraction so that you can avoid details.}

\subsection{Stakeholders}

The stakeholders for this project include:
\begin{itemize}
    \item \textbf{Music Producers}: Professionals who are looking to generate new ideas, experiment with different genres, and augment their existing works.
    \item \textbf{Hobbyist Musicians}: Individuals interested in exploring and tinkering with familiar sounds and experimenting with new, creative compositions.
    \item \textbf{Music Theorists}: Users who seek to study and analyze different musical elements (e.g., pitch, rhythm) as a part of their research or hobby.
    \item \textbf{Audio Engineers}: Audio experts who can use the system to better understand sound characteristics and possibly optimize their workflows.
    \item \textbf{Independent Artists and Musicians}: Artists looking for accessible and affordable tools to experiment with and create music.
    \item \textbf{Music Educators}: Teachers seeking innovative ways to introduce students to music theory and composition.
    \item \textbf{Content Creators and Streamers}: Creators in need of unique music or soundtracks for their content.
    \item \textbf{Sound Designers}: Professionals working in film, games, or other industries where custom soundscapes and music are needed.
    \item \textbf{DJs}: DJs who want to generate new tracks and remixes based on existing music for performances.
    \item \textbf{Music App Developers}: Developers looking to integrate music recommendation and generation into their own apps.
    \item \textbf{Music Therapy Practitioners}: Therapists using music to help clients and generate tailored soundscapes for treatment.
    \item \textbf{Casual Music Listeners}: Users who want to discover and generate music for personal enjoyment.
    \item \textbf{Record Labels}: Companies interested in discovering new trends and helping artists explore experimental genres.
\end{itemize}


\subsection{Environment}

\wss{Hardware and software environment}

\section{Goals}

\section{Stretch Goals}

\section{Challenge Level and Extras}

\wss{State your expected challenge level (advanced, general or basic).  The
challenge can come through the required domain knowledge, the implementation or
something else.  Usually the greater the novelty of a project the greater its
challenge level.  You should include your rationale for the selected level.
Approval of the level will be part of the discussion with the instructor for
approving the project.  The challenge level, with the approval (or request) of
the instructor, can be modified over the course of the term.}

\wss{Teams may wish to include extras as either potential bonus grades, or to
make up for a less advanced challenge level.  Potential extras include usability
testing, code walkthroughs, user documentation, formal proof, GenderMag
personas, Design Thinking, etc.  Normally the maximum number of extras will be
two.  Approval of the extras will be part of the discussion with the instructor
for approving the project.  The extras, with the approval (or request) of the
instructor, can be modified over the course of the term.}

\newpage{}

\section*{Appendix --- Reflection}

\wss{Not required for CAS 741}

\input{../Reflection.tex}

\begin{enumerate}
    \item What went well while writing this deliverable? 
    \item What pain points did you experience during this deliverable, and how
    did you resolve them?
    \item How did you and your team adjust the scope of your goals to ensure
    they are suitable for a Capstone project (not overly ambitious but also of
    appropriate complexity for a senior design project)?
\end{enumerate}  

\end{document}