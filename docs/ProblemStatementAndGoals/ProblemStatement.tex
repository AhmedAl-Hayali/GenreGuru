\documentclass{article}

\usepackage{tabularx}
\usepackage{booktabs}

\title{Problem Statement and Goals\\\progname}

\author{\authname}

\date{}

\input{../Comments}
%% Common Parts

\newcommand{\progname}{GenreGuru} % PUT YOUR PROGRAM NAME HERE
\newcommand{\authname}{Team 8 -- Rhythm Rangers\\
\\ Ansel Chen
\\ Muhammad Jawad
\\ Mohamad-Hassan Bahsoun
\\ Matthew Baleanu
\\ Ahmed Al-Hayali} % AUTHOR NAMES                  

\usepackage{hyperref}
    \hypersetup{colorlinks=true, linkcolor=blue, citecolor=blue, filecolor=blue,
                urlcolor=blue, unicode=false}
    \urlstyle{same}
                                


\begin{document}

\maketitle

\begin{table}[hp]
\caption{Revision History} \label{TblRevisionHistory}
\begin{tabularx}{\textwidth}{llX}
\toprule
\textbf{Date} & \textbf{Developer(s)} & \textbf{Change}\\
\midrule
Date1 & Name(s) & Description of changes\\
Date2 & Name(s) & Description of changes\\
... & ... & ...\\
\bottomrule
\end{tabularx}
\end{table}

\section{Problem Statement}

\wss{You should check your problem statement with the
\href{https://github.com/smiths/capTemplate/blob/main/docs/Checklists/ProbState-Checklist.pdf}
{problem statement checklist}.} 

\wss{You can change the section headings, as long as you include the required
information.}

\subsection{Problem}

\subsection{Inputs and Outputs}

\wss{Characterize the problem in terms of ``high level'' inputs and outputs.  
Use abstraction so that you can avoid details.}\\


There are two main components of this project. They are both generative systems, one being a recommendation
system, the other being a pseudo-music generator. There is also a song analysis system, necessary in
order to get the recommendation system and generative systems working as they will most likely operate on 
the data that the song analysis system provides. \\

Inputs:\\
A "track" is defined as a full song. A snippet is defined as a fragment of a track. These are objects that 
contain a sound file and some labels describing the song. 
High-Level: 
For system 1 (recommendation): a list of track(s). \\
For system 2 (music generation): a reference track or snippet. \\
For system 3 (analysis): a singular reference track. \\

Explanation: The main idea behind the inputs is essentially to make them all related to each other as they are
fundamentally connected within the system. This means that roughly, all the files are either lists containing
the track as an object and a snippet is functionally under the hood a track, simply labelled differently for
categorization purposes. This allows the system to share behaviors along the sub systems for 
faster implementation and design. As for format specifically, we can pull these from the spotify API song previews,
API song features. \\

Outputs: \\
For System 1 (recommendation): A list of track(s), indexed with a similarity score. \\
For System 2 (music generation): a snippet. \\
For System 3 (analysis): a breakdown of the main aspects of the track.\\

Explanation: for system 1, it should output a list of recommended tracks and some form of a similarity score
as a way to justify itself. This score could be derived on factors such as genre, sound signature, artist, etc.

For system 2, the snippest is an object containing the sound file. 

For system 3, a score that categorizes the main features of the song, such as the musical range, genre, rythm, etc. \\

\subsection{Stakeholders}

\subsection{Environment}

\wss{Hardware and software environment}

\section{Goals}

\section{Stretch Goals}

\section{Challenge Level and Extras}

\wss{State your expected challenge level (advanced, general or basic).  The
challenge can come through the required domain knowledge, the implementation or
something else.  Usually the greater the novelty of a project the greater its
challenge level.  You should include your rationale for the selected level.
Approval of the level will be part of the discussion with the instructor forx
approving the project.  The challenge level, with the approval (or request) of
the instructor, can be modified over the course of the term.}

\wss{Teams may wish to include extras as either potential bonus grades, or to
make up for a less advanced challenge level.  Potential extras include usability
testing, code walkthroughs, user documentation, formal proof, GenderMag
personas, Design Thinking, etc.  Normally the maximum number of extras will be
two.  Approval of the extras will be part of the discussion with the instructor
for approving the project.  The extras, with the approval (or request) of the
instructor, can be modified over the course of the term.}

\newpage{}

\section*{Appendix --- Reflection}

\wss{Not required for CAS 741}

\input{../Reflection.tex}

\begin{enumerate}
    \item What went well while writing this deliverable? 
    \item What pain points did you experience during this deliverable, and how
    did you resolve them?
    \item How did you and your team adjust the scope of your goals to ensure
    they are suitable for a Capstone project (not overly ambitious but also of
    appropriate complexity for a senior design project)?
\end{enumerate}  

\end{document}