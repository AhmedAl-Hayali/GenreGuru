\documentclass{article}

\usepackage{tabularx}
\usepackage{booktabs}

\title{Problem Statement and Goals\\\progname}

\author{\authname}

\date{}

\input{../Comments}
%% Common Parts

\newcommand{\progname}{GenreGuru} % PUT YOUR PROGRAM NAME HERE
\newcommand{\authname}{Team 8 -- Rhythm Rangers\\
\\ Ansel Chen
\\ Muhammad Jawad
\\ Mohamad-Hassan Bahsoun
\\ Matthew Baleanu
\\ Ahmed Al-Hayali} % AUTHOR NAMES                  

\usepackage{hyperref}
    \hypersetup{colorlinks=true, linkcolor=blue, citecolor=blue, filecolor=blue,
                urlcolor=blue, unicode=false}
    \urlstyle{same}
                                


\begin{document}

\maketitle

\begin{table}[hp]
\caption{Revision History} \label{TblRevisionHistory}
\begin{tabularx}{\textwidth}{llX}
\toprule
\textbf{Date} & \textbf{Developer(s)} & \textbf{Change}\\
\midrule
Date1 & Name(s) & Description of changes\\
Date2 & Name(s) & Description of changes\\
... & ... & ...\\
\bottomrule
\end{tabularx}
\end{table}

\section{Problem Statement}

\wss{You should check your problem statement with the
\href{https://github.com/smiths/capTemplate/blob/main/docs/Checklists/ProbState-Checklist.pdf}
{problem statement checklist}.} 

\wss{You can change the section headings, as long as you include the required
information.}

\subsection{Problem}

\subsection{Inputs and Outputs}

\wss{Characterize the problem in terms of ``high level'' inputs and outputs.  
Use abstraction so that you can avoid details.}

\subsection{Stakeholders}

\subsection{Environment}
We strive to launch an on-premise server operating with a version of Ubuntu server, likely the most recent version, \href{https://ubuntu.com/download/server}{24.04.1}. The server shall respond to and process requests from a web-application front-end, making the service accessible to many different devices, but requiring a network.

\section{Goals}
\begin{table}[h]
    \centering
    \begin{tabular}{|| p{0.45\textwidth} | p{0.45\textwidth} ||}
        \hline
        \textbf{Goals} & \textbf{Importance} \\
        \hline
        The system shall \textcolor{red}{adequately} process and respond to 
        requests involving widely-published music genres, e.g., pop, hip-hop, 
        and rock. & Widely-published music genres have the largest corpus of 
        data that can be used to train the featurization and generation 
        mechanisms of the system, i.e., the system must perform favourably in 
        tasks that it is well-trained on. \\
        \hline
        The system shall generate tabular features that correspond to characteristics 
        of the input song (snippet), akin to those of Spotify, e.g., danceability, 
        instrumentalness, and energy. & Structured tabular data can be rapidly process, 
        making the task of song recommendation more efficient and song generation 
        more explainable. \\
        \hline
        The system shall produce a list of songs that are \textcolor{red}{similar} 
        to a single song provided or a collecetion of songs provided by the user. & 
        This is a core feature of the system. Its inclusion should facilitate users 
        to explore a music genre or \textcolor{red}{``sound''} of interest. \\
        \hline
    \end{tabular}
\end{table}

\section{Stretch Goals}

\section{Challenge Level and Extras}

\wss{State your expected challenge level (advanced, general or basic).  The
challenge can come through the required domain knowledge, the implementation or
something else.  Usually the greater the novelty of a project the greater its
challenge level.  You should include your rationale for the selected level.
Approval of the level will be part of the discussion with the instructor for
approving the project.  The challenge level, with the approval (or request) of
the instructor, can be modified over the course of the term.}

\wss{Teams may wish to include extras as either potential bonus grades, or to
make up for a less advanced challenge level.  Potential extras include usability
testing, code walkthroughs, user documentation, formal proof, GenderMag
personas, Design Thinking, etc.  Normally the maximum number of extras will be
two.  Approval of the extras will be part of the discussion with the instructor
for approving the project.  The extras, with the approval (or request) of the
instructor, can be modified over the course of the term.}

\newpage{}

\section*{Appendix --- Reflection}

\wss{Not required for CAS 741}

\input{../Reflection.tex}

\begin{enumerate}
    \item What went well while writing this deliverable? 
    \item What pain points did you experience during this deliverable, and how
    did you resolve them?
    \item How did you and your team adjust the scope of your goals to ensure
    they are suitable for a Capstone project (not overly ambitious but also of
    appropriate complexity for a senior design project)?
\end{enumerate}  

\end{document}